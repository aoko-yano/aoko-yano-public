\title{虚無・無知・飢餓\\―不安の時代における諸学の綜合―}
\author{Aoko Yano}
\date{2024-09-22}
\usepackage{amsmath,amssymb}
\usepackage{iftex}
\usepackage {pdfpages}
\usepackage {fancyhdr}
\ifPDFTeX
  \usepackage[T1]{fontenc}
  \usepackage[utf8]{inputenc}
  \usepackage{textcomp} % provide euro and other symbols
\else % if luatex or xetex
  \usepackage{unicode-math} % this also loads fontspec
  \defaultfontfeatures{Scale=MatchLowercase}
  \defaultfontfeatures[\rmfamily]{Ligatures=TeX,Scale=1}
\fi
\usepackage{lmodern}
\ifPDFTeX\else
  % xetex/luatex font selection
\fi
% Use upquote if available, for straight quotes in verbatim environments
\IfFileExists{upquote.sty}{\usepackage{upquote}}{}
\IfFileExists{microtype.sty}{% use microtype if available
  \usepackage[]{microtype}
  \UseMicrotypeSet[protrusion]{basicmath} % disable protrusion for tt fonts
}{}
\makeatletter
\@ifundefined{KOMAClassName}{% if non-KOMA class
  \IfFileExists{parskip.sty}{%
    \usepackage{parskip}
  }{% else
    \setlength{\parindent}{0pt}
    \setlength{\parskip}{6pt plus 2pt minus 1pt}}
}{% if KOMA class
  \KOMAoptions{parskip=half}}
\makeatother
\usepackage{xcolor}
\setlength{\emergencystretch}{3em} % prevent overfull lines
\providecommand{\tightlist}{%
  \setlength{\itemsep}{0pt}\setlength{\parskip}{0pt}}
\setcounter{secnumdepth}{-\maxdimen} % remove section numbering
\usepackage{bookmark}
\IfFileExists{xurl.sty}{\usepackage{xurl}}{} % add URL line breaks if available
\urlstyle{same}
\hypersetup{
  hidelinks,
  pdfcreator={LaTeX via pandoc}}
\begin{document}
\maketitle


% 標題
\begin{titlepage}
  \vspace*{10mm}
  \begin{center}
    \noindent{\magicfont{\fontsize{30pt}{30pt}\gtfamily\bfseries Cosmo-Psycho Nomicon}}
    \vfill
    {\magicfont{\gtfamily\bfseries\huge Aoko Yano}}
  \end{center}
  \vspace{20mm}
\end{titlepage}
  % 標題終わり  

\frontmatter

謝辞

\begin{itemize}
\tightlist
\item
  2024-07-27に記載(2024-07-28最終更新):

  \begin{itemize}
  \tightlist
  \item
    この研究は、『アレ
    vol.9』所収の市川遊佐「ラカニアン・アジャイル―『四つのディスクール』から考える中間集団論/組織論としての『スクラム』」を主要先行研究として、サークル〈アレ★Club〉の活動の中で作られた。
  \end{itemize}
\end{itemize}


\tableofcontents % 目次

\mainmatter

\part{この世界の構造について}

\section{緒言}\label{ux7dd2ux8a00}

本稿において、「序論」が「問題設定とそれに対する答え方の構え」を示すものである一方で、「緒言」は「本稿が書かれた時代の知の配置の中で、本稿がどのような位置づけの考察なのか」を示すものだ。本稿は、誰に何をもたらすのか。そして、それを可能にするために何をするのか。

本稿がまず念頭においている読者は、私たちを駆り立てるべくますます騒がしさを増してきつつある世の喧騒に対してうんざりしつつも、実際のところどのように生きるべきかについて思いめぐらせている人々だ。そうした人々にとって、本稿は静寂を取り戻す助けとなることだろう。本稿は、人々が混乱の中に沈むことを防ぐ小舟のような存在となるに違いない。その小舟の上からは、世の喧騒が如何に簡単な無理解に立脚したものであるかがすぐに分かるようになる。そして読者は、本稿を読むことで、本質的に未解決な問題については地道な探究を重ねるしかなく、そこに魔法のような解決があることは稀なのだと、腹の底から納得できるようになるだろう。

このようにして得られる静寂において、読者の前にはこの宇宙の中で、私たちが如何に小さな存在であるかが分かるようになる。そこでは宇宙が、古の人間が思い描いたであろうような意味に満ちたものではなく、茫漠とした無根拠と無意味として現れてくる。

しかし、それにもかかわらず、読者はそこでただ単にすべてから切り離された根無し草になるわけではない。読者はそこで、自身の身体に出会う。その身体とは、自身がそこでそのように感じ、そう欲する以外の仕方では欲することができないような「固着した飢え」である。

飢えは救いではないが、飢えを無視することは苦しみを生む。固着した飢えという立脚点を得て、その渇きを癒すために、幸運ならば、読者は本稿という知識の小舟から無知が渦巻く宇宙という大海へと最後は飛び込むだろう。本稿が提供する知識の体系は、その際に自身の身を混沌から守る一助にもなるだろう。本稿は、そのようなものだ。

そのような議論を、本稿はどのようにして可能にするのか。それは、本稿は複数の専門分野で発見された知識を相互に束縛させあい、接続させあうことによってだ。そうすることで、本稿は複数の専門分野を一つに繋ぎ合わせ、統合的な知の道筋を浮かび上がらせる。その際に、本稿は各専門分野の基本的な知識しか用いない。それは、本稿が示す道筋が、実は現代人の眼前に当然のごとく在り続けていたにもかかわらず、ほとんど誰にも顧みられなかったものだからだ。この意味で、本稿は特定の専門分野に直接寄与するものではないし、文学的なテクストによって人々の情動を揺さぶるものでもない。それらは本稿の目的ではないからだ。同様に、特定の専門分野への寄与や文学的なテクストを期待する読者に、本稿が与えられるものはない。それにもかかわらず、本稿が示す道筋は、現代人の知識を整理することによって、様々な専門分野や文学に従事する人々を含む幅広い層に寄与するだろう。

\newpage

\section{第一部の意義}\label{ux7b2cux4e00ux90e8ux306eux610fux7fa9}

\subsection{問題意識あるいは問いの設定}\label{ux554fux984cux610fux8b58ux3042ux308bux3044ux306fux554fux3044ux306eux8a2dux5b9a}

世界のすべてについて知ることができる立場が、広く人々に与えられたことはない。それゆえに、人々はこれまで常に無知を抱え、その無知ゆえに世界に翻弄され、困惑し、そして不安を抱えてきた。不条理な現実、耐えがたい受苦、取り返しのつかない過去の過ちへの後悔\ldots\ldots これらを浴びせられたときに、それが「無駄(umsonst)」であるなどとは、到底受け入れがたいものだ(註1)。人が不可解な自身の生について、「何も欲さない」でいることは至難の業である(註2)。人は、飢えているのだ。

そこで、人々は「無を欲する」ことにした(註3)。自身の知を超えた彼方に崇高なるものの存在を措定し、そこから惨めな己の生に救いの光を当てることにしたのだ。自然界の、血族の、国家の、人類の神性が信じられ、また超越的な神が信じられるようになった。それらが真に実在するかどうかなどは、人間には知るべくもない。しかし、信じるだけでも、そこに「意味」が与えられ、それによって人の心は救われるのだ。

だが、かつて信じられた物語の多くは、科学的営為の積み重ねによる人類の知の増加によって、寓話として捉えるのでもなければ、あまりにも迷信的なものに見えるようになってしまった。それでも、まだかつて人類が希望を失わない時代があった。

それは、「希望に満ちた未来」という物語が代わりに信じられるようになったからだ。新しく人々の前面に現れた資本主義という「欲求の体系(註4)」においては、貨幣などの通用力を持つ媒介を用いることで、人々の不満と生産力とがボトムアップで調整されながら、ますます人類の力が増していった。そこで必要なのは、暴力的な破壊や盗みを合理的な行動選択として成立させないための安全保障体制だけ、のはずだった。しかし、この動的な物語もまた、瓦解を始めることになった(=「大きな物語の終焉」(註5))。

まず第一に、20世紀後半になると、無尽蔵にも思われた地球の資源および環境が、人類の活動によって大いに攪乱されうるものであり、ナイーブにそれらを収奪し続けていては文明は維持できなくなるかもしれないのだということが分かってきた。人類の果てしない進歩を漫然と信じられる時代は終わってしまった。「存続するための十分な努力ができなければ、廃れ、やがては滅びるかもしれない」という未来が見えてきたのである。

続いて第二に、21世紀に入ると、科学技術は高度な教育なしではとても信じられないような水準に達するようになった。世界と社会のフロンティアは魔術的(註6)になり、そうしたフロンティアを理解するだけのリテラシーがない者は、わけもわからず時代の流れに振り回されるようになっていった。それは新たな「疎外」状況であり、その疎外の中で「社会から見捨てられた」と反感を溜め込む者たちも多く現れるようになった。

そして第三に、2010年代以降、玉石混交の情報がインターネットを通じて社会に氾濫するようになった。前世紀に比すれば世界は概ね豊かになった(註7)
ものの、世界各地の酸鼻極まる悲惨が事実として存在しており、人々はそうしさ悲惨についての情報を毎日のように浴びせられている。悲惨がなくならない以上、再びそこに「なぜ?」という問いが持ち上がる。そして、その問いに対しては「その悲惨の背後に邪悪な陰謀を張り巡らせる権力者がおり、説得など通じないならず者がおり、私服を肥やすことにしか関心のない悪徳商人がおり\ldots\ldots」といった物語が与えられることになるわけだ。もちろん、そうした物語りは気に食わない他者の信用を毀損するための偽情報にすぎない場合が少なくない(なお、本当にそうした陰謀や悪が実在する場合もある)のだが、そうした物語を魅力的に感じる人は後を絶たず、今日ではそうした偽情報の氾濫にまつわる混乱が社会問題となっている。しかも、そこでは混乱した状況に対して「もう何も分からん、とりあえず社会が変化するのは何やら恐ろしいからやめてくれ!」と判断させることを目標にしたメタな連中すらもいる始末だ(=「ディスインフォメーション」(註8))。

要するに、過去より存在する物語を信仰し続けるのには筋金入りの信心が必要であり、ナイーブに未来を信じるのには事態は差し迫りすぎており、真面目に社会を前進させるには人類の平均的知力は既に「落伍しないのがやっと」のレベルになっていて、かつ現に取りこぼされた人々が「救い」として縋りつくのは(世界に溢れる悲惨を直に引き受けるのでなければ)往々にして陰謀論的な「真実」であったりする\ldots\ldots という有様なのだ。

結局、人類は自身の不安と悲しみを慰撫するための新たな物語を渇望し続けている。人間の歴史とは、無を仰ぐ「ニヒリズム」の歴史であった。本論にて詳述するが、人類がさらに進化したとしても、おそらくこれからも―少なくとも当分は―そうだろう。

\begin{itemize}
\tightlist
\item
  (註1) ニーチェ(1974=1984:275-284)\cite{Nietzsche1}参照。
\item
  (註2) ニーチェ(1887=1940:271)\cite{Nietzsche2}参照。
\item
  (註3) ニーチェ(1887=1940:271)\cite{Nietzsche2}参照。
\item
  (註4) ヘーゲル(1821=2001)\cite{Hegel}参照。
\item
  (註5)
  言葉としての「大きな物語の終焉」はリオタール(1979=1989)\cite{Lyotard}で提唱された。その第一部における意味合いとしては松本(2018:15-16)\cite{Matsumoto}を参照。
\item
  (註6) 落合(2015)\cite{Ochiai}参照。
\item
  (註7)
  ロスリング/ロスリング/ロンランド(2018=2019)\cite{RoslingEtAl}参照
\item
  (註8)
  陰謀論とディスインフレーションについては、小泉/桒原/小宮山(2023)\cite{KoizumiEtAl}における記述と、津田(2024)\cite{Tsuda}における「これは権威主義国家でよく使われる手法なんですけど、ウソか本当か分らない情報を、ソーシャルメディアなどを使ってとにかく大量に流すんですね。すると人びとはどの情報を信じればよいかわからず、変化よりも現状維持を選ぶようになると。つまり、権力者が自分たちの体制を維持しようと思ったら、まじめに説得するより、訳のわからない情報を流して思考停止にさせた方が、人びとを楽にコントロールできるというわけです」という記述を参照。
\end{itemize}

\subsection{第一部の目的あるいは目論見}\label{ux7b2cux4e00ux90e8ux306eux76eeux7684ux3042ux308bux3044ux306fux76eeux8ad6ux898b}

第一部には、三つの目標がある。第一部はその三つの目標を達成することによって、先述したような状況の中で人々がより広く「満足」できる社会を構想することを目的としている。

第一の目標は、人類において不満と満足が現れるダイナミズムをモデル化して示すことだ。人はどのようなときに不満を覚え、どのようにして満足を得るのか。その仕組みのどこが必然的なものであり、どこがそうではないのか。それを科学的な知見に基づいて明らかにするのがこの第一の目標だ。そこでは、物語を信じることには不満を解消し満足を獲得する上での様々な有効性があり、それゆえに「信じるに値する物語を獲得すること」が効果的なのだということも説かれる。この目標が達成されることで、以下の第二の目標を目指すことができるようになる。

第二の目標は、同じく科学的な知見に基づいて「新しい物語が満たすべき基準としての『推奨ボーダーライン』」を引くことにある。そのボーダーラインには、二つの要件がある。一つ目の要件は、その物語が科学的な知見との整合性を持つことであり、二つ目の要件は、その物語が人間を「双数的」な関係ではなく「象徴的」あるいは「現実的」関係におくことである(これらの用語の意味は本論で説明する)。ここで、「推奨ボーダーライン」という歯切れの悪い表現をしているのには理由がある。「推奨」という語と「ボーダーライン」という語のそれぞれについて、その語を採用した二つの理由を以下で説明する。

まず第一に、「ボーダーライン」という語を選んだ意味についてだが、これは(少なくとも現時点における人類の)科学的知見は世界の全貌を解き明かすような水準には達していないため、その総体を集成してもそれで世界の全貌について説明してくれる物語が即座に生み出されるわけではないという事情から説明される。すなわち、科学的知見と整合的な物語を志向するにしても、第一部が提示しようとするボーダーラインはあくまでそうした物語の「必要条件」に留まることになるのだ。この必要条件を満たす物語を立ち上げるにしても、第一部が示すボーダーラインの範囲を超えた「肉付け」や「選択」の部分には相当な程度の自由度が残されており、その自由度の扱い方については読者に委ねられることになる。

そして第二に、「推奨」という語を選んだ意味についてだが、これは「そうしたボーダーラインを満たさないことが、直ちにその物語が間違っているということを意味するわけではない」という事情から説明される。科学的知見というのは科学者が発案したストーリーにすぎず、それが永遠普遍的に正しいものであるとは限らない。だから、第一部が提示するボーダーラインを満たさない物語を真理として信仰することは完全に可能なのだ。最初から科学的知見との整合性を度外視した物語であっても、それを真理として信仰するという行為に関しては、何の問題もなく成立する。第一部はこの点について、あくまで「科学的知見との整合性に難のある物語に基づいて生きていくと、そのために実生活上で様々な不都合や不満足が生じやすくなる結果、そこから生を肯定することがより困難な状況に追い込まれる可能性が考えられる」という観点から、それを積極的には推奨しないというだけのことだ。

第三の目標は、以上二つの目標を達成した結果を踏まえて、「社会において複数の世界観が作り出す生態系を如何に構成していくべきか」という問いを立て、この問いに答える必要性を説くことである。そこでは、社会の成員それぞれが選好性を示す生き方のタイプが異なることと、その生き方のいずれかが他の生き方に優越すると論じることはできないを踏まえて、「複数の世界観が、それぞれを選好する成員を包括した上で、相互に交流しつつ棲み分ける」ような一つの生態系が構築されることが望ましいと訴える。そうした生態系が成立することによって、単一の世界観が多様な成員に押し付けられる事態を避けながら、それぞれの世界観の間にある対立を緩和することが可能になる。そこでこそ、人々は自身が信じたいものを信じながら満足をより安心して追求できるようになるはずだ。

\subsection{第一部の構成および序論}\label{ux7b2cux4e00ux90e8ux306eux69cbux6210ux304aux3088ux3073ux5e8fux8ad6}

第一部はこの三つの目標を達成するために、生が持つ可能性を様々な角度から検討することした。そうした検討を行うために、具体的な方法論としては第一部は「専門細分化した諸学のそれぞれから生を検討した上で、それらの結果を統合的に要約する」ことにした。

そこで第一部は、この具体的な方法論に従い、生を検討する際に参照する学問領域ごとに一枚の図と章を切り出すことにした。そして、そのそれぞれの図&章の間に論理的な接続関係を引くことによって、諸学の統合を図ることにした。それぞれの図&章は、第一章から第三章までの三つの部分に分けられた上で、下記のように各学問領域と対応している。

なお、第一章では生命一般について人類という枠を超えて成立する知見を見ていき、生命の限界とその仕組みを理解した上で脳の振る舞いの中に「差異に開かれた弁証法」を見出す。続く第二章では、ラカン派精神分析の知見を見ていき、体験とシニフィアンとの間で形成される安定的な均衡状態としての四つのディスクールに論を進める。そして第三章では、四つのディスクールを軸に人間社会のダイナミズムを理解していく。

まず、図1&第一章第一節は「物理学・化学・生態学」に対応している。そこでは、この宇宙を支配する原則としての物理的なプロセスが提示され、その部分集合としての化学的なプロセスが提示される。そして、その化学的なプロセスに従って長大な時間的空間的スケールで描かれる生物圏(バイオスフィア)の中で、それぞれのニッチェの中に住まい、そしてそのニッチェと共に変化しておくものとして生物種を生態学的に描く。

図2&第一章第二節は「分子生物学」に対応している。分子生物学の知見を踏まえることで、図1&第一章で示した大局的=積分的な変化がどのような局所的=微分的な力学から生じるかを示す。そこでは、各個体の生存を保ちかつ種の進化を可能にするための仕組みとして、「遺伝情報の自己複製」と「遺伝子の発現制御」と「世代交代」の三つの機能が説明される。この章の記述を前提として、図3&第三章以降の記述が進められることとなる。

図3&第一章第三節では、近年注目を集めている脳科学の学説である「自由エネルギー原理」に基づいて生物の思考と行為が従う原則を説明する。そこで示される「予想」と「予想誤差」をめぐる簡潔な数式は、生物の思考と行為が持つ弁証法的なあり方を表現している。

この弁証法的なあり方を蝶番として、図4&第二章第一節と図5&第二章第二節では「ラカン派精神分析」を用いた自然科学的な世界説明から意味と生の観点からの世界説明へと移る。この架橋は、予想と予想誤差との関係が「シニフィアンの体系」と「(シニフィアンの体系による象徴化を逃れた)残余」との関係に対応していると解釈することによりなされる。続く図6&第二章第三節から図8&第二章第五節までは、ラカン派精神分析の諸概念を第一部の趣旨に関わる最低限のレベルで説明してある。具体的には、ラカン派精神分析における「神経症」と「精神病」の違いと、神経症的な心的構造が確立されるまでの「エディプス・コンプレックスの成立過程」と、神経症者の取る思考と行為のタイプ分けとしての「四つのディスクール」が順に説明される。

図9&第三章第一節では、ラカン派精神分析における四つのディスクールと広い意味でのエンジニアリングが接続される。この接続は「自然を制御する仕組みだけではなく文化や社会制度に至るまでの幅広い範囲の制作物が人間の認識に沿ったものであり、そして人間の認識は四つのディスクールの各局面が現れるのに伴って構築あるいは解体されるのだから、それらの制作物が生み出されたり廃棄されたりする過程もまた四つのディスクールによって記述される」という発想に基づいている(註1)。

図10&第三章第二節では「エンジニアリングの過程に伴い人間の認識と制作物が複雑に組み合わさっていくと、その前提的な役割を果たしている部分は容易には変えられなくなってしまう」ということを指摘した上で、その硬直が四つのディスクールの各局面にそれぞれ新たな効果を付与することを指摘する。この状態は、社会が「プレモダン(註2)」の段階に至ったことを意味している。そこでは、国家や宗教の権威が強力な力を持ち、人々は官僚主義的なヒエラルキーの中で生きていくことになる。

図11&第三章第三節では、このプレモダンの段階から、貨幣などの媒介によって権威が宙吊りにされて、「モダン(註3)」な資本主義に移行した後の社会について記述する。そこでは、「労働者」あるいは「資本家」が生産活動を通じて貨幣や資本を増大させようとする側面と、「消費者」の持つ不満が商品の購入によって速やかに解消される側面とが描かれる。また、この二つの側面が社会を秩序付ける主な力として台頭する過程でプレモダンな権威が相対的に力を失うことによって、人々が異質な他者に対して耐える力が弱まり、「レイシズム」などの差別が勃興してくることを指摘する。

ここまでの議論を踏まえ、「諸学の綜合」の章となる第四章では第一部の三つの目標を達成していく。すなわち、第一に人間の不満と満足が現れるダイナミズムをモデル化し、第二に「新しい物語が満たすべき基準としての『推奨ボーダーライン』」を示し、第三に(その新しい物語を含めた)多様な物語を社会の中で共存させる必要性について検討する。

\begin{itemize}
\tightlist
\item
  (註1) 市川(2021)\cite{Ichikawa}参照。
\item
  (註2) 浅田(1983)\cite{Asada}参照。
\item
  (註3) 浅田(1983)\cite{Asada}参照。
\end{itemize}

\newpage

\chapter{生命一般を縛る限界から脳を支配する弁証法へ}

\section{生命と環境}\label{ux751fux547dux3068ux74b0ux5883}

\subsection{議論の進め方と構図の取り方について}\label{ux8b70ux8ad6ux306eux9032ux3081ux65b9ux3068ux69cbux56f3ux306eux53d6ux308aux65b9ux306bux3064ux3044ux3066}

まずは、私たちが生きる宇宙がどのようにできているかを、物理化学的な観点からおさらいしていく。このように述べると、人によっては「宇宙や地球、あるいは生命について『理系』的な話をすることが、生き方や社会のあり方などを考えるという『文系』的な話にどう繋がるのかピンとこない」と思うかもしれないが、その疑問にはさし当たりこう答えておこう。すなわち、宇宙・地球・生命についての知見を追うことで、我々自身の生の輪郭やその将来の可能性、さらにはその限界を「大外から掘り出していく」ことが可能になるのだと。そして、この「大外から掘り出していく」スタンスによって、議論をクリアなものにすることができるのだと。

議論に宇宙・地球・生命についての知見を盛り込むことの意義はひとまず上記の通りだが、それでも議論をどこから始めるのかという点については選択の余地が残る。この点について第一部は、我々の生に直接与えられる所与から議論を始めるのではなく、先人たちの努力によって構築された客観的な世界についての説明を引き回すとことから議論を始める。それは「宇宙の中にいる我々が、宇宙を観測する」という構図を採用することで、議論をより平易にするためだ。我々の生に直接与えられる所与の観点から見た議論は、後の「04-体験とシニフィアン」以降で、「01-生命と環境」から「03-脳と自由エネルギー原理」までの議論と接続する形で行う。

\subsection{物質・物・生命}\label{ux7269ux8ceaux7269ux751fux547d}

さて、物理化学的な観点から見ると、宇宙は一つの「場」であり、その中には複数の「物質」がひしめき合っている(\#1.1)。物質の振る舞いは、他の物質との関係によって十分に規定される(物質以外の要素によって物質の振る舞いが左右されることはない)。物質の振る舞いを説明する際には、物質は他の物質に対して力を及ぼしているとされる。つまり、物質が他の物質に力を及ぼすことで、力を受けた側の物質は、受けた力に応じて振る舞いを変えるのだ。こうして物質が相互に力を及ぼし合うことで、一部の物質は相互に組み合わさって、一つの「物」としてまとまるようになる(\#1.2)。なお、「なぜ宇宙が存在するのか?」という問いに対して、宇宙を構成する場や物質を用いて回答することはできない(\#1.3)。なぜなら、宇宙を構成する場や物質を用いた回答は「(宇宙の中にある)Aは、B(という性質)だから」という形式しかとることはできないため、存在する理由を答えることはできないからだ。

\begin{note}{}
  \begin{itemize}
    \tightlist
    \item{\#1.1}複数の「物質」が一つの「場」の中で相互作用をしながら運動する。
    \item{\#1.2}一部の物質は相互に組み合わさり、一つの「物」としての挙動を示す。
    \item{\#1.3}なぜ物質や場が存在するのか(=「世界が存在する理由」)は、人間には回答できない。
  \end{itemize}
\end{note}

さて、物質が他の物質に力を及ぼすことができる度合いを「エネルギー量」と呼ぶ。力を及ぼした側の物質はその力の分だけエネルギーを失い、力を受けた側の物質はその力の分だけエネルギーを得る。宇宙の始まりにおいて、物質はそれぞれ異なるエネルギー量を持っていたが、相互に力を及ぼし合う中で物質が持つエネルギーは平準化されていく。これを「エントロピー(乱雑さ)の増大」という。例えば、一つの容器の中に一枚の仕切り板を入れ、その仕切りを隔てて暖かい水と冷たい水を入れておくとする。暖かい水はエネルギー量の多い水であり、冷たい水はエネルギー量の少ない水である。この状況下でこの仕切り板を外すと、暖かい水が持っていたエネルギーは冷たい水へと移り、全体がぬるい水になる。これがエントロピーの増大である。

物にとってのエントロピーの増大は、その構造の崩壊を最終的には意味している。ある物のエントロピーが最大となるのは、その物の中にある物質をすべて乱雑に混ぜ合わせ、そこで起こり得る化学反応(これも力のやり取りの一形態であり、エネルギーの平準化の一形態である)をすべて起こさせた場合だからだ。ここから翻って考えると、物の中に構造があるのは、その物のエントロピーがまだ最大になっていないからであり、その物のエントロピーを最大化させると、そこにある構造は崩壊する他ないのだと分かる。

宇宙のエントロピーは増大し続けているが、局所的に見た場合、ある物の中でエントロピーがそこまで増大しない(あるいは、減少する)ことがある。それは、その物が「自身を存続させる仕組み」を持っているからだ(\#1.4)。そうした物は、自身の外部からエネルギー量の多い物質を取り込み、自身の中でのエントロピーの増大を相殺するために取りこんんだ物質のエネルギーを使い、エネルギーを失った物質を自身の外に排出することができる。このような「自身を存続させる仕組み」を持つことが、生命の物理化学的な本質である(\#1.5)。例えば、人は食事により他の生物の死骸を取り込み、その死骸を消化液で分解することによって吸収可能な養分に変換した上で、養分を消化管から吸収する。吸収された養分の一部は、呼吸により得られた酸素と結びつくことで力を解放し、生体内の様々な化学反応に用いられるエネルギー通貨としてのATP(アデノシン三リン酸)という高エネルギー物質の合成に用いられる。また、吸収された養分の別の一部は、生体内の様々な化学反応に用いられる素材となる。こうして養分から取り出されたエネルギーと素材は、生体を動かす用途だけではなく、生体が負った傷を修復するためにも使われる。このような一連のプロセスを通じて構造を維持することで、生体はエントロピーを増大させることを避け、存続することができるわけだ。ここで、エネルギーを奪われた養分の残り滓や、生体内の化学反応の果てに生み出された不要な物は、生体外に排出されることになる。

\begin{note}{}
  \begin{itemize}
    \tightlist
    \item{\#1.4}物の中には、「自身を存続させる仕組み」を持つものがある。
    \item{\#1.5}自身を存続させる仕組みを持つことが、生命の本質である。
  \end{itemize}
\end{note}

もちろん、生命それ自体のエントロピーが少なく保たれている間も、生命が取り込んだ物と生命が排出した物とを勘定に入れると、それら全体でのエントロピーはしっかり増大している。それゆえに、生命は物理化学的な法則には何も矛盾していない。これは、エアコンを働かせると、部屋の中は外気温よりも涼しく保つことができる(つまり、エントロピーを低く保つことができる)一方で、エアコンを稼働させるための電気は発電所でタービンを回して作られるのと同様である。発電所でタービンが回される際には、水の落下や燃料の燃焼による水蒸気の発生などにより、必ずエントロピーの増大が起きているのだ。

\subsection{場・環境・ニッチェ}\label{ux5834ux74b0ux5883ux30cbux30c3ux30c1ux30a7}

ある場において、その中にある物が他の物と相互作用をしているとき、その場をその物にとっての「環境」と言い表そう(\#1.6)。そのとき、生命が「自身を存続させる仕組み」を作動させてエントロピーの増大に抗うことができるのは、「環境内に存在する多様な物が、相互作用を通じて形成する布置」の中の特定の場所に限られる(\#1.7)のだとわかる。例えば、地球上の生命はすべて地球という環境の中で生きてきたわけであり、通常はそれぞれの「生存に適した場所」でのみ生存し続けることができるわけだが、その「生存に適した場所」というのは環境内の他の生命などとの関係性から規定されるのだ。実際、人類は酸素濃度が20\%前後の1気圧程度の環境下でしか生き続けることができないが、大気中に酸素が持続的に存在し続けているのは植物などが光合成を通じて酸素を供給し続けているからにほかならない。

\begin{note}{}
  \begin{itemize}
    \tightlist
    \item{\#1.6}この場を「環境」ともいう。
    \item{\#1.7}自身を存続させる仕組みは、環境の中にある特定の「ニッチェ」の中で作動する。
  \end{itemize}
\end{note}

このような「ある環境において、様々な生命などが形成する布置の中に作られる、ある生命が持続的に存在し続けることを可能にする部分」のことを、その生命のその環境における「ニッチェ(=生態学的地位)」と呼ぼう(\#1.8)。すると、一つの環境は複数の生命のための複数のニッチェを持つことが可能だということがわかる(\#1.9)。例えば、地球という環境上にも様々な生命のための様々なニッチェがある。人類のニッチェには先ほど述べたような大気が存在することが必要であるが、一方で偏性嫌気性菌というタイプの生命は同程度の酸素に曝されると死んでしまうことから、偏性嫌気性菌のニッチェには酸素が乏しいことが必要なのだと分かる(なお、偏性嫌気性菌は珍しい存在ではない。事実、人類の腸内に棲息するビフィズス菌もまた、偏性嫌気性菌だ)。

\begin{note}{}
  \begin{itemize}
    \tightlist
    \item{\#1.8}ニッチェとは、その中の環境がほぼ同一の状況を保てるような環境の一部分のことである。
    \item{\#1.9}環境はニッチェを複数持つ。
  \end{itemize}
\end{note}

また、それぞれのニッチェの大きさや形や性質などは、環境がその内外から受ける様々な作用によって時間的に変化していく(\#1.10)。そうした作用の中には、地球惑星科学的なものだけでなく、生物学的なものもある(\#1.11)。まず、地球惑星科学的な作用がニッチェを変えた例としては「大規模な火山の噴火や隕石の衝突が地球の気候を大きく変動させ、様々な生命が生きてきたニッチェが失われたことでそれらを絶滅に追いやった一方で、それらの生命がいなくなった場所は残された生命が進出してニッチェとする新たな余地にもなった」という話を挙げることができる。また、生物学的な作用がニッチェを変えた例としては「地球上の酸素濃度は、地球誕生からしばらくのうちはゼロに近かったものの、光合成を行う生命がその活動範囲を拡大させることで現代の水準へと増加してきたことで、酸素呼吸を行なう生命が地球上の様々な場所でニッチェを獲得できるようになった」という話を挙げることができる。

\begin{note}{}
  \begin{itemize}
    \tightlist
    \item{\#1.10}環境の変動に合わせてニッチェも変化する。
    \item{\#1.11}生命の活動が、環境を変動させることを通じて、ニッチェを変化させることがある。
  \end{itemize}
\end{note}

\subsection{同一性の維持と変化}\label{ux540cux4e00ux6027ux306eux7dadux6301ux3068ux5909ux5316}

このように、生命が持つ「自身を存続させる仕組み」によって、生命はニッチェからエネルギーを取得して自身における同一性を維持してきたが(\#1.12)、その一方で、生命はニッチェの変動に応じて変化してもきた(\#1.13)。例えば、ガラパゴス諸島に棲息するダーウィンフィンチ類という鳥類は、他の島々を伝って200万年から300万年前ほどに絶海の孤島であるガラパゴス諸島に渡来したものが、環境中に新たなニッチェを獲得していったことで多様な形態へと分かれていったものだと考えられている(=「適応放散」)。ダーウィンフィンチ類は、食物とする生物に合わせて様々な嘴を持っているが、それらは系統的には近縁であることから、元は単一の形態であったものが複数の形態へと分かれていった結果だと考えられているのだ。ダーウィンフィンチ類の例に限らず、ある生命が絶滅して空いたニッチェへと進出してきた生命や、酸素濃度の増大によって行動範囲を広げた生命も、その新たなニッチェに合わせて様々にその姿を多様化していったのだと考えられている。

\begin{note}{}
  \begin{itemize}
    \tightlist
    \item{\#1.12}自身を存続させる仕組みは、ニッチェからエネルギーを取得し、自身における同一性の維持のために用いる。
    \item{\#1.13}生命は、ニッチェの変動の影響を受けて変化する。
  \end{itemize}
\end{note}

これはつまり、生命とニッチェは互いに相互作用を及ぼしあっており、どちらかが他方を一方的に規定するという関係にはないということだ。生命がニッチェを変え、ニッチェが生命を変える。両社は影響を及ぼしあいながら共に変化してきたのだ。

ここまでの議論を総括すると、生命は「同一性を維持」をしつつも「変化」しているということが分かる。地球の生命においては、この二つの相反する機能は、相互に絡み合った三つの仕組みによって実現されている(\#1.14)。次章では、その三つの仕組みについて見ていくことで、生命が持つ可塑性について具体的に掘り下げていく。

\begin{note}{}
  \begin{itemize}
    \tightlist
    \item{\#1.14}地球の生命は、「同一性の維持」と「変化」を両立させる仕組みを持つ。 これらを両立させる仕組みは、下記の三つに大別できる。
  \end{itemize}
\end{note}

\subsection{布石:地球上の生命の死・宇宙の終わり}\label{ux5e03ux77f3ux5730ux7403ux4e0aux306eux751fux547dux306eux6b7bux5b87ux5b99ux306eux7d42ux308fux308a}

次章からは同一性の維持と変化を成り立たせる三つの仕組みへと視点をズームインしていくが、その前に一度、視点をズームアウトして地球上の生命と宇宙が辿る運命について見ていこう。そうすることで、我々が置かれている「真の位置」に自覚的になることができ、そこから我々が答えなければならない問いが浮かび上がってくるからだ。その問いは、第一部が追い求めている「新しい物語」が答えなければならない問いである。

まず、地球上の環境は永遠に一定であるわけではない。地球上の環境は、主に太陽からのエネルギーを受け続けることで一定に保たれているのだが、そのエネルギーが一定ではないからだ(この他にも、地中からのエネルギー供給が一定ではないという現象や、地球の大気が少しずつ散逸しているという現象もあるのだが、その影響は太陽エネルギーの変化と比べてゆっくりとしか現れないため、ここでは無視する)。太陽からのエネルギーは、太陽がその中心核で起こしている核融合に源泉を持つ。その燃料は中心核に存在する水素なのだが、その水素の量が有限なのだ。中心核に存在する水素が減少すると、中心核が縮んで温度が上がり、核融合の出力が上がる。そのため、太陽は一億年に約1\%の割合で光量を増加させることになる。この影響は地球の表面温度に現れる。表面温度が上がった地球では二酸化炭素が岩石と化学反応を起こして吸着され、大気中の二酸化炭素濃度が減少する。それに伴って酸素光合成を行う植物が絶滅していき、約10億年後には大気中の酸素濃度がほぼゼロになる。その結果、酸素呼吸を行う生物はすべて絶滅することになる。酸素に依存しない単細胞生物は海底や岩石中などでも生存を続ける可能性が高いが、それらも太陽中心核から水素が尽きるよりもはるか先に高温によって絶滅することになる。

太陽中心核の水素が尽きるのは今から約60億年後だと考えられている。その後も太陽はわずかな期間にわたってヘリウムなどの核融合の産物を燃料とした核融合を起こすことになるが、それらの燃料も尽きると太陽は外層部のガスを放出して白色矮星という中心核の燃え滓を残すことになる。その後の太陽はただひたすら冷えていき、最後は光を発しない黒色矮星になる。

核融合の停止に伴うこのような変化は太陽以外の恒星でも起きていき、その恒星を形成していた物質の量の違いによって残るものが黒色矮星ではなく中性子星やブラックホールである場合もあるが、宇宙はやがてこれらの「コンパクト天体」ばかりが漂う空間になっていく。

そして、それらの天体を存在させている宇宙がひたすら加速度的に膨張していく中で、コンパクト天体も(衝突や合体などが起こる場合はあるかもしれないが、それでも)やがてはより小さな素粒子へと「蒸発」し、冷たく暗い闇の中に溶けていくと考えられている。これが「宇宙の終わり」だ。

たしかに、人類や人類に縁のある生命が太陽の死を超えて生きながらえる可能性は、現在の時点では否定できない。工学や生物学の分野において何が実現できるかの予測は、そこで予測される対象があまりに複雑であるがゆえに、少なくとも現在の技術では不可能だからだ。未来においては、物理法則に反しない限りで、様々なことが起こり得る。

だが、それでも宇宙の終わりから逃れることはできない。もちろん、現在の我々による未来予想が間違っており、宇宙に終わりがない可能性もありうる。だが、第一部では悲観的に「宇宙には終わり」があると前提した上で、それでも「人間の無知と不安に分かちがたく紐づいた、不可解で理不尽なこの現実の生」において「満足」を得る方途について考えていくことにしよう。

\subsection{この章のまとめ}\label{ux3053ux306eux7ae0ux306eux307eux3068ux3081}

エントロピーを増大させながら滅びに向かっていく宇宙の片隅で、エネルギーを取り込み続けることで生命は変化しつつも維持されていく。それゆえに生の系譜は有限だ。これが、我々が置かれている「真の位置」だ。そこに、一体どのような(不満と)満足がありうるのか。これが、第一部は答えようとする問いだ。

\newpage
\includepdf[pages=-, scale=0.95, pagecommand={\thispagestyle{plain}}]{../figures/図1.pdf}

\section{同一性と変化}\label{ux540cux4e00ux6027ux3068ux5909ux5316}

\subsection{細胞・遺伝情報・セントラルドグマ}\label{ux7d30ux80deux907aux4f1dux60c5ux5831ux30bbux30f3ux30c8ux30e9ux30ebux30c9ux30b0ux30de}

地球上の生命の身体は、細胞という単位からできている。細胞は「DNA(デオキシリボ核酸)」という化学物質を中に持っているが、そのDNAは四種類の「ヌクレオチド」という単純な化学物質が並んで結合することでできている。その四種類のヌクレオチドの配列パターンによって、その細胞がどのように「自身を存続させる仕組み」を作動させるかの大枠が決まる。具体的には、DNAをもとに「RNA(リボ核酸)」という化学物質が生成され(=「転写」)、そのRNAを元にタンパク質が生成される(=「翻訳」)。タンパク質は約20種類のアミノ酸という単純な化学物質が並んで結合してできており、生体内の様々な化学反応において触媒として作用する。つまり、生体内の化学反応を司るタンパク質がどのようなアミノ酸の配列パターンを持つかが、元を正せばDNAにおけるヌクレオチドの配列パターンによって決まっているわけだ。その意味で、DNAは生体内の化学反応を規定している。

細胞は自己複製を行って増殖するが、このとき、DNAが持つ配列パターンも「複製」され、細胞の複製に伴って細胞が行う生体内の化学反応のあり方も複製されることになる。こうした事情から、DNAが持つ配列パターンは「遺伝情報」と呼ばれる。遺伝情報の総体を「ゲノム」と呼ぶが、これは通常、細胞内に含まれるすべてのDNA分子の配列パターンの総体のことを指している。このようにして行われる遺伝情報の複製が、第一章の最後に述べた「同一性の維持」のための仕組みだ(\#2.1)。

\begin{note}{}
  \begin{itemize}
    \tightlist
    \item{\#2.1}同一性の維持の仕組みは、「遺伝情報の複製」である。
  \end{itemize}
\end{note}

ここまで述べてきたような「DNA→RNA→タンパク質」というゲノムから生体内の化学反応に至るまでの情報の流れと、「DNA→DNA」というゲノム自体の複製の流れを合わせて、「分子生物学のセントラルドグマ」という。地球上の生命は、基本的にセントラルドグマに沿った仕方で「自身を存続させる仕組み」を作動させている。なお、セントラルドグマに沿わない化学反応も存在している。例えば、「RNAの配列パターンに従ってDNAが作り出され(=『逆転写』)、そのDNAがある生物のDNAの中に挿入される」という現象もある。生命はニッチェとの相互作用の中から場当たり的に変化して生まれてきたものなので、単純な原理原則から逸脱する例外に満ちている。

\subsection{遺伝子と発現制御}\label{ux907aux4f1dux5b50ux3068ux767aux73feux5236ux5fa1}

このように書いたものの、生体内で合成されるタンパク質を構成するアミノ酸の配列パターンに対応しているのは、ゲノム中の配列パターンのうちの一部にすぎない。ゲノム中のタンパク質の配列パターンを直接規定している部分のことを、特に「遺伝子」と呼ぶ。

ゲノム中の遺伝子ではない部分は何をしているのかの全貌はまだ分かっていないが、その少なくない部分が「どのような条件下で、どの遺伝子からタンパク質を合成するか」を規定している。遺伝子からタンパク質が合成されることを、その遺伝子が「発現する」と表示し、そのタイミングなどを制御することを遺伝子の「発現制御」という。この発現制御が、生命に「変化」をもたらす二つの仕組みの一つだ(\#2.2)。

\begin{note}{}
  \begin{itemize}
    \tightlist
    \item{\#2.2}変化の仕組みの一つ目は、遺伝子の「発現制御」である。
  \end{itemize}
\end{note}

遺伝子とその発現制御について理解すると、日常生活の様々なことについても納得しやすくなる。例えば、いわゆる「お酒に強いか否か」は、各個人がその細胞のうちに持つアルコールデヒドロゲナーゼというアルコールを分解するタンパク質に関わる遺伝子上の配列パターンに、その原因の大部分を求めることができると分かっている。また、「筋トレをすると筋肉量が増える」のは、筋トレにより細胞内でアクチンとミオシンという筋肉の伸縮に関わるタンパク質がより多く合成されるようになったり、筋繊維の外側に張り付いているサテライト細胞という細胞が増殖して筋繊維を増やすようになるからだということが分かってきている。

このように、遺伝子を含むゲノムだけで身体がどのようになるかが決まるわけではないし、育った環境だけでそれが決まるわけでもない。良く話題に上がる「生まれか育ちか」という議論の背後には、このような生物学的プロセスがある。

\subsection{突然変異と種}\label{ux7a81ux7136ux5909ux7570ux3068ux7a2e}

どの遺伝子がどのように発現するのかは細胞が置かれた状況によって規定されているわけだが、そもそも「アルコールデヒドロゲナーゼには、アルコールをよく分解できるものと、うまく分解できないものがある」といったDNAの配列パターンの多様性はどこから現れるのか。その原因は、DNAに損傷が入ったり、あるいはDNAが複製される際に正しく複製されなかったりすることで変異が入ること(=突然変異)に求められる(\#2.3)。これが、生命に「変化」をもたらす二つの仕組みのもう一つだ。

\begin{note}{}
  \begin{itemize}
    \tightlist
    \item{\#2.3}変化の仕組みの二つ目は、遺伝子の「突然変異」である。
  \end{itemize}
\end{note}

ところで、通常では突然変異はすぐに修復される(\#2.4)のだが、この突然変異が次の時代まで残留することもある。具体的には、細胞一つで「個体」として生きていくことが前提となっている単細胞生物(例えば、細菌)の場合では、ある個体に入った変異がそのまま分裂などによって多くの個体に増えていくことになるし、同一の遺伝情報を持った細胞が複数集まった状態で「個体」を形成して生きていくことが前提となっている多細胞生物(例えば、人類)の場合では、親個体に入った変異が子孫へと伝わっていくことになる。これらのプロセスを通じて、様々な遺伝情報を持つ個体が現れるようになる。

\begin{note}{}
  \begin{itemize}
    \tightlist
    \item{\#2.4}遺伝情報は、刺激により棄損されることがあるが、生命にはそれを修復する能力がある。
  \end{itemize}
\end{note}

個体間の遺伝情報の違いが大きくなると、それらは互いに似た形態をしなくなっていったり、似た生態をとらなくなっていったり、あるいは近しい個体間で成立する関係性が成立しなくなったりする。最後に挙げた「近しい個体間で成立する関係性」の代表的な例が、「交配して子孫を残せるか、生まれた子孫は親同様に他の近しい個体と交配してさらなる子孫を残すことができるか」というものだ。例えば、イヌとネコの間には子供はできないし、雄のロバと雌のウマから生まれるラバは子供を残すことができない一代限りの動物だ。このような、ある程度の違いの幅を持った複数の類似した個体の集合を「種」という。なお、地球上の生命一般に対して「何をもって同じ『種』とみなすか」を規定する明確な統一的な見解は存在しない。

ここまでの議論で、生命が持つ「同一性の維持」と「変化」のための三つの仕組みについて、その概略を説明できた。生命は、基本的には同一性を維持する仕組みに従って生存していく。だが、ある個体から受け継がれる遺伝情報は、突然変異による時間的な揺らぎを受けていく。その揺らぎの結果として、ある個体たちと他の個体たちとの間に種の違いという境界線が生じてくることになる。その上、各個体はその生の中で発現制御を通じて周囲の環境から影響を受け、変化していく。これらの仕組みによって、生命は、その生命が生きるニッチェと共に、揺れ動く環境の中で変化していくのだ。

この議論を前提として、次章からは、「ある人間個体に対して、遺伝子の発現制御によって後天的な変化を与える仕組み」の一つである「脳」の振る舞いへと議論の対象をズームインしていく。脳は、その発達を通じて環境への特に繊細な適応を可能にする。この脳の振る舞いを見ていくことで、人間の思考がどのような制限下にあるのかの輪郭を浮き上がらせることができるだろう。

\subsection{布石:種にとって寿命とは何か}\label{ux5e03ux77f3ux7a2eux306bux3068ux3063ux3066ux5bffux547dux3068ux306fux4f55ux304b}

だが、ここで第一章と同様に、視点を一度ズームアウトして「ある種にとって寿命とは何か」という問いについて考えておく。それは、そうすることが寿命を持つ種であるヒト(ホモ・サピエンス)を生きる私たちがいずれ死ぬ理由について考えることにもなるからだ(註1)。

「種にとって寿命とは何か」を考えるために、寿命がない種を想定してみよう。寿命とは誕生後の時間経過と共に細胞の機能が低下していくことだと言えるので、ここではそうした機能低下が存在しないと前提してみる。すると、その種の個体は事故や病気以外では死亡しないことになる。このように想定すると、ここでは時間経過による細胞の機能低下が存在しないと想定しているので、年齢の低い世代からだけでなく、年齢の高い世代からも子孫が生まれることになる。この想定では、新しい時代に形成された遺伝情報だけからではなく、古い時代に形成された遺伝情報からも、新しい世代の遺伝情報が形成されることになる。そのため、種に新たに供給される遺伝情報は、現実世界のものと比べて、全体としてはより「古い」遺伝情報に偏ることになる。

これだけだと特に大きな問題にはならなさそうだが、ここに「ニッチェが許容する個体数の上限」を考慮の対象に追加すると、どうだろうか。ニッチェには許容する個体数に上限がある。例えば、種に供給される食料は、捕食される他の生物の量によって律速される。したがって、ある特定の種の個体数の上限値は寿命の長さにかかわらずそれほど変えられないということがわかる。この点を加味すると、寿命がない種では、寿命で死ぬ個体数が存在しない分だけ新たに出生する個体数が減っていなければならないことになる。さもなくば、その種は飢餓に見舞われてしまうだろう。

寿命がある場合と比較して「出生数が減り、さらに出生する新しい世代の遺伝情報がより『古い』ものに偏っている」となると、その種が持つ遺伝情報にもたらされる変化のスピードはより遅くなってしまう。新しい世代を生むのが仮に集団内の若い世代に限られる場合(つまり、生殖能力だけは老化すると仮定した場合)では、上記の事情は「出生数が減る」というだけなるので緩和されるが、その場合でも寿命がある場合よりは種の遺伝情報が変化するスピードは遅くならざるを得ない。

種の遺伝情報の変化速度が低下するということは、その種に属する個体の身体が変わる上限速度も低下するということを意味している。これは、急激な環境変動などに対応して身体を変化させることがより難しくなるという点で、生存に不利になる可能性がある。

ある種に寿命が存在することに上記のような背景があるのだとすれば、ここから翻って「寿命」と「世代交代」が、種の遺伝情報に変化を繰り込む手段の一つなのだと考えることができる(\#2.5)。世代交代は新たな遺伝情報が種に組み込まれる機会なのであり(\#2.6)、寿命が古い世代を排除することで、古い世代の遺伝情報が種から排除され、それが新たな遺伝情報が種に入り込む余地になるのだ(\#2.7)。種の遺伝情報には流動性が繰り込まれており、その流動性が環境とニッチェの中での種の変形を可能にする。

\begin{note}{}
  \begin{itemize}
    \tightlist
    \item{\#2.5}「寿命」と「世代交代」は、種の遺伝情報に変化の機会を繰り込む手段の一つである。
    \item{\#2.6}世代交代により、新たな遺伝情報が種に繰り込まれる。
    \item{\#2.7}寿命により、古い世代の遺伝情報が種から排除される。
  \end{itemize}
\end{note}

人類が持つ寿命も、そうした種に流動性を繰り込む仕組みに起源を持つと考えられる。それゆえ、人類から寿命による死を遠ざけることには、ヒトという種の遺伝情報に変化を繰り込む機会を減少させることを意味するだろう。その場合、遺伝情報の変化という観点からみれば、ヒトはより硬直した、環境とニッチェの変化に対して脆弱な存在になる可能性が考えられる。

もちろん、「世界がどのようになっているか」という事実は「人がどのように生きるべきか」という当為を一切導かない。また、ここでも第一章と同様に、技術的な工夫が隘路を切り開く可能性は否定できない。それでも、この事実を認識することで、それに逆らうことにいかなる困難が伴うのかを明確に認識できるようになる。このような困難についての認識は、私たちが自らの生き方を構想し、また世界についての納得できる物語を作り上げる際に、野放図に空想が膨らむのを避ける役に立つことだろう。

\begin{itemize}
\tightlist
\item
  (註1)
  以下の議論については主に小林(2021)\cite{Kobayashi}を参考にした。
\end{itemize}

\subsection{この章のまとめ}\label{ux3053ux306eux7ae0ux306eux307eux3068ux3081}

環境の変化に合わせて生き方を変え続けるために、生命は自身に変化を繰り込む仕組みを持つ必要がある。そうした変化を繰り込む仕組みとして、ヒトは脳を持ち、寿命を持つ。こうして、ヒトの有限性に寿命という束縛が加わりつつも、脳が可能にする柔軟性も与えられる。第一部は、この条件を踏まえた上で探求される。

\newpage
\includepdf[pages=-, scale=0.95, pagecommand={\thispagestyle{plain}}]{../figures/図2.pdf}

\input{03-第一章第三節}
\newpage
\includepdf[pages=-, scale=0.95, pagecommand={\thispagestyle{plain}}]{../figures/図3.pdf}

\chapter{体験とシニフィアンとの弁証法が形成する四つのディスクール}

\section{体験とシニフィアン}\label{ux7b2cux56dbux7ae0ux4f53ux9a13ux3068ux30b7ux30cbux30d5ux30a3ux30a2ux30f3}

あ

\subsection{この章のまとめ}\label{ux3053ux306eux7ae0ux306eux307eux3068ux3081}

あ

\newpage
\includepdf[pages=-, scale=0.95, pagecommand={\thispagestyle{plain}}]{../figures/図4.pdf}

\section{\texorpdfstring{対象\(a\)と欲動の主体}{対象aと欲動の主体}}\label{ux5bfeux8c61aux3068ux6b32ux52d5ux306eux4e3bux4f53}

\subsection{予測誤差、反復強迫、欲動}\label{ux4e88ux6e2cux8aa4ux5deeux53cdux5fa9ux5f37ux8febux6b32ux52d5}

予測との誤差が大きい体験、すなわち言語化されていない体験、経験に昇華されていない体験、内的体験、トラウマ、〈物〉は、その体験自体に中毒性があるため、脳において「反復」される。これを「反復強迫」と呼ぶ。この反復強迫には「享楽」が伴う。

この反復強迫を通じて、体験を予測できるようにしようとする機制が働き、これを「欲動」と呼ぶ。体験が予測できるようになって欲動が解消されると、満足がもたらされる。

\begin{note}{}
  \begin{itemize}
    \tightlist
    \item{\#5.1}予測との誤差が大きい体験
      \begin{itemize}
        \tightlist
        \item (=言語化されていない体験)
        \item (=経験に昇華されていない体験)
        \item (=「内的体験」)
        \item (=「トラウマ」)
        \item (=「〈物〉」)
      \end{itemize}は、その体験自体に中毒性があるため、脳において「反復」される(=「反復強迫」)。
    \item{\#5.2}反復脅迫には、「享楽」が伴う。
    \item{\#5.3}反復脅迫を通じて、体験を予測できるようにしようとする機制(=「欲動」)が働く。
    \item{\#5.4}体験が予測できるようになって欲動が解消されると、満足がもたらされる。
  \end{itemize}
\end{note}

\subsection{存在論と体験}\label{ux5b58ux5728ux8ad6ux3068ux4f53ux9a13}

予測誤差を体験したとき、人は概念に収まりきらない「存在」を感じる。

\begin{note}{}
  \begin{itemize}
    \tightlist
    \item{\#5.5}予測誤差を体験したとき、概念に収まりきらない「存在」を人は感じる。
  \end{itemize}
\end{note}

\subsection{原抑圧と経験に昇華されない体験}\label{ux539fux6291ux5727ux3068ux7d4cux9a13ux306bux6607ux83efux3055ux308cux306aux3044ux4f53ux9a13}

この経験に昇華されていない体験は、ランダムで無秩序なものではなく、独自の「内包」を持つ。シニフィアンの体系に参入する際には、経験に昇華されていない体験も同時に発生することになる。これは「原抑圧」「性関係の排除」「一般化排除」「疎外」「エディプス第一の時」「前エディプス期」などと呼ばれる。

\begin{note}{}
  \begin{itemize}
    \tightlist
    \item{\#5.6}経験に昇華されていない体験は、ランダムで無秩序なものではなく、独自の「内包」を持つ。
    \item{\#5.7}シニフィアンの体系に参入する際に、経験に昇華されていない体験も同時に発生することになる
      \begin{itemize}
        \tightlist
        \item (=「原抑圧」)
        \item (=「性関係の排除」)
        \item (=「一般化排除」)
        \item (=「疎外」)
        \item (=「エディプス第一の時」)  
        \item (=「前エディプス期」)。
      \end{itemize}
  \end{itemize}
\end{note}

\subsection{\texorpdfstring{対象\(a\)の顕現と不安}{対象aの顕現と不安}}\label{ux5bfeux8c61aux306eux9855ux73feux3068ux4e0dux5b89}

体験が経験へと昇華されていない状態は、「世界と体験との間に『葛藤』がある状態」だと表現できる。原抑圧により生じる、独自の内包を持った反復強迫する体験が、「対象\(a\)」(「〈物〉の断片」)である。対象\(a\)が意識に現れること(「対象\(a\)の顕現」)は、自身が採用しているシニフィアンの体系では体験を説明しきれないことを証明してしまうため、その体験を統御できない「不安」と、その不安を解消するための「防衛機制」を呼び起こす。対象\(a\)の顕現は、「大他者の非一貫性(\(\cancel{\textrm{A}}\))」(「象徴界の穴」)を露呈させる。

\begin{note}{}
  \begin{itemize}
    \tightlist
    \item{\#5.8}体験が経験へと昇華されていない状態は、「世界と体験との間に『葛藤』がある状態」だと表現できる。
    \item{\#5.9}原抑圧により生じる、独自の内包を持った反復強迫する体験が、「対象$a$」(=「〈物〉の断片」)である。
    \item{\#5.10}対象$a$が意識に現れること(=「対象$a$の顕現」)は、「自身が採用しているシニフィアンの体系では体験を説明しきれない」ことを証明してしまうため、その体験を統御できない「不安」と、その不安を解消するための「防衛機制」を呼び起こす。
    \item{\#5.11}対象aの顕現は、「大他者の非一貫性(=$\cancel{\textrm{A}}$)」(=「象徴界の穴」)を露呈させる。
  \end{itemize}
\end{note}

\subsection{主体と葛藤の解消}\label{ux4e3bux4f53ux3068ux845bux85e4ux306eux89e3ux6d88}

主体による欲動に対する防衛は速やかに行われ、この「葛藤」を解消する行為を行うものを「主体」と呼ぶ。葛藤の解消と、主体の行為と、対象\(a\)の顕現に対する防衛とは、等価である。

\begin{note}{}
  \begin{itemize}
    \tightlist
    \item{\#5.12}主体による欲動に対する防衛は速やかに行われる。
    \item{\#5.13}「葛藤」を解消する行為を行うものを「主体」と呼ぶ。
    \item{\#5.14}葛藤の解消と、主体の行為と、対象$a$の顕現に対する防衛とは、等価である。
  \end{itemize}
\end{note}

\subsection{この章のまとめ}\label{ux3053ux306eux7ae0ux306eux307eux3068ux3081}

あ

\newpage
\includepdf[pages=-, scale=0.95, pagecommand={\thispagestyle{plain}}]{../figures/図5.pdf}

\subsection{人間の行為}\label{ux4ebaux9593ux306eux884cux70ba}

知と無知の弁証法以外の知性は存在しない
変分ベイズ推論以外の思考は存在しない 生は無知を孕んでいる

\newpage
\includepdf[pages=-, scale=0.90, pagecommand={\thispagestyle{plain}}]{../figures/ex.pdf}

\section{神経症と精神病における葛藤の解消}\label{ux795eux7d4cux75c7ux3068ux7cbeux795eux75c5ux306bux304aux3051ux308bux845bux85e4ux306eux89e3ux6d88}

\subsection{心的構造の基本概念}\label{ux5fc3ux7684ux69cbux9020ux306eux57faux672cux6982ux5ff5}

自我とリアリティの体系(「心的構造」)は、各個人の幼児期にその大まかな形態が確定される。心的構造により、葛藤の解消の仕方には違いがある。心的構造は「神経症」と「精神病」に大別される。

\begin{note}{}
  \begin{itemize}
    \tightlist
    \item{\#6.1}自我とリアリティの体系(=「心的構造」)は、  各個人の幼児期にその大まかな形態が確定される。
    \item{\#6.2}心的構造により、  葛藤の解消の仕方には違いがある。
    \item{\#6.3}心的構造は「神経症」と「精神病」に大別される。
  \end{itemize}
\end{note}

\subsection{神経症における父性隠喩と防衛の方法}\label{ux795eux7d4cux75c7ux306bux304aux3051ux308bux7236ux6027ux96a0ux55a9ux3068ux9632ux885bux306eux65b9ux6cd5}

神経症者は、幼児期に対象\(a\)の顕現に対して防衛する機制としての「父性隠喩」を確立している(「分離」)。父性隠喩とは、「シニフィアンの体系の全体に対して固定的な意味を保証するシニフィアン(「象徴的ファルス」)を持つアクター(「現実的父」)が存在する」という形式の「幻想」を信じることで生じる、「シニフィアンの体系の全体に対して象徴的ファルスに影響を受けた意味が付与される仕組み」である。つまり、非一貫的なものである大他者が、現実的父により統御され、その任意の部分の意味は象徴的ファルスにより保証されるということである。父性隠喩を確立するとは、「父性の欺瞞を受け入れる」ことであり、「幻想を形成する」ことである(「分離」)。神経症者は、父性隠喩を用いて対象\(a\)を隠喩化(「抑圧」)することで、対象\(a\)の顕現に対して防衛する。

\begin{note}{}
  \begin{itemize}
    \tightlist
    \item{\#6.4}神経症者は、  幼児期に対象$a$の顕現に対して防衛する機制としての「父性隠喩」を確立している(=「分離」)。
    \item{\#6.5}父性隠喩とは、「シニフィアンの体系の全体に対して固定的な意味を保証するシニフィアン(=「象徴的ファルス」)を持つアクター(=「現実的父」)が存在する」(=非一貫的なものである大他者が、現実的父により統御され、その任意の部分の意味は象徴的ファルスにより保証される)という形式の「幻想」を信じることで生じる、「シニフィアンの体系の全体に対して象徴的ファルスに影響を受けた意味が付与される仕組み」である。
    \item{\#6.6}父性隠喩を確立するとは、「父性の欺瞞を受け入れる」ことであり、「幻想を形成する」ことである(=「分離」)。
    \item{\#6.7}神経症者は、父性隠喩を用いて対象$a$を隠喩化(=「抑圧」)することで、対象aの顕現に対して防衛する。
  \end{itemize}
\end{note}

\subsection{精神病者による防衛の方法}\label{ux7cbeux795eux75c5ux8005ux306bux3088ux308bux9632ux885bux306eux65b9ux6cd5}

精神病者は、幼児期に対象\(a\)の顕現に対して防衛しなければならない状況を経験しておらず、そのため父性隠喩を確立してもいない。精神病者は、対象\(a\)を意識から排除する(「否認」「知ろうとしない」)ことで、対象\(a\)の顕現に対して防衛する。

\begin{note}{}
  \begin{itemize}
    \tightlist
    \item{\#6.8}精神病者は、幼児期に対象$a$の顕現に対して防衛しなければならない状況を経験しておらず、そのため父性隠喩を確立してもいない。
    \item{\#6.9}精神病者は、対象$a$を意識から排除する
      \begin{itemize}
        \tightlist
        \item (=「否認」)
        \item (=「知ろうとしない」)
      \end{itemize}ことで、対象$a$の顕現に対して防衛する。
  \end{itemize}
\end{note}

\subsection{神経症と精神病における症状の違い}\label{ux795eux7d4cux75c7ux3068ux7cbeux795eux75c5ux306bux304aux3051ux308bux75c7ux72b6ux306eux9055ux3044}

修正の結果構築される自我とリアリティが、他の人間個体のそれからは整合性を保てない場合、そのような修正を行った人間個体は「病的」であるとされる。神経症者の「症状」は、反復強迫する対象\(a\)となった出来事(\(\textrm{S}_1\))を、「隠喩」的もしくは「文字」的あるいは「音素」的につながりのある「言葉」を経由する(\(\textrm{S}_1\rightarrow\textrm{S}_2\))ことで、間接的に解消して満足するものである(「象徴的加工」)。一方、精神病者の「症状」は、反復強迫する対象\(a\)となった出来事(\(\textrm{S}_1\))を、直接的・無媒介的に呼び起して解決することによって、解消して満足するものである。直接的・無媒介的に呼び起された出来事は、「パラノイア」の場合では、「妄想形成(\(\textrm{S}_1\rightarrow\textrm{S}_2\))」によって解決され、「スキゾフレニー」の場合では、シニフィアンの体系を用いずにそのまま身体で享楽を受け止める(「\(\textrm{S}_1\)の散乱状態」)。

\begin{note}{}
  \begin{itemize}
    \tightlist
    \item{\#6.10}修正の結果構築される自我とリアリティが、他の人間個体のそれからは整合性を保てない場合、そのような修正を行った人間個体は「病的」であるとされる。
    \item{\#6.11}神経症者の「症状」は、反復強迫する対象$a$となった出来事($\textrm{S}_1$)を、「隠喩」的もしくは「文字」的あるいは「音素」的につながりのある「言葉」を経由する($\textrm{S}_1\rightarrow\textrm{S}_2$)ことで、間接的に解消して満足するものである(=「象徴的加工」)。
    \item{\#6.12}精神病者の「症状」は、反復強迫する対象$a$となった出来事($\textrm{S}_1$)を、直接的・無媒介的に呼び起して解決することによって、解消して満足するものである。
    \item{\#6.13}直接的・無媒介的に呼び起された出来事は、「パラノイア」の場合では、「妄想形成($\textrm{S}_1\rightarrow\textrm{S}_2$)」によって解決される。
    \item{\#6.14}直接的・無媒介的に呼び起された出来事は、「スキゾフレニー」の場合では、シニフィアンの体系を用いずにそのまま身体で享楽を受け止める(=「$\textrm{S}_1$の散乱状態」)。
  \end{itemize}
\end{note}

\subsection{この章のまとめ}\label{ux3053ux306eux7ae0ux306eux307eux3068ux3081}

あ

\newpage
\includepdf[pages=-, scale=0.90, pagecommand={\thispagestyle{plain}}]{../figures/図6.pdf}

\input{07-第二章第四節}
\newpage
\includepdf[pages=-, scale=0.90, pagecommand={\thispagestyle{plain}}]{../figures/図7.pdf}

\input{08-第二章第五節}
\newpage
\includepdf[pages=-, scale=0.90, pagecommand={\thispagestyle{plain}}]{../figures/図8.pdf}

\chapter{四つのディスクールが形成する人間社会のダイナミズム}

\input{09-第三章第一節}
\newpage
\includepdf[pages=-, scale=0.90, pagecommand={\thispagestyle{plain}}]{../figures/図9.pdf}

\section{エンジニアリングが持つダイナミズムからの疎外の結果1(抑圧と反抗)}\label{ux30a8ux30f3ux30b8ux30cbux30a2ux30eaux30f3ux30b0ux304cux6301ux3064ux30c0ux30a4ux30caux30dfux30baux30e0ux304bux3089ux306eux758eux5916ux306eux7d50ux679cuxff11ux6291ux5727ux3068ux53cdux6297}

\subsection{複雑化・硬直・プレモダン}\label{ux8907ux96d1ux5316ux786cux76f4ux30d7ux30ecux30e2ux30c0ux30f3}

\begin{note}{}
  \begin{itemize}
    \tightlist
    \item{\#10.1}人間が構築した自然を制御する仕組みや制度は、より単純なものを組み合わせることでより複雑なものとなる。
    \item{\#10.2}人間は、自身が生み出した仕組みや制度を通じて相互に協働しながら生きる(=「社会生活」)。
    \item{\#10.3}社会が複雑化すると、上位の仕組みや制度に変更を加えるのは容易ではなくなっていく。
    \item{\#10.4}これは、上位の$\textrm{S}_1$を失墜させることができないということと等価である。
    \item{\#10.5} $\textrm{S}_1$を失墜させることができない状況下では、エンジニアリングの際とは異なり、四つのディスクールの各局面はそのディスクールのまま固定されやすくなる。
  \end{itemize}
\end{note}

あ

\subsection{主人のディスクール}\label{ux4e3bux4ebaux306eux30c7ux30a3ux30b9ux30afux30fcux30eb}

\begin{note}{}
  \begin{itemize}
    \tightlist
    \item{\#10.6} $\textrm{S}_1$を失墜させることができない状況における主人のディスクール:
      \begin{itemize}
        \tightlist
        \item 確立された$\textrm{S}_1$から新たに規定される$\textrm{S}_2$が枯渇してしまっているため、新しい未既定の領域が眼前に現れない限り、(通常は)主人のディスクールが発生しなくなる。
        \item ただし、分析家のディスクールを経て、新たな視点(=$\textrm{S}_1$)に基づく世界解釈の可能性を発見した場合、その$\textrm{S}_1$に基づいた世界の再解釈が行われるようになることがある(それが端的に新奇な解釈であることもあるが、実際の社会のあり方にそぐわない妄想的な解釈であることもある)。
      \end{itemize}

$$
\uparrow\frac{\mathrm{S_1}}{\mathrm{\cancel{S}}}\genfrac{}{}{0pt}{}{\longrightarrow}{//}\frac{\mathrm{S_2}}{a}\downarrow
$$
  \end{itemize}
\end{note}

\subsection{大学のディスクール}\label{ux5927ux5b66ux306eux30c7ux30a3ux30b9ux30afux30fcux30eb}

\begin{note}{}
  \begin{itemize}
    \tightlist
    \item{\#10.7} $\textrm{S}_1$を失墜させることができない状況における大学のディスクール:
      \begin{itemize}
        \tightlist
        \item 大学のディスクールは$\textrm{S}_1$の失墜を試みないため、  この状況下において大学のディスクールは最も適合的なスタンスとなる。
        \item ただし、社会に適合的であることと不満(=$a$)が解消されることとは別である。
        \item 他のディスクールに移ることを十分に学ばないまま身を持ち崩して大学のディスクールの中で評価されない周縁(=$a$)に追いやられた場合、大学のディスクールにおける自己滅却的な主体(=$\frac{a}{\cancel{\textrm{S}}}\downarrow$)(=$\textrm{S}_1//\cancel{\textrm{S}}$)は破滅的な選択肢を取るかもしれない。
      \end{itemize}

$$
\uparrow\frac{\mathrm{S_2}}{\mathrm{S_1}}\genfrac{}{}{0pt}{}{\longrightarrow}{//}\frac{a}{\mathrm{\cancel{S}}}\downarrow
$$
  \end{itemize}
\end{note}

あ

\subsection{ヒステリー者のディスクール}\label{ux30d2ux30b9ux30c6ux30eaux30fcux8005ux306eux30c7ux30a3ux30b9ux30afux30fcux30eb}

\begin{note}{}
  \begin{itemize}
    \tightlist
    \item{\#10.8} $\textrm{S}_1$を失墜させることができない状況におけるヒステリー者のディスクール:
      \begin{itemize}
        \tightlist
        \item ヒステリー者のディスクールは、$\textrm{S}_1$により提供される$\textrm{S}_2$が主体の不満(=$\uparrow\frac{\cancel{\textrm{S}}}{a}$)を満足させられないことを明らかにするが、
        \item それにもかかわわらず$\textrm{S}_1$を失墜させることができないため、不満を抱えたままの状態に置かれる。
        \item 不満を持っている者同士が集まることもあるが、ヒステリー者のディスクールは新たなS1を打ち立てるものでもないため、不満を持つ者の集団から秩序が生まれることもない。
      \end{itemize}

$$
\uparrow\frac{\mathrm{\cancel{S}}}{a}\genfrac{}{}{0pt}{}{\longrightarrow}{//}\frac{\mathrm{S_1}}{\mathrm{S_2}}\downarrow
$$
  \end{itemize}
\end{note}

あ

\subsection{分析家のディスクール}\label{ux5206ux6790ux5bb6ux306eux30c7ux30a3ux30b9ux30afux30fcux30eb}

\begin{note}{}
  \begin{itemize}
    \tightlist
    \item{\#10.9} $\textrm{S}_1$を失墜させることができない状況における分析家のディスクール:
      \begin{itemize}
        \tightlist
        \item ・分析家のディスクールでは、うまくいかなさ(=$a$)を抱えた当人にそのうまくいかなさを解消する$\textrm{S}_1$を生み出させる(=$\frac{\cancel{\textrm{S}}}{\textrm{S}_1}\downarrow$)ことで、当人なりの新しい世界解釈を生み出す結果につながる場合がありうる(社会の$\textrm{S}_1$を失墜させることができない状況下では、社会のあり方を変えること自体は困難なままである)。
      \end{itemize}

$$
\uparrow\frac{a}{\mathrm{S_2}}\genfrac{}{}{0pt}{}{\longrightarrow}{//}\frac{\mathrm{\cancel{S}}}{\mathrm{S_1}}\downarrow
$$
  \end{itemize}
\end{note}

あ

\subsection{硬直したプレモダン的労働における疎外}\label{ux786cux76f4ux3057ux305fux30d7ux30ecux30e2ux30c0ux30f3ux7684ux52b4ux50cdux306bux304aux3051ux308bux758eux5916}

フォーディズム・設計主義
人間は機械の一部として量的に扱われる(リソース(=資材)としての労働力)
精神分析は帝国主義とフォーディズムの時代の産物かも

\newpage
\includepdf[pages=-, scale=0.90, pagecommand={\thispagestyle{plain}}]{../figures/図10.pdf}

\section{エンジニアリングが持つダイナミズムからの疎外の結果2(不安と暴力)}\label{ux30a8ux30f3ux30b8ux30cbux30a2ux30eaux30f3ux30b0ux304cux6301ux3064ux30c0ux30a4ux30caux30dfux30baux30e0ux304bux3089ux306eux758eux5916ux306eux7d50ux679cuxff12ux4e0dux5b89ux3068ux66b4ux529b}

\subsection{〈父の名〉の衰退}\label{ux7236ux306eux540dux306eux8870ux9000}

\begin{note}{}
  \begin{itemize}
    \tightlist
    \item{\#11.1}社会において上位の$\textrm{S}_1$が効力を失う場合がある
      \begin{itemize}
        \tightlist
        \item (=「神の死」)
        \item (=「〈父の名〉の衰退」)
        \item (=「象徴界の機能不全」)
      \end{itemize}
  \end{itemize}
\end{note}

あ

\subsection{貨幣による秩序}\label{ux8ca8ux5e63ux306bux3088ux308bux79e9ux5e8f}

\begin{note}{}
  \begin{itemize}
    \tightlist
    \item{\#11.2}資本主義社会において、社会を秩序付けるS1は「貨幣」である。
    \item{\#11.3}貨幣が社会を秩序付ける力を持つのは、貨幣が「商品」と交換されうるからだ。貨幣自体に力があるわけではない。
    \item{\#11.4}商品とは、社会の問題(=$a$)を解決する(=$\cancel{\textrm{S}}$)と市場において銘打たれたものである。
    \item{\#11.5}資本主義社会において、人々は所有する貨幣を増大させる方向に秩序付けられ、貨幣を量的に増大させるために人々は貨幣を「生産手段」と「生産力」の購入に再投下し(=「資本の蓄積」)、さらに高価な商品を作ろうとする。
  \end{itemize}
\end{note}

資本主義に本質的な問題と、プレモダンな専制に本質的な問題とを混同してはならない。資本主義の本質は、その運動を支える「素材」としての諸物をよりよく理解し、より効率よく利用することにある。それは労働力としての人間の扱いについても然りだ。資本主義の激化に伴い発生する労働問題として、労働者が

\subsection{不安・排除・レイシズム}\label{ux4e0dux5b89ux6392ux9664ux30ecux30a4ux30b7ux30baux30e0}

\begin{note}{}
  \begin{itemize}
    \tightlist
    \item{\#11.6}人は上位の$\textrm{S}_1$が衰退すると、自身の理解を超えた行動パターンを取る異質な他者の行動(=$a$)が、自身に危害を与えずに社会的に統御されるとは信じられなくなり、不安(=$\cancel{\textrm{A}}$)になる。
    \item{\#11.7}この不安において、異質な他者は「社会が本来的な状態になることを妨害している者」として理解されるようになり、その理解から逆転して「異質な他者を排除すれば社会の本来的な状態を回復させることができる」という幻想が生じる(=「レイシズム」)。
    \item{\#11.8}この幻想は、あたかも安寧な社会が実が可能であるかのように感じさせるものであるため、父性隠喩の確立と同じ効果を主体にもたらすがゆえに、主体に強い満足感を与えることができる。
  \end{itemize}
\end{note}

あ

\subsection{資本主義のディスクール}\label{ux8cc7ux672cux4e3bux7fa9ux306eux30c7ux30a3ux30b9ux30afux30fcux30eb}

\begin{note}{}
  \begin{itemize}
    \tightlist
    \item{\#11.9}資本主義における人々の行動は、「資本主義のディスクール(上図)」によって記述できる。
    \item{\#11.10}資本主義において、人々は「労働者」あるいは「資本家」として貨幣や資本の増大を図る一方で、「消費者」としては、自身が抱える不満(=$a$)を商品の購入(=$\cancel{\textrm{S}}$)により速やかに解消できる状況に置かれるため、神経症的な幻想や欲望を構築する際に現れたような自身の不満足の原因について思考する契機を奪われることになる。
  \end{itemize}
\end{note}

双数性

\subsection{柔軟性のあるモダンな労働における疎外}\label{ux67d4ux8edfux6027ux306eux3042ux308bux30e2ux30c0ux30f3ux306aux52b4ux50cdux306bux304aux3051ux308bux758eux5916}

モダンな労働における疎外は、欲動の外から与えられた欲望にすり替えられることで起こる
プレモダンな労働と同じく、モダンな労働でも人は生産・消費の両面で量的なリソース(資材)として扱われる

コモンが搾取や疎外を避けるわけではない。株主からかかる利潤第一主義への圧力に従属しないことがコモンズの利点だと斉藤は主張するが、それは株式会社でも可能だ。アマゾンを見ろ。株主を説得できるだけのビジョンがない知的怠惰がダメなのであって、株式会社がダメなわけではない。コモンの中でも疎外と搾取は発生しうる。改善のプロセスから逃げてはならない。

資本主義をマルクス・ガブリエルが言う「倫理資本主義」に近いものに変えると考えられる。

\subsection{大義の条件}\label{ux5927ux7fa9ux306eux6761ux4ef6}

何かに対する敵対は、原理としての大義にはならない
大義は問いであり、顕現しない(顕現した原理は専制である)

\newpage
\includepdf[pages=-, scale=0.90, pagecommand={\thispagestyle{plain}}]{../figures/図11_1.pdf}
\newpage
\includepdf[pages=-, scale=0.90, pagecommand={\thispagestyle{plain}}]{../figures/図11_2.pdf}

\chapter{人間の限界とその先}

\input{12-第四章第一節}
\newpage

\input{13-第四章第二節}
\newpage

\input{14-第四章第三節}
\newpage

\part{戯れ}

\chapter{異なる社会}

\chapter{異なる知性}

\chapter{異なる宇宙}

\backmatter

\printbibliography

\input{end}
