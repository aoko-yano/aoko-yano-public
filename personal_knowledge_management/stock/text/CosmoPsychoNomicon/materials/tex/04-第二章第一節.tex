\section{体験とシニフィアン}\label{ux4f53ux9a13ux3068ux30b7ux30cbux30d5ux30a3ux30a2ux30f3}

\subsection{脳の可塑性と体験}\label{ux8133ux306eux53efux5851ux6027ux3068ux4f53ux9a13}

人間の脳は高い可塑性を持ち、他の生物には見られない独自の挙動を示す。この脳に外界からの刺激が入力されると、それは「体験」として表現される。

\begin{note}{}
  \begin{itemize}
    \tightlist
    \item{\#4.1}人間の脳は可塑性が高く、かつ独自の挙動を示す。
    \item{\#4.2}人間の脳における外界からの刺激は「体験」と表現される。
  \end{itemize}
\end{note}

\subsection{シニフィアンによる外界に対する予測}\label{ux30b7ux30cbux30d5ux30a3ux30a2ux30f3ux306bux3088ux308bux5916ux754cux306bux5bfeux3059ux308bux4e88ux6e2c}

人間の脳は、この体験を基に外界に対する予測を行うが、その機能の一部は「シニフィアン」によって実現されている。シニフィアンとは、体験の特定の部分から特徴量を抽出し、その特徴量に基づいて「類似する部分」と「類似しない部分」を区別できるように括り出したものである。これは「シニフィアンは主体を他のシニフィアンに代表象する」という表現で説明される現象である。

\begin{note}{}
  \begin{itemize}
    \tightlist
    \item{\#4.3}人間において、外界に対して予測を行う機能の一部は、「シニフィアン」により実現されている。
    \item{\#4.4}体験のうちのある部分から特徴量を抽出した上で、「その部分に類似する部分」を括り出し、「その部分に類似しない部分」から区別できるようにした際に、その括り出したものをシニフィアンと呼ぶ(=「シニフィアンは主体を他のシニフィアンに代表象する」)。
  \end{itemize}
\end{note}

\subsection{言語と経験の関係}\label{ux8a00ux8a9eux3068ux7d4cux9a13ux306eux95a2ux4fc2}

人間が用いる言語は、このシニフィアンを活用したシステムである。体験のうち、シニフィアンを用いてモデル化され説明可能となった範囲を特に「経験」と呼ぶ。これは体験の「言語化」あるいは経験への「昇華」とも表現される。このシニフィアンの体系全体は「大他者(=A)」、または「原‐象徴界」と呼ばれる。

\begin{note}{}
  \begin{itemize}
    \tightlist
    \item{\#4.5}言語はシニフィアンを用いたものである。
    \item{\#4.6}体験のうち、シニフィアンを用いてモデル化され説明されるようになった範囲を、特に「経験」と呼ぶ(=体験の「言語化」)(=経験への「昇華」)。
    \item{\#4.7}シニフィアンの体系を、「大他者(=A)」(=「原‐象徴界」)と表現する。
  \end{itemize}
\end{note}

\subsection{世界の構成要素}\label{ux4e16ux754cux306eux69cbux6210ux8981ux7d20}

経験によってモデル化された範囲が、その人間にとっての「世界」を形成する。この世界は「リアリティ」と「自我」という二つの重要な要素を含んでいる。これらは共に、モデル内の仮象であり、体験を秩序立てて予測するために構築されたものである。

\begin{note}{}
  \begin{itemize}
    \tightlist
    \item{\#4.8}経験でモデル化された範囲が、その人間にとっての「世界」である。
    \item{\#4.9}世界は「リアリティ」と「自我」を含む。
    \item{\#4.10}「自我」と「リアリティ」は共に
      \begin{itemize}
        \tightlist
        \item モデル内の仮象であり
        \item 体験を秩序立てて予測するために構築されたもの
      \end{itemize}である。
  \end{itemize}
\end{note}

\subsection{自我とリアリティの機能}\label{ux81eaux6211ux3068ux30eaux30a2ux30eaux30c6ux30a3ux306eux6a5fux80fd}

「自我」は体験の流れを引き受ける焦点としての役割を果たす。一方、「リアリティ」は二つの重要な役割を担う。第一に体験の流れを発生させる場としての役割、第二に自我を内部に含む場としての役割である。

\begin{note}{}
  \begin{itemize}
    \tightlist
    \item{\#4.11}「自我」は、体験の流れを引き受ける焦点の役割を果たす。
    \item{\#4.12}「リアリティ」は、
      \begin{itemize}
        \tightlist
        \item 体験の流れを発生させる場
        \item 自我を内部に含む場
      \end{itemize}の役割を果たす。
  \end{itemize}
\end{note}

\subsection{この章のまとめ}\label{ux3053ux306eux7ae0ux306eux307eux3068ux3081}

あ
