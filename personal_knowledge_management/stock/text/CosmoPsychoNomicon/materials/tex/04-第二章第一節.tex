\section{体験とシニフィアン}\label{ux4f53ux9a13ux3068ux30b7ux30cbux30d5ux30a3ux30a2ux30f3}

\begin{note}{この小節で扱う命題}
  \begin{itemize}
    \tightlist
    \item{\#4.1}人間の脳は可塑性が高く、かつ独自の挙動を示す。
    \item{\#4.2}人間の脳における外界からの刺激は「体験」と表現される。
    \item{\#4.3}人間において、外界に対して予測を行う機能の一部は、「シニフィアン」により実現されている。
    \item{\#4.4}体験のうちのある部分から特徴量を抽出した上で、「その部分に類似する部分」を括り出し、「その部分に類似しない部分」から区別できるようにした際に、その括り出したものをシニフィアンと呼ぶ(=「シニフィアンは主体を他のシニフィアンに代表象する」)。
    \item{\#4.5}言語はシニフィアンを用いたものである。
    \item{\#4.6}体験のうち、シニフィアンを用いてモデル化され説明されるようになった範囲を、特に「経験」と呼ぶ(=体験の「言語化」)(=経験への「昇華」)。
    \item{\#4.7}シニフィアンの体系を、「大他者(=A)」(=「原‐象徴界」)と表現する。
    \item{\#4.8}経験でモデル化された範囲が、その人間にとっての「世界」である。
    \item{\#4.9}世界は「リアリティ」と「自我」を含む。
    \item{\#4.10}「自我」と「リアリティ」は共に
      \begin{itemize}
        \tightlist
        \item モデル内の仮象であり
        \item 体験を秩序立てて予測するために構築されたもの
      \end{itemize}である。
    \item{\#4.11}「自我」は、体験の流れを引き受ける焦点の役割を果たす。
    \item{\#4.12}「リアリティ」は、
      \begin{itemize}
        \tightlist
        \item 体験の流れを発生させる場
        \item 自我を内部に含む場
      \end{itemize}の役割を果たす。
  \end{itemize}
\end{note}

あ

\subsection{この章のまとめ}\label{ux3053ux306eux7ae0ux306eux307eux3068ux3081}

あ
