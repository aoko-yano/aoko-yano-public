\section{神経症と精神病における葛藤の解消}\label{ux795eux7d4cux75c7ux3068ux7cbeux795eux75c5ux306bux304aux3051ux308bux845bux85e4ux306eux89e3ux6d88}

\subsection{心的構造の基本概念}\label{ux5fc3ux7684ux69cbux9020ux306eux57faux672cux6982ux5ff5}

自我とリアリティの体系(「心的構造」)は、各個人の幼児期にその大まかな形態が確定される。心的構造により、葛藤の解消の仕方には違いがある。心的構造は「神経症」と「精神病」に大別される。

\begin{note}{}
  \begin{itemize}
    \tightlist
    \item{\#6.1}自我とリアリティの体系(=「心的構造」)は、  各個人の幼児期にその大まかな形態が確定される。
    \item{\#6.2}心的構造により、  葛藤の解消の仕方には違いがある。
    \item{\#6.3}心的構造は「神経症」と「精神病」に大別される。
  \end{itemize}
\end{note}

\subsection{神経症における父性隠喩と防衛の方法}\label{ux795eux7d4cux75c7ux306bux304aux3051ux308bux7236ux6027ux96a0ux55a9ux3068ux9632ux885bux306eux65b9ux6cd5}

神経症者は、幼児期に対象\(a\)の顕現に対して防衛する機制としての「父性隠喩」を確立している(「分離」)。父性隠喩とは、「シニフィアンの体系の全体に対して固定的な意味を保証するシニフィアン(「象徴的ファルス」)を持つアクター(「現実的父」)が存在する」という形式の「幻想」を信じることで生じる、「シニフィアンの体系の全体に対して象徴的ファルスに影響を受けた意味が付与される仕組み」である。つまり、非一貫的なものである大他者が、現実的父により統御され、その任意の部分の意味は象徴的ファルスにより保証されるということである。父性隠喩を確立するとは、「父性の欺瞞を受け入れる」ことであり、「幻想を形成する」ことである(「分離」)。神経症者は、父性隠喩を用いて対象\(a\)を隠喩化(「抑圧」)することで、対象\(a\)の顕現に対して防衛する。

\begin{note}{}
  \begin{itemize}
    \tightlist
    \item{\#6.4}神経症者は、  幼児期に対象$a$の顕現に対して防衛する機制としての「父性隠喩」を確立している(=「分離」)。
    \item{\#6.5}父性隠喩とは、「シニフィアンの体系の全体に対して固定的な意味を保証するシニフィアン(=「象徴的ファルス」)を持つアクター(=「現実的父」)が存在する」(=非一貫的なものである大他者が、現実的父により統御され、その任意の部分の意味は象徴的ファルスにより保証される)という形式の「幻想」を信じることで生じる、「シニフィアンの体系の全体に対して象徴的ファルスに影響を受けた意味が付与される仕組み」である。
    \item{\#6.6}父性隠喩を確立するとは、「父性の欺瞞を受け入れる」ことであり、「幻想を形成する」ことである(=「分離」)。
    \item{\#6.7}神経症者は、父性隠喩を用いて対象$a$を隠喩化(=「抑圧」)することで、対象aの顕現に対して防衛する。
  \end{itemize}
\end{note}

\subsection{精神病者による防衛の方法}\label{ux7cbeux795eux75c5ux8005ux306bux3088ux308bux9632ux885bux306eux65b9ux6cd5}

精神病者は、幼児期に対象\(a\)の顕現に対して防衛しなければならない状況を経験しておらず、そのため父性隠喩を確立してもいない。精神病者は、対象\(a\)を意識から排除する(「否認」「知ろうとしない」)ことで、対象\(a\)の顕現に対して防衛する。

\begin{note}{}
  \begin{itemize}
    \tightlist
    \item{\#6.8}精神病者は、幼児期に対象$a$の顕現に対して防衛しなければならない状況を経験しておらず、そのため父性隠喩を確立してもいない。
    \item{\#6.9}精神病者は、対象$a$を意識から排除する
      \begin{itemize}
        \tightlist
        \item (=「否認」)
        \item (=「知ろうとしない」)
      \end{itemize}ことで、対象$a$の顕現に対して防衛する。
  \end{itemize}
\end{note}

\subsection{神経症と精神病における症状の違い}\label{ux795eux7d4cux75c7ux3068ux7cbeux795eux75c5ux306bux304aux3051ux308bux75c7ux72b6ux306eux9055ux3044}

修正の結果構築される自我とリアリティが、他の人間個体のそれからは整合性を保てない場合、そのような修正を行った人間個体は「病的」であるとされる。神経症者の「症状」は、反復強迫する対象\(a\)となった出来事(\(\textrm{S}_1\))を、「隠喩」的もしくは「文字」的あるいは「音素」的につながりのある「言葉」を経由する(\(\textrm{S}_1\rightarrow\textrm{S}_2\))ことで、間接的に解消して満足するものである(「象徴的加工」)。一方、精神病者の「症状」は、反復強迫する対象\(a\)となった出来事(\(\textrm{S}_1\))を、直接的・無媒介的に呼び起して解決することによって、解消して満足するものである。直接的・無媒介的に呼び起された出来事は、「パラノイア」の場合では、「妄想形成(\(\textrm{S}_1\rightarrow\textrm{S}_2\))」によって解決され、「スキゾフレニー」の場合では、シニフィアンの体系を用いずにそのまま身体で享楽を受け止める(「\(\textrm{S}_1\)の散乱状態」)。

\begin{note}{}
  \begin{itemize}
    \tightlist
    \item{\#6.10}修正の結果構築される自我とリアリティが、他の人間個体のそれからは整合性を保てない場合、そのような修正を行った人間個体は「病的」であるとされる。
    \item{\#6.11}神経症者の「症状」は、反復強迫する対象$a$となった出来事($\textrm{S}_1$)を、「隠喩」的もしくは「文字」的あるいは「音素」的につながりのある「言葉」を経由する($\textrm{S}_1\rightarrow\textrm{S}_2$)ことで、間接的に解消して満足するものである(=「象徴的加工」)。
    \item{\#6.12}精神病者の「症状」は、反復強迫する対象$a$となった出来事($\textrm{S}_1$)を、直接的・無媒介的に呼び起して解決することによって、解消して満足するものである。
    \item{\#6.13}直接的・無媒介的に呼び起された出来事は、「パラノイア」の場合では、「妄想形成($\textrm{S}_1\rightarrow\textrm{S}_2$)」によって解決される。
    \item{\#6.14}直接的・無媒介的に呼び起された出来事は、「スキゾフレニー」の場合では、シニフィアンの体系を用いずにそのまま身体で享楽を受け止める(=「$\textrm{S}_1$の散乱状態」)。
  \end{itemize}
\end{note}

\subsection{この章のまとめ}\label{ux3053ux306eux7ae0ux306eux307eux3068ux3081}

あ
