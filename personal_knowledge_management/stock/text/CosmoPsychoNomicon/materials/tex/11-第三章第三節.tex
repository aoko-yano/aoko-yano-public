\section{エンジニアリングが持つダイナミズムからの疎外の結果2(不安と暴力)}\label{ux30a8ux30f3ux30b8ux30cbux30a2ux30eaux30f3ux30b0ux304cux6301ux3064ux30c0ux30a4ux30caux30dfux30baux30e0ux304bux3089ux306eux758eux5916ux306eux7d50ux679cuxff12ux4e0dux5b89ux3068ux66b4ux529b}

\subsection{〈父の名〉の衰退}\label{ux7236ux306eux540dux306eux8870ux9000}

\begin{note}{この小節で扱う命題}
  \begin{itemize}
    \tightlist
    \item{\#11.1}社会において上位の$\textrm{S}_1$が効力を失う場合がある
      \begin{itemize}
        \tightlist
        \item (=「神の死」)
        \item (=「〈父の名〉の衰退」)
        \item (=「象徴界の機能不全」)
      \end{itemize}
  \end{itemize}
\end{note}

あ

\subsection{貨幣による秩序}\label{ux8ca8ux5e63ux306bux3088ux308bux79e9ux5e8f}

\begin{note}{この小節で扱う命題}
  \begin{itemize}
    \tightlist
    \item{\#11.2}資本主義社会において、社会を秩序付けるS1は「貨幣」である。
    \item{\#11.3}貨幣が社会を秩序付ける力を持つのは、貨幣が「商品」と交換されうるからだ。貨幣自体に力があるわけではない。
    \item{\#11.4}商品とは、社会の問題(=$a$)を解決する(=$\cancel{\textrm{S}}$)と市場において銘打たれたものである。
    \item{\#11.5}資本主義社会において、人々は所有する貨幣を増大させる方向に秩序付けられ、貨幣を量的に増大させるために人々は貨幣を「生産手段」と「生産力」の購入に再投下し(=「資本の蓄積」)、さらに高価な商品を作ろうとする。
  \end{itemize}
\end{note}

資本主義に本質的な問題と、プレモダンな専制に本質的な問題とを混同してはならない。資本主義の本質は、その運動を支える「素材」としての諸物をよりよく理解し、より効率よく利用することにある。それは労働力としての人間の扱いについても然りだ。資本主義の激化に伴い発生する労働問題として、労働者が

\subsection{不安・排除・レイシズム}\label{ux4e0dux5b89ux6392ux9664ux30ecux30a4ux30b7ux30baux30e0}

\begin{note}{この小節で扱う命題}
  \begin{itemize}
    \tightlist
    \item{\#11.6}人は上位の$\textrm{S}_1$が衰退すると、自身の理解を超えた行動パターンを取る異質な他者の行動(=$a$)が、自身に危害を与えずに社会的に統御されるとは信じられなくなり、不安(=Ⱥ)になる。
    \item{\#11.7}この不安において、異質な他者は「社会が本来的な状態になることを妨害している者」として理解されるようになり、その理解から逆転して「異質な他者を排除すれば社会の本来的な状態を回復させることができる」という幻想が生じる(=「レイシズム」)。
    \item{\#11.8}この幻想は、あたかも安寧な社会が実が可能であるかのように感じさせるものであるため、父性隠喩の確立と同じ効果を主体にもたらすがゆえに、主体に強い満足感を与えることができる。
  \end{itemize}
\end{note}

あ

\subsection{資本主義のディスクール}\label{ux8cc7ux672cux4e3bux7fa9ux306eux30c7ux30a3ux30b9ux30afux30fcux30eb}

\begin{note}{この小節で扱う命題}
  \begin{itemize}
    \tightlist
    \item{\#11.9}資本主義における人々の行動は、「資本主義のディスクール(上図)」によって記述できる。
    \item{\#11.10}資本主義において、人々は「労働者」あるいは「資本家」として貨幣や資本の増大を図る一方で、「消費者」としては、自身が抱える不満(=$a$)を商品の購入(=$\cancel{\textrm{S}}$)により速やかに解消できる状況に置かれるため、神経症的な幻想や欲望を構築する際に現れたような自身の不満足の原因について思考する契機を奪われることになる。
  \end{itemize}
\end{note}

双数性

\subsection{柔軟性のあるモダンな労働における疎外}\label{ux67d4ux8edfux6027ux306eux3042ux308bux30e2ux30c0ux30f3ux306aux52b4ux50cdux306bux304aux3051ux308bux758eux5916}

モダンな労働における疎外は、欲動の外から与えられた欲望にすり替えられることで起こる
プレモダンな労働と同じく、モダンな労働でも人は生産・消費の両面で量的なリソース(資材)として扱われる

コモンが搾取や疎外を避けるわけではない。株主からかかる利潤第一主義への圧力に従属しないことがコモンズの利点だと斉藤は主張するが、それは株式会社でも可能だ。アマゾンを見ろ。株主を説得できるだけのビジョンがない知的怠惰がダメなのであって、株式会社がダメなわけではない。コモンの中でも疎外と搾取は発生しうる。改善のプロセスから逃げてはならない。

資本主義をマルクス・ガブリエルが言う「倫理資本主義」に近いものに変えると考えられる。

\subsection{大義の条件}\label{ux5927ux7fa9ux306eux6761ux4ef6}

何かに対する敵対は、原理としての大義にはならない
大義は問いであり、顕現しない(顕現した原理は専制である)
