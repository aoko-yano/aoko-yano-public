% Options for packages loaded elsewhere
\PassOptionsToPackage{unicode}{hyperref}
\PassOptionsToPackage{hyphens}{url}
%
\documentclass[8pt, a5paper]{ltjsarticle}
\pagestyle{empty}
\title{目次}
\author{}
\date{}
\usepackage{amsmath,amssymb}
\usepackage{iftex}
\ifPDFTeX
  \usepackage[T1]{fontenc}
  \usepackage[utf8]{inputenc}
  \usepackage{textcomp} % provide euro and other symbols
\else % if luatex or xetex
  \usepackage{unicode-math} % this also loads fontspec
  \defaultfontfeatures{Scale=MatchLowercase}
  \defaultfontfeatures[\rmfamily]{Ligatures=TeX,Scale=1}
\fi
\usepackage{lmodern}
\ifPDFTeX\else
  % xetex/luatex font selection
\fi
% Use upquote if available, for straight quotes in verbatim environments
\IfFileExists{upquote.sty}{\usepackage{upquote}}{}
\IfFileExists{microtype.sty}{% use microtype if available
  \usepackage[]{microtype}
  \UseMicrotypeSet[protrusion]{basicmath} % disable protrusion for tt fonts
}{}
\makeatletter
\@ifundefined{KOMAClassName}{% if non-KOMA class
  \IfFileExists{parskip.sty}{%
    \usepackage{parskip}
  }{% else
    \setlength{\parindent}{0pt}
    \setlength{\parskip}{6pt plus 2pt minus 1pt}}
}{% if KOMA class
  \KOMAoptions{parskip=half}}
\makeatother
\usepackage{xcolor}
\setlength{\emergencystretch}{3em} % prevent overfull lines
\providecommand{\tightlist}{%
  \setlength{\itemsep}{0pt}\setlength{\parskip}{0pt}}
\setcounter{secnumdepth}{-\maxdimen} % remove section numbering
\usepackage{bookmark}
\IfFileExists{xurl.sty}{\usepackage{xurl}}{} % add URL line breaks if available
\urlstyle{same}
\hypersetup{
  hidelinks,
  pdfcreator={LaTeX via pandoc}}
\begin{document}
\maketitle

\section{README}\label{readme}

\begin{itemize}
\tightlist
\item
  \href{aoko-yano-public/personal_knowledge_management/stock/text/虚無・無知・飢餓―不安の時代における諸学の綜合/README}{README}
\end{itemize}

\section{図表}\label{ux56f3ux8868}

\begin{itemize}
\tightlist
\item
  \url{-1-緒言}
\item
  \url{00-序論:本稿の意義}
\item
  \url{01-第一章:生命と環境}
\item
  \url{02-第二章:同一性と変化}
\item
  \url{03-第三章:脳と自由エネルギー原理}
\item
  \url{04-第四章:体験とシニフィアン}
\item
  \url{05-第五章:対象aと欲動の主体}
\item
  \url{ex-結節点1:人間の行為}
\item
  \url{06-第六章:神経症と精神病における葛藤の解消}
\item
  \url{07-第七章:エディプス・コンプレックスの諸段階}
\item
  \url{08-第八章:神経症的主体における四つのディスクール}
\item
  \url{09-第九章:自然への抵抗としてのエンジニアリングと芸術}
\item
  \url{10-第十章:エンジニアリングが持つダイナミズムからの疎外の結果1(抑圧と反抗)}
\item
  \url{11-第十一章:エンジニアリングが持つダイナミズムからの疎外の結果2(不安と暴力)}
\item
  \url{12-結論1:人間の不満と満足が現れるダイナミズムのモデル化}
\item
  \url{13-結論2:新しい物語の推奨ボーダーライン}
\item
  \url{14-結論3:満足を可能にする生態系}
\end{itemize}

\end{document}
