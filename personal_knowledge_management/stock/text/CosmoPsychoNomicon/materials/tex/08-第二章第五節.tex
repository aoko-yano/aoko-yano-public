\section{神経症的主体における四つのディスクールと自然についての理解}\label{ux7b2cux516bux7ae0ux795eux7d4cux75c7ux7684ux4e3bux4f53ux306bux304aux3051ux308bux56dbux3064ux306eux30c7ux30a3ux30b9ux30afux30fcux30ebux3068ux81eaux7136ux306bux3064ux3044ux3066ux306eux7406ux89e3}

\subsection{ディスクールの要素}\label{ux30c7ux30a3ux30b9ux30afux30fcux30ebux306eux8981ux7d20}

神経症的な欲望の主体が対象\(a\)を「(〈父〉により解決される)問題」とするとき、主体はその問題に対して「(〈父〉という)根拠」と「(問題が解決された結果である)結論」が存在すると信じている(\#8.1)。このように解釈をするとき、神経症者の思考や行動を表現する次の四つの要素を用いて、この章で説明する「四つのディスクール」を描くことができる(\#8.2)。

\begin{itemize}
\tightlist
\item
  \(a\):対象\(a\)あるいは残余\(a\)。転じて、予測誤差としての不確実性、加えて、そこにファルスが与えられるべきとされるもの。
\item
  \(\cancel{\textrm{S}}\):欲望の主体。残余\(a\)の解消を試みてシニフィアンを操作する。
\item
  \(\textrm{S}_1\):象徴的父。転じて、シニフィアンを連鎖させて構築する「言説」の根拠であり、根拠の選択の仕方により規定される「問いの枠組み(プロブレマティク)」。
\item
  \(\textrm{S}_2\):象徴的父がもたらす法。転じて、言説の結論であり、問いの枠組みにおいて根拠に従属する諸命題。
\end{itemize}

この四つの要素は、
\(a \rightarrow \cancel{\textrm{S}} \rightarrow \textrm{S}_1 \rightarrow \textrm{S}_2 \rightarrow a \rightarrow ...\)
という順に並べられる。つまり、「予測誤差が思考としての主体を作動させ、新しいシニフィアンの秩序を形成するが、そうして形成された秩序からも予測誤差が必ず発生する」という思考の弁証法がここにもみられるのだ。

\subsection{ディスクールの構造}\label{ux30c7ux30a3ux30b9ux30afux30fcux30ebux306eux69cbux9020}

四つのディスクールを構成する各位置には、右のような役割がある(\#8.3)。この四つの位置は、思考の弁証法において「動因が他者を新たな位置に据える際に、そのような仕方で動因が動かなければならない原因が真理にあり、動因が他者を新たな位置に据えた結果として生産物が発生してしまうという」ということを意味している。

\begin{itemize}
\tightlist
\item
  主体が当初同一化しているものが真理である。
\item
  真理には十全でないところがあり、それが動因を発生させる。
\item
  動因は他者に働きかけ、他者は生産物を算出する。
\item
  生産物は真理を十全にすべく生じたものだが、それは実現しない。
\end{itemize}

\[
\uparrow\frac{\mathrm{動因}}{\mathrm{真理}}\genfrac{}{}{0pt}{}{\longrightarrow}{//}\frac{\mathrm{他者}}{生産物}\downarrow
\]

ここからはエディプス・コンプレックスの流れに即して四つのディスクールをみていくが、この四つのディスクールは、主体が世界を理解するプロセスに対応しているとも言える。そこで、本章では具体例として人間が自然を理解する場合ではどのように四つのディスクールが働いているのかも見ていくことにする。

\subsection{分析家のディスクール}\label{ux5206ux6790ux5bb6ux306eux30c7ux30a3ux30b9ux30afux30fcux30eb}

分離が始まる瞬間(=エディプス第二の時)に対応するのが、右の「分析家のディスクール」である(\#8.4)。そこでは、まず主体は既存のシニフィアンの体系(=\(\textrm{S}_2\))に同一化している。だが、シニフィアンの体系には非一貫性があり、予測誤差としての残余\(a\)が生じる。残余\(a\)は主体(=\(\cancel{\textrm{S}}\))を作動させ、主体は革新的な視点(=\(\textrm{S}_1\))を得る。なお、新たな視点は既存のシニフィアンの体系と調和せず(=\(\textrm{S}_2//\textrm{S}_2\))、シニフィアンの体系を組みかえはじめる。このディスクールは不安定であり、速やかに下記の「主人のディスクール」へと移行する。

\[
\uparrow\frac{a}{\mathrm{S_2}}\genfrac{}{}{0pt}{}{\longrightarrow}{//}\frac{\mathrm{\cancel{S}}}{\mathrm{S_1}}\downarrow
\]

このようなディスクールの具体例としては、新しい世界観の提唱を挙げることができるだろう。天体運動論は、その中でも特にスッキリとした数学的解決が得られた例として参照することができる。そこでは、まず所与の感覚としての夜闇と光点の動き
\(\textrm{S}_2\) がある。その乱雑な所与の感覚は不安 \(a\)
を催し、それを理解しようとする主体 \(\cancel{\textrm{S}}\)
を作動させる。そうして、地球の周りを天体が回るという「天動説」のアイデアが\(\textrm{S}_1\)として設定される。この設定によって、天空を巡る光点の群れという混沌を秩序付けられるようになる。

\subsection{主人のディスクール}\label{ux4e3bux4ebaux306eux30c7ux30a3ux30b9ux30afux30fcux30eb}

父性隠喩を確立する段階(=「エディプス第三の時」)に対応するのが、右の主人のディスクールである(\#8.5)。主人のディスクールは、主体(=\(\cancel{\textrm{S}}\))は新たな根拠となるシニフィアン(=\(\textrm{S}_1\))を生み出すところから始まる(分析家のディスクールで\(\textrm{S}_1\)が生み出される段階に等しい)。すると、そこから新たな根拠に基づいて様々な命題が生み出されていく(=\(\textrm{S}_1\rightarrow\textrm{S}_2\))。しかし、そうして構築された新たなシニフィアンの体系にも非一貫性(=\(a\))がある。この非一貫性は、このディスクールで最初に欲望の主体が解消しようとしたものとは異なる新たな残余\(a\)だ。こうして生み出された残余\(a\)と主体との間には断絶があるが(=\(\cancel{\textrm{S}}\)//a)、主体はこの断絶が克服されうるものなのだという幻想を信じている(=\(\cancel{\textrm{S}}\)◇a)。

\[
\uparrow\frac{\mathrm{S_1}}{\mathrm{\cancel{S}}}\genfrac{}{}{0pt}{}{\longrightarrow}{//}\frac{\mathrm{S_2}}{a}\downarrow
\]

天体運動論の例でいえば、天動説というアイデアに基づいて所与の感覚を天体の動きとして秩序付ける動きが始まることで、主体が抱える不安は解消に向かうことになる(=\(\uparrow\frac{\textrm{S}_1}{\cancel{\textrm{S}}}\))。そうして、そのアイデアに沿って、個々の天体がどれくらいの周期で天球上を巡るかという具体的な説明がされるようになる(=\(\textrm{S}_1\rightarrow\textrm{S}_2\))。だが、天体の軌道は地球を中心とした単純な円によってはうまく説明しきれない。惑星は天球上を逆行することもあるからだ。こうして、理論との誤差(=\(a\))が生まれる(=\(\frac{\textrm{S}_2}{a}\downarrow\))。この誤差は、素朴な天動説によっては解消することができない(\(\cancel{\textrm{S}}//a\))。

惑星の動きほどスッキリとした解決が得られる場合は少ないが、自然の動きについても本章でみるのと同じプロセスを経て理解が進むことになる。まずは混沌とした所与の中に、秩序の源泉となる点が打ち立てられる(=分析家のディスクール)。そして、そこを根拠として物事が体系的に説明されるようになり、それに合わせて人は制度や仕組みを作るようになるのだ(=主人のディスクール)。

\subsection{大学のディスクール}\label{ux5927ux5b66ux306eux30c7ux30a3ux30b9ux30afux30fcux30eb}

確立した父性隠喩について、現実的父に同一化し象徴的ファルスを持っていると思いたい者は「大学のディスクール」を好むようになる(\#8.6)。そこでは、主体(=\(\cancel{\textrm{S}}\))は言説の根拠(=\(\textrm{S}_1\))を所持する者に同一化している。言説の根拠はそれ単独ではシニフィアンの体系を形成できず、自身に基づいた様々な命題を持っている(=\(\uparrow\frac{\textrm{S}_2}{\textrm{S}_1}\))。様々な命題は、新たな残余\(a\)を既存の問いの枠組みを保持したまま解決しようとする(=\(\textrm{S}_2\rightarrow a\))。だが、その試みは不徹底に終わり、新たな欲望の主体(=\(\cancel{\textrm{S}}\))を発生させる。しかし、新たな欲望の主体に従って再びシニフィアンの体系を組みかえることは、現在の主体の同一化を放棄させることを意味するので、この新たな欲望の主体は抑圧される。

\[
\uparrow\frac{\mathrm{S_2}}{\mathrm{S_1}}\genfrac{}{}{0pt}{}{\longrightarrow}{//}\frac{a}{\mathrm{\cancel{S}}}\downarrow
\]

大学のディスクールは、主人のディスクールとは違い、新たな枠組みの創始を望まない官僚主義的なディスクールだ{[}\^{}1{]}。。例えば、天動説という世界観(=\(\textrm{S}_1\))に基づいた説明(=\(\uparrow\frac{\textrm{S}_2}{\textrm{S}_1}\))では、説明できないままに留まっていた惑星の軌道(=\(a\))に対していくつかの概念が追加された。そうすることによって、理論は精緻化され、予測誤差を縮小することができるからだ{[}\^{}2{]}。実際、これらの概念を追加することによって惑星の逆行もかなり精度よく計算できた。だが、予測誤差がゼロになることはなく、したがって認識を発展させる弁証法的作用としての主体(=\(\cancel{\textrm{S}}\))も燻り続ける(=\(\frac{a}{\cancel{\textrm{S}}}\downarrow\))。とはいえ、追加された概念によって天体の軌道予測は精緻にし続けていくことができるのだから、大学のディスクールに立つ主体は。あえて天動説を放棄して混沌を再び解き放とうとは思わない(=\(\textrm{S}_1//\cancel{\textrm{S}}\))。

\begin{itemize}
\tightlist
\item
  {[}\^{}1{]} 向井(2016: 381)を参照。
\item
  {[}\^{}2{]}
  追加される概念は「離心円(地球の中心とは違う点に中心を持つ円)」と「従円(周転円に比べて大きな円。この従円が離心円になっている)・周転円(従円の円周上に中心を持つ点)・エカント(地球の中心とも従円の中心とも違う場所に、エカント点を打ち、周転円の中心をエカントから見て一定の角速度で動くようにする)」である。詳しい説明はホイル(1973=1974)を参照。今日の私たちは、惑星は太陽を焦点とする楕円軌道を回るということ、その際に惑星は面積速度が一定になるように運動することなどを知っている。そのことを知っているからこそ、そこから計算して天体の軌道を計算することなど簡単だと思い込んでいる。だが、地上から見える光点の不可解な動きのみを出発点としてこの科学的知見に至ることは大変難しい。
\end{itemize}

\subsection{ヒステリー者のディスクール}\label{ux30d2ux30b9ux30c6ux30eaux30fcux8005ux306eux30c7ux30a3ux30b9ux30afux30fcux30eb}

確立した父性隠喩について、象徴的ファルスに同一化し現実的父に欲望されることを欲望する者は、「ヒステリー者のディスクール」を好むようになる(\#8.7)。そこでは主体は、対象\(a\)の位置に来るべき象徴的ファルスに同一化するために、ファルスに仮装する(=\(\uparrow\frac{\cancel{\textrm{S}}}{a}\))。仮装した主体は自身では対象\(a\)を解消できない。そのため、仮装した主体は対象\(a\)を解消すべく、現実的父になりえそうな他者に働きかけて(=\(\cancel{\textrm{S}}\rightarrow\textrm{S}_1\))様々な命題を吐き出させる(=\(\frac{\textrm{S}_1}{\textrm{S}_2}\downarrow\))。しかし、いかなる命題も対象\(a\)そのものを根絶することはない(=\(a//\textrm{S}_2\))。そのため、それらの命題の根拠(=\(\textrm{S}_1\))も失墜する。\(\textrm{S}_1\)は超越的な対象についてのシニフィアンであるが、それ自体が超越的であるわけではないため、失墜させることができるという点がポイントだ。

\[
\uparrow\frac{\mathrm{\cancel{S}}}{a}\genfrac{}{}{0pt}{}{\longrightarrow}{//}\frac{\mathrm{S_1}}{\mathrm{S_2}}\downarrow
\]

このようなヒステリー者のディスクールは、世界を理解するプロセスにおいては「予測誤差を常に前面に掲げ続け、既存の世界観の限界を明らかにし、その正当性や信頼を失墜させる」という革新的な動きをもたらす。そこでは、まず予測誤差あるいは不確実性(=\(a\))の解決(=\(\uparrow\frac{\cancel{\textrm{S}}}{a}\))を、既に確立された視点/問題の枠組み/権威(=\(\textrm{S}_1\))によって達成しようとする。だが、\(\textrm{S}_1\)は有限の知(=\(\textrm{S}_2\))しか生みだせず(=\(\frac{\textrm{S}_1}{\textrm{S}_2}\downarrow\))、それが予測誤差や不確実性を解決することはない(=\(a//\textrm{S}_2\))。こうして、ヒステリー者のディスクールは既存の世界観を頼ることで、かえって逆にその無能力を露呈させてしまう。その結果は\(\textrm{S}_1\)に対する失望に終わり、\(\textrm{S}_1\)は手段としての信頼を失墜させる。この\(\textrm{S}_1\)の失墜の結果、ヒステリー者は混乱の中に投げ出されてしまう。

周転円をいくら追加しても予測誤差がゼロにできない中でも、背後ではブラーエ(1546-1601)らによってそれまでよりも精密な計測データ
\(\textrm{S}_2\)
が溜まっていっていた。これらの計測データもまた、既存の世界観の中で用いられていた既存の計測機器を使用して貯められてきたきたものだ(=\(\frac{\textrm{S}_1}{\textrm{S}_2}\downarrow\))。こうして貯められたデータが、新しい天体モデルの誕生を準備することになった。

\subsection{分析家のディスクール、再び}\label{ux5206ux6790ux5bb6ux306eux30c7ux30a3ux30b9ux30afux30fcux30ebux518dux3073}

このような蓄積された知と混乱の中で、分析家のディスクールを通じて、人は新しい問題の枠組みを生み出す。天動説の例でいえば、天動説を唱えていたプトレマイオス(83頃-163頃)の著作についての批判的研究からコペルニクス(1473-1543)が太陽中心説(=地動説)を唱えた。しかし、この時点の地動説ではエカントが除去されたのみで、周転円の数が減ったわけでもなく、精度面でもそれまでの天動説より優位に立てたわけではなかった。つまり、コペルニクスの時点では地動説は完成していなかったのだ。地動説の完成は、ブラーエのデータをもとに天体の軌道を計算し、その軌道上で面積速度一定の法則などを発見したケプラー(1571-1630)の登場や、慣性の法則や金星の満ち欠けを発見したガリレイ(1564-1642)の登場、そして微積分法・古典力学・万有引力の法則を発見したニュートン(1642-1727)らの登場を待たねばならない。

ただし、それらの発見によって地動説が完成し、天体の運動がより簡潔に説明できるようになったとしても、それは天動説が「論破」されたということを意味するわけではない。世界に起こる現象を説明するやり方は一通りではないからだ。例えば、「すべての物事は完全に自分中心で動いているのであって、そうでなうように思う他人はすべて錯覚に陥っているのだ。単純なモデルでは表せないことかもしれないが、実はそうなのだ」と矛盾なく言い張ることはできる。ただ、そうした不必要に複雑なモデルは、世界を説明する手段として採用する者がいなくなりがちだというだけだ。こうして、人々が採用する世界観において、支持者の喪失と獲得を通じた「パラダイム転換」が起こることになる。

\begin{itemize}
\tightlist
\item
  {[}\^{}1{]} クーン(1970=1971)を参照。
\end{itemize}

\subsection{この章のまとめ}\label{ux3053ux306eux7ae0ux306eux307eux3068ux3081}

あ
