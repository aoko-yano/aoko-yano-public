\section{同一性と変化}\label{ux7b2cux4e8cux7ae0ux540cux4e00ux6027ux3068ux5909ux5316}

\subsection{細胞・遺伝情報・セントラルドグマ}\label{ux7d30ux80deux907aux4f1dux60c5ux5831ux30bbux30f3ux30c8ux30e9ux30ebux30c9ux30b0ux30de}

地球上の生命の身体は、細胞という単位からできている。細胞は「DNA(デオキシリボ核酸)」という化学物質を中に持っているが、そのDNAは四種類の「ヌクレオチド」という単純な化学物質が並んで結合することでできている。その四種類のヌクレオチドの配列パターンによって、その細胞がどのように「自身を存続させる仕組み」を作動させるかの大枠が決まる。具体的には、DNAをもとに「RNA(リボ核酸)」という化学物質が生成され(=「転写」)、そのRNAを元にタンパク質が生成される(=「翻訳」)。タンパク質は約20種類のアミノ酸という単純な化学物質が並んで結合してできており、生体内の様々な化学反応において触媒として作用する。つまり、生体内の化学反応を司るタンパク質がどのようなアミノ酸の配列パターンを持つかが、元を正せばDNAにおけるヌクレオチドの配列パターンによって決まっているわけだ。その意味で、DNAは生体内の化学反応を規定している。

細胞は自己複製を行って増殖するが、このとき、DNAが持つ配列パターンも「複製」され、細胞の複製に伴って細胞が行う生体内の化学反応のあり方も複製されることになる。こうした事情から、DNAが持つ配列パターンは「遺伝情報」と呼ばれる。遺伝情報の総体を「ゲノム」と呼ぶが、これは通常、細胞内に含まれるすべてのDNA分子の配列パターンの総体のことを指している。このようにして行われる遺伝情報の複製が、第一章の最後に述べた「同一性の維持」のための一つ目の仕組みだ(\#2.1)。

ここまで述べてきたような「DNA→RNA→タンパク質」というゲノムから生体内の化学反応に至るまでの情報の流れと、「DNA→DNA」というゲノム自体の複製の流れを合わせて、「分子生物学のセントラルドグマ」という。地球上の生命は、基本的にセントラルドグマに沿った仕方で「自身を存続させる仕組み」を作動させている。なお、セントラルドグマに沿わない化学反応も存在している。例えば、「RNAの配列パターンに従ってDNAが作り出され(=『逆転写』)、そのDNAがある生物のDNAの中に挿入される」という現象もある。生命はニッチェとの相互作用の中から場当たり的に変化して生まれてきたものなので、単純な原理原則から逸脱する例外に満ちている。

\subsection{遺伝子と発現制御}\label{ux907aux4f1dux5b50ux3068ux767aux73feux5236ux5fa1}

このように書いたものの、生体内で合成されるタンパク質を構成するアミノ酸の配列パターンに対応しているのは、ゲノム中の配列パターンのうちの一部にすぎない。ゲノム中のタンパク質の配列パターンを直接規定している部分のことを、特に「遺伝子」と呼ぶ。

ゲノム中の遺伝子ではない部分は何をしているのかの全貌はまだ分かっていないが、その少なくない部分が「どのような条件下で、どの遺伝子からタンパク質を合成するか」を規定している。遺伝子からタンパク質が合成されることを、その遺伝子が「発現する」と表示し、そのタイミングなどを制御することを遺伝子の「発現制御」という。この発現制御が、生命に「変化」をもたらす二つの仕組みの一つだ(\#2.3)。

遺伝子とその発現制御について理解すると、日常生活の様々なことについても納得しやすくなる。例えば、いわゆる「お酒に強いか否か」は、各個人がその細胞のうちに持つアルコールデヒドロゲナーゼというアルコールを分解するタンパク質に関わる遺伝子上の配列パターンに、その原因の大部分を求めることができると分かっている。また、「筋トレをすると筋肉量が増える」のは、筋トレにより細胞内でアクチンとミオシンという筋肉の伸縮に関わるタンパク質がより多く合成されるようになったり、筋繊維の外側に張り付いているサテライト細胞という細胞が増殖して筋繊維を増やすようになるからだということが分かってきている。

このように、遺伝子を含むゲノムだけで身体がどのようになるかが決まるわけではないし、育った環境だけでそれが決まるわけでもない。良く話題に上がる「生まれか育ちか」という議論の背後には、このような生物学的プロセスがある。

\subsection{突然変異と種}\label{ux7a81ux7136ux5909ux7570ux3068ux7a2e}

どの遺伝子がどのように発現するのかは細胞が置かれた状況によって規定されているわけだが、そもそも「アルコールデヒドロゲナーゼには、アルコールをよく分解できるものと、うまく分解できないものがある」といったDNAの配列パターンの多様性はどこから現れるのか。その原因は、DNAに損傷が入ったり、あるいはDNAが複製される際に正しく複製されなかったりすることで変異が入ること(=突然変異)に求められる(\#2.4)。これが、生命に「変化」をもたらす二つの仕組みのもう一つだ。

ところで、通常では突然変異はすぐに修復される(\#2.2)のだが、この突然変異が次の時代まで残留することもある。具体的には、細胞一つで「個体」として生きていくことが前提となっている単細胞生物(例えば、細菌)の場合では、ある個体に入った変異がそのまま分裂などによって多くの個体に増えていくことになるし、同一の遺伝情報を持った細胞が複数集まった状態で「個体」を形成して生きていくことが前提となっている多細胞生物(例えば、人類)の場合では、親個体に入った変異が子孫へと伝わっていくことになる。これらのプロセスを通じて、様々な遺伝情報を持つ個体が現れるようになる。

個体間の遺伝情報の違いが大きくなると、それらは互いに似た形態をしなくなっていったり、似た生態をとらなくなっていったり、あるいは近しい個体間で成立する関係性が成立しなくなったりする。最後に挙げた「近しい個体間で成立する関係性」の代表的な例が、「交配して子孫を残せるか、生まれた子孫は親同様に他の近しい個体と交配してさらなる子孫を残すことができるか」というものだ。例えば、イヌとネコの間には子供はできないし、雄のロバと雌のウマから生まれるラバは子供を残すことができない一代限りの動物だ。このような、ある程度の違いの幅を持った複数の類似した個体の集合を「種」という。なお、地球上の生命一般に対して「何をもって同じ『種』とみなすか」を規定する明確な統一的な見解は存在しない。

ここまでの議論で、生命が持つ「同一性の維持」と「変化」のための三つの仕組みについて、その概略を説明できた。生命は、基本的には同一性を維持する仕組みに従って生存していく。だが、ある個体から受け継がれる遺伝情報は、突然変異による時間的な揺らぎを受けていく。その揺らぎの結果として、ある個体たちと他の個体たちとの間に種の違いという境界線が生じてくることになる。その上、各個体はその生の中で発現制御を通じて周囲の環境から影響を受け、変化していく。これらの仕組みによって、生命は、その生命が生きるニッチェと共に、揺れ動く環境の中で変化していくのだ。

この議論を前提として、次章からは、「ある人間個体に対して、遺伝子の発現制御によって後天的な変化を与える仕組み」の一つである「脳」の振る舞いへと議論の対象をズームインしていく。脳は、その発達を通じて環境への特に繊細な適応を可能にする。この脳の振る舞いを見ていくことで、人間の思考がどのような制限下にあるのかの輪郭を浮き上がらせることができるだろう。

\subsection{布石:種にとって寿命とは何か}\label{ux5e03ux77f3ux7a2eux306bux3068ux3063ux3066ux5bffux547dux3068ux306fux4f55ux304b}

だが、ここで第一章と同様に、視点を一度ズームアウトして「ある種にとって寿命とは何か」という問いについて考えておく。それは、そうすることが寿命を持つ種であるヒト(ホモ・サピエンス)を生きる私たちがいずれ死ぬ理由について考えることにもなるからだ{[}\^{}1{]}。

「種にとって寿命とは何か」を考えるために、寿命がない種を想定してみよう。寿命とは誕生後の時間経過と共に細胞の機能が低下していくことだと言えるので、ここではそうした機能低下が存在しないと前提してみる。すると、その種の個体は事故や病気以外では死亡しないことになる。このように想定すると、ここでは時間経過による細胞の機能低下が存在しないと想定しているので、年齢の低い世代からだけでなく、年齢の高い世代からも子孫が生まれることになる。この想定では、新しい時代に形成された遺伝情報だけからではなく、古い時代に形成された遺伝情報からも、新しい世代の遺伝情報が形成されることになる。そのため、種に新たに供給される遺伝情報は、現実世界のものと比べて、全体としてはより「古い」遺伝情報に偏ることになる。

これだけだと特に大きな問題にはならなさそうだが、ここに「ニッチェが許容する個体数の上限」を考慮の対象に追加すると、どうだろうか。ニッチェには許容する個体数に上限がある。例えば、種に供給される食料は、捕食される他の生物の量によって律速される。したがって、ある特定の種の個体数の上限値は寿命の長さにかかわらずそれほど変えられないということがわかる。この点を加味すると、寿命がない種では、寿命で死ぬ個体数が存在しない分だけ新たに出生する個体数が減っていなければならないことになる。さもなくば、その種は飢餓に見舞われてしまうだろう。

寿命がある場合と比較して「出生数が減り、さらに出生する新しい世代の遺伝情報がより『古い』ものに偏っている」となると、その種が持つ遺伝情報にもたらされる変化のスピードはより遅くなってしまう。新しい世代を生むのが仮に集団内の若い世代に限られる場合(つまり、生殖能力だけは老化すると仮定した場合)では、上記の事情は「出生数が減る」というだけなるので緩和されるが、その場合でも寿命がある場合よりは種の遺伝情報が変化するスピードは遅くならざるを得ない。

種の遺伝情報の変化速度が低下するということは、その種に属する個体の身体が変わる上限速度も低下するということを意味している。これは、急激な環境変動などに対応して身体を変化させることがより難しくなるという点で、生存に不利になる可能性がある。

ある種に寿命が存在することに上記のような背景があるのだとすれば、ここから翻って「寿命」と「世代交代」が、種の遺伝情報に変化を繰り込む手段の一つなのだと考えることができる(\#2.5)。世代交代は新たな遺伝情報が種に組み込まれる機会なのであり(\#2.6)、寿命が古い世代を排除することで、古い世代の遺伝情報が種から排除され、それが新たな遺伝情報が種に入り込む余地になるのだ(\#2.7)。種の遺伝情報には流動性が繰り込まれており、その流動性が環境とニッチェの中での種の変形を可能にする。

人類が持つ寿命も、そうした種に流動性を繰り込む仕組みに起源を持つと考えられる。それゆえ、人類から寿命による死を遠ざけることには、ヒトという種の遺伝情報に変化を繰り込む機会を減少させることを意味するだろう。その場合、遺伝情報の変化という観点からみれば、ヒトはより硬直した、環境とニッチェの変化に対して脆弱な存在になる可能性が考えられる。

もちろん、「世界がどのようになっているか」という事実は「人がどのように生きるべきか」という当為を一切導かない。また、ここでも第一章と同様に、技術的な工夫が隘路を切り開く可能性は否定できない。それでも、この事実を認識することで、それに逆らうことにいかなる困難が伴うのかを明確に認識できるようになる。このような困難についての認識は、私たちが自らの生き方を構想し、また世界についての納得できる物語を作り上げる際に、野放図に空想が膨らむのを避ける役に立つことだろう。

\begin{itemize}
\tightlist
\item
  {[}\^{}1{]} 以下の議論については主に小林(2021)を参考にした。
\end{itemize}

\subsection{この章のまとめ}\label{ux3053ux306eux7ae0ux306eux307eux3068ux3081}

環境の変化に合わせて生き方を変え続けるために、生命は自身に変化を繰り込む仕組みを持つ必要がある。そうした変化を繰り込む仕組みとして、ヒトは脳を持ち、寿命を持つ。こうして、ヒトの有限性に寿命という束縛が加わりつつも、脳が可能にする柔軟性も与えられる。本稿は、この条件を踏まえた上で探求される。
