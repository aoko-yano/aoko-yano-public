\section{エディプス・コンプレックスの諸段階}\label{ux30a8ux30c7ux30a3ux30d7ux30b9ux30b3ux30f3ux30d7ux30ecux30c3ux30afux30b9ux306eux8af8ux6bb5ux968e}

\subsection{幻想の形成過程}\label{ux5e7bux60f3ux306eux5f62ux6210ux904eux7a0b}

幻想の形成は、下記のように行われる。まず、幼児期において予期されない体験としての対象\(a\)が顕現した際に、養育者がそれを繰り返し解決するという前段階がある。その段階において、養育者は対象\(a\)の解消と結びつけて認識される。養育者は様々な感覚的特徴を持ち、また様々な働きかけを幼児に対して繰り返し行う。養育者の現前と養育者の様々な働きかけは、幼児の脳の中で対象\(a\)の解消(=満足)と結びついたシニフィアンとして蓄積されていく。そうして形成されるシニフィアンの体系が「大他者(=A)(=『(精神分析的)母』)」として前象徴界を形成する(=「前エディプス期」)。

\begin{note}{}
  \begin{itemize}
    \tightlist
    \item{\#7.1}幻想の形成は、下記のように行われる。
    \item{\#7.2}まず、幼児期において予期されない体験としての対象$a$が顕現した際に、養育者がそれを繰り返し解決するという前段階がある。
    \item{\#7.3}その段階において、養育者は対象$a$の解消と結びつけて認識される。
    \item{\#7.4}養育者は様々な感覚的特徴を持ち、また様々な働きかけを幼児に対して繰り返し行う。
    \item{\#7.5}養育者の現前と養育者の様々な働きかけは、幼児の脳の中で対象aの解消(=満足)と結びついたシニフィアンとして蓄積されていく。
    \item{\#7.6}そうして形成されるシニフィアンの体系が「大他者(=A)(=『(精神分析的)母』)」として前象徴界を形成する(=「前エディプス期」)。
  \end{itemize}
\end{note}

\subsection{エディプス第一の時}\label{ux30a8ux30c7ux30a3ux30d7ux30b9ux7b2cux4e00ux306eux6642}

しかし、大他者は以下の二点で対象\(a\)を十全に解消することはない: -
シニフィアンは体験との予測誤差をゼロにすることはない -
養育者は現前と不在を繰り返し、幼児を不安にさせる

上記二点が「大他者の非一貫性(=Ⱥ)」を形成する(=「不満」)。そこで、幼児は「大他者を一貫したものにする要素(=「ファルス」)」を探し求める。このとき、幼児に取ってファルスは「自分が『それ』になることができるかもしれないもの」としての「想像的ファルス」として現れている。

\begin{note}{}
  \begin{itemize}
    \tightlist
    \item{\#7.7}しかし、大他者は以下の二点で対象aを十全に解消することはない。
      \begin{itemize}
        \tightlist
        \item シニフィアンは体験との予測誤差をゼロにすることはない
        \item 養育者は現前と不在を繰り返し、幼児を不安にさせる
      \end{itemize}上記二点が「大他者の非一貫性(=Ⱥ)」を形成する(=「不満」)。
    \item{\#7.8}そこで、幼児は「大他者を一貫したものにする要素(=「ファルス」)」を探し求める。
    \item{\#7.9}このとき、幼児に取ってファルスは「自分が『それ』になることができるかもしれないもの」としての「想像的ファルス」として現れている。
  \end{itemize}
\end{note}

\subsection{エディプス第二の時における想像的父による剥奪}\label{ux30a8ux30c7ux30a3ux30d7ux30b9ux7b2cux4e8cux306eux6642ux306bux304aux3051ux308bux60f3ux50cfux7684ux7236ux306bux3088ux308bux5265ux596a}

エディプス第一の時において、養育者が「養育者の現前と不在を司る対象(=「(精神分析的)父」)」をシニフィアンとして幼児に示すとき、幼児は「父」を用いた幻想の構築を開始する。父が父として幼児に作用するためには、父は幼児の前に現前するものから超越していなければならないため、父は幼児の前に現前してはならない。この段階では、幼児は父を「大他者からファルスを『剥奪』した『想像的父』」として解釈するようになる。

\begin{note}{}
  \begin{itemize}
    \tightlist
    \item{\#7.10}エディプス第一の時において、養育者が「養育者の現前と不在を司る対象(=「(精神分析的)父」)」をシニフィアンとして幼児に示すとき、幼児は「父」を用いた幻想の構築を開始する。
    \item{\#7.11}父が父として幼児に作用するためには、父は幼児の前に現前するものから超越していなければならないため、父は幼児の前に現前してはならない。
    \item{\#7.12}まず、幼児は父を「大他者からファルスを『剥奪』した『想像的父』」として解釈するようになる。
  \end{itemize}
\end{note}

\subsection{エディプス第三の時における現実的父による去勢}\label{ux30a8ux30c7ux30a3ux30d7ux30b9ux7b2cux4e09ux306eux6642ux306bux304aux3051ux308bux73feux5b9fux7684ux7236ux306bux3088ux308bux53bbux52e2}

しかし、このとき父は「ファルスを持つ者」としても現れている。その側面を受容するとき、幼児はファルスの存在をファルスが現前しない状況のまま信じられるようになるので、幼児は大他者の非一貫性を大他者の本質として認められるようになる(=大他者の「去勢」を受け入れる)(=S(Ⱥ))。幼児に大他者の去勢を認めさせる者としての父を「現実的父」と呼ぶ。

\begin{note}{}
  \begin{itemize}
    \tightlist
    \item{\#7.13}しかし、このとき父は「ファルスを持つ者」としても現れている。その側面を受容するとき、幼児はファルスの存在をファルスが現前しない状況のまま信じられるようになるので、幼児は大他者の非一貫性を大他者の本質として認められるようになる(=大他者の「去勢」を受け入れる)(=S(Ⱥ))。
    \item{\#7.14}幼児に大他者の去勢を認めさせる者としての父を「現実的父」と呼ぶ。
  \end{itemize}
\end{note}

\subsection{象徴的父・象徴的ファルス・欲望の主体}\label{ux8c61ux5fb4ux7684ux7236ux8c61ux5fb4ux7684ux30d5ux30a1ux30ebux30b9ux6b32ux671bux306eux4e3bux4f53}

父がファルスを持つと解釈されるとき、父は超越的な「法」によって大他者を統御する者と解釈されるようになる。このような父を「象徴的父(=『父の名』)」と呼ぶ。父が持つ法の根拠としてのファルスは「象徴的ファルス」と呼ばれる。これは、幼児が自身の対象\(a\)について「父および父の持つファルスを用いることで究極的には解決可能なものである」と解釈できるようになることと等価である。そこから、主体は対象\(a\)を解消するために自身もファルスを持つことを「欲望」するようになる(=「欲望の主体」の誕生)。

\begin{note}{}
  \begin{itemize}
    \tightlist
    \item{\#7.15}父がファルスを持つと解釈されるとき、父は超越的な「法」によって大他者を統御する者と解釈されるようになる。
    \item{\#7.16}このような父を「象徴的父(=『父の名』)」と呼ぶ。
    \item{\#7.17}父が持つ法の根拠としてのファルスは「象徴的ファルス」と呼ばれる。
    \item{\#7.18}これは、幼児が自身の対象aについて「父および父の持つファルスを用いることで究極的には解決可能なものである」と解釈できるようになることと等価である。
    \item{\#7.19}そこから、主体は対象aを解消するために自身もファルスを持つことを「欲望」するようになる(=「欲望の主体」の誕生)。
  \end{itemize}
\end{note}

\subsection{精神分析的な男と女}\label{ux7cbeux795eux5206ux6790ux7684ux306aux7537ux3068ux5973}

現実的父に同一化し、自身も象徴的ファルスを父のように持とうとする主体を「(精神分析的)男」という。

象徴的ファルスに同一化し、ファルスを持つ現実的父に欲望されることでファルスを間接的に持とうとする主体を「(精神分析的)女」という。

\begin{note}{}
  \begin{itemize}
    \tightlist
    \item{\#7.20}現実的父に同一化し、自身も象徴的ファルスを父のように持とうとする主体を「(精神分析的)男」という。
    \item{\#7.21}象徴的ファルスに同一化し、ファルスを持つ現実的父に欲望されることでファルスを間接的に持とうとする主体を「(精神分析的)女」という。
  \end{itemize}
\end{note}

\subsection{この章のまとめ}\label{ux3053ux306eux7ae0ux306eux307eux3068ux3081}

あ
