\section{結論3:満足を可能にする生態系の必要性}\label{ux7d50ux8ad6uxff13ux6e80ux8db3ux3092ux53efux80fdux306bux3059ux308bux751fux614bux7cfbux306eux5fc5ux8981ux6027}

\subsection{「優れた生き方」の不在}\label{ux512aux308cux305fux751fux304dux65b9ux306eux4e0dux5728}

あ

\subsection{複数の生き方の並存可能性}\label{ux8907ux6570ux306eux751fux304dux65b9ux306eux4e26ux5b58ux53efux80fdux6027}

あ

\subsection{目標3:複数の世界観を共存させる一つの生態系の構想}\label{ux76eeux6a19uxff13ux8907ux6570ux306eux4e16ux754cux89b3ux3092ux5171ux5b58ux3055ux305bux308bux4e00ux3064ux306eux751fux614bux7cfbux306eux69cbux60f3}

善のイデア(プラトン)・コナトゥス(スピノザ)・力への意思(ニーチェ)の復権
唯一の普遍的思想としての「意思」の再発見

物質循環とエネルギー 生命が問いとしての大義となるべきだ
