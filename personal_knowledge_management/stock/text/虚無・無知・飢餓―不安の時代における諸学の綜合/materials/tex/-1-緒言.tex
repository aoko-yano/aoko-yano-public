% Options for packages loaded elsewhere
\PassOptionsToPackage{unicode}{hyperref}
\PassOptionsToPackage{hyphens}{url}
%
\documentclass[
]{ltjsarticle}
\usepackage{amsmath,amssymb}
\usepackage{iftex}
\ifPDFTeX
  \usepackage[T1]{fontenc}
  \usepackage[utf8]{inputenc}
  \usepackage{textcomp} % provide euro and other symbols
\else % if luatex or xetex
  \usepackage{unicode-math} % this also loads fontspec
  \defaultfontfeatures{Scale=MatchLowercase}
  \defaultfontfeatures[\rmfamily]{Ligatures=TeX,Scale=1}
\fi
\usepackage{lmodern}
\ifPDFTeX\else
  % xetex/luatex font selection
\fi
% Use upquote if available, for straight quotes in verbatim environments
\IfFileExists{upquote.sty}{\usepackage{upquote}}{}
\IfFileExists{microtype.sty}{% use microtype if available
  \usepackage[]{microtype}
  \UseMicrotypeSet[protrusion]{basicmath} % disable protrusion for tt fonts
}{}
\makeatletter
\@ifundefined{KOMAClassName}{% if non-KOMA class
  \IfFileExists{parskip.sty}{%
    \usepackage{parskip}
  }{% else
    \setlength{\parindent}{0pt}
    \setlength{\parskip}{6pt plus 2pt minus 1pt}}
}{% if KOMA class
  \KOMAoptions{parskip=half}}
\makeatother
\usepackage{xcolor}
\setlength{\emergencystretch}{3em} % prevent overfull lines
\providecommand{\tightlist}{%
  \setlength{\itemsep}{0pt}\setlength{\parskip}{0pt}}
\setcounter{secnumdepth}{-\maxdimen} % remove section numbering
\usepackage{bookmark}
\IfFileExists{xurl.sty}{\usepackage{xurl}}{} % add URL line breaks if available
\urlstyle{same}
\hypersetup{
  hidelinks,
  pdfcreator={LaTeX via pandoc}}

\author{}
\date{}

\begin{document}

本稿において、「序論」が「問題設定とそれに対する答え方の構え」を示すものである一方で、「緒言」は「本稿が書かれた時代の知の配置の中で、本稿がどのような位置づけの考察なのか」を示すものだ。本稿は、一体何なのか。誰のために何をするのか。読むことで何が得られるのか。

まず第一に、本稿は複数の専門分野で発見された知識を相互に束縛させあい、接続させあう。それによって、本稿は複数の専門分野を一つに繋ぎ合わせる。そうすることで、浮かび上がり指し示される道筋というものがあるからだ。

そして第二に、本稿は、その際に各専門分野の基本的な知識しか用いない。それは、本稿が示す道筋が、実は現代人の眼前に当然のごとく在り続けていたにもかかわらず、ほとんど誰にも顧みられなかったものだからだ。

加えて第三に、本稿は特定の専門分野に直接寄与するものではないし、文学的なテクストによって人々の情動を揺さぶるものでもない。それらは本稿の目的ではないからだ。

それゆえに、特定の専門分野への寄与や文学的なテクストを期待する読者に、本稿が与えられるものはない。それにもかかわらず、本稿が示す道筋は、現代人の知識を整理することによって、様々な専門分野や文学に従事する人々を含む幅広い層に寄与するだろう。

専門細分化が進み、その全貌を大掴みにでも把握することが困難になった時代において、本稿は人々が混乱の中に沈むことを防ぐ小舟のような存在となるはずだ。その小舟の上からは、世の喧騒が如何に簡単な無理解に立脚したものであるかがすぐに分かるようになる。そして、本質的に未解決な問題については地道な探究を重ねるしかなく、そこに魔法のような解決があることも稀だということも分かるようになる。

すると、次第に静寂が得られるようになってくる。この静寂において、読者の前にはこの宇宙の中で、人間が如何に小さな存在であるかが分かるようになるだろう。そこでは宇宙が、古の人間が思い描いたであろうような意味に満ちたものではなく、茫漠とした無根拠と無意味として現れてくる。

しかし、これは本稿が示すことだが、読者はそこでただ単にすべてから切り離された根無し草になるわけではない。読者はそこで、自身の身体に出会う。その身体とは、自身がそこでそのように感じ、そう欲する以外の仕方では欲することができないような「固着した飢え」である。

飢えは救いではないが、飢えを無視することは苦しみを生む。固着した飢えという立脚点を得て、その渇きを癒すために、幸運ならば、読者は本稿という知識の小舟から無知が渦巻く宇宙という大海へと最後は飛び込むだろう。本稿が提供する知識の体系は、その際に自身の身を混沌から守る一助にもなるだろう。本稿は、そのようなものだ。

\end{document}
