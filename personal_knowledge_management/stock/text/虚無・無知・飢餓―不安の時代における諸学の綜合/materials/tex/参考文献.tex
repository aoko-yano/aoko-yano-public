\section{参考文献}\label{ux53c2ux8003ux6587ux732e}

\subsection{序論}\label{ux5e8fux8ad6}

\begin{itemize}
\tightlist
\item
  フリードリヒ・ニーチェ,三島憲一(訳),1974=1984,『ニーチェ全集 第九巻(第Ⅱ期) 遺された断想(一八八五年秋--八七年秋)』白水社.
\item
  --,木場深定(訳),1887=1940,『道徳の系譜』岩波書店.
\item
  ジャン=フランソワ・リオタール,小林康夫(訳),1979=1989,『ポストモダンの条件』水声社.
\item
  ハンス・ロスリング/オーラ・ロスリング/アンナ・ロスリング・ロンランド,上杉周作/関美和(訳),2018=2019,『FACTFULNESS(ファクトフルネス)
  10の思い込みを乗り越え、データを基に世界を正しく見る習慣』日経BP.
\item
  浅田彰,1983,『構造と力--記号論を超えて』勁草書房.
\item
  市川遊佐,2021,「ラカニアン・アジャイル―『四つのディスクール』から考える中間集団論/組織論としての『スクラム』」(『アレ』Vol.9収録,アレ★Club:240-313).
\item
  落合陽一,2015,『魔法の世紀』PLANETS.
\item
  小泉悠/桒原響子/小宮山功一郎,2023,『偽情報戦争 あなたの頭の中で起こる戦い』ウェッジ.
\item
  津田正太郎,2024,「メディア社会とは何か 1. 国民国家とマスメディア―トイ人」(2024年8月21日取得,\url{https://www.toibito.com/toibito/articles/\%E5\%9B\%BD\%E6\%B0\%91\%E5\%9B\%BD\%E5\%AE\%B6\%E3\%81\%A8\%E3\%83\%9E\%E3\%82\%B9\%E3\%83\%A1\%E3\%83\%87\%E3\%82\%A3\%E3\%82\%A2}).
\item
  松本卓也,2018,『享楽社会論』人文書院.
\end{itemize}

\subsection{第一章}\label{ux7b2cux4e00ux7ae0}

\begin{itemize}
\tightlist
\item
\end{itemize}

\subsection{第二章}\label{ux7b2cux4e8cux7ae0}

\begin{itemize}
\tightlist
\item
  小林武彦,2021,『生物はなぜ死ぬのか』講談社.
\end{itemize}

\subsection{第三章}\label{ux7b2cux4e09ux7ae0}

\begin{itemize}
\tightlist
\item
  乾敏郎・坂口豊,2020,『脳の大統一理論―自由エネルギー原理とはなにか』岩波書店.
\item
  気象庁,2023,「気象庁|過去の気象データ検索」(2024年9月12日取得,\url{https://www.data.jma.go.jp/obd/stats/etrn/view/daily_h1.php?prec_no=44&block_no=00&year=2023&month=08&day=&view=p3}).
\item
  小坂修平,2004,『図解雑学現代思想』ナツメ社.
\end{itemize}

\subsection{第四章}\label{ux7b2cux56dbux7ae0}

\begin{itemize}
\tightlist
\item
\end{itemize}

\subsection{第五章}\label{ux7b2cux4e94ux7ae0}

\begin{itemize}
\tightlist
\item
\end{itemize}

\subsection{第六章}\label{ux7b2cux516dux7ae0}

\begin{itemize}
\tightlist
\item
\end{itemize}

\subsection{第七章}\label{ux7b2cux4e03ux7ae0}

\begin{itemize}
\tightlist
\item
\end{itemize}

\subsection{第八章}\label{ux7b2cux516bux7ae0}

\begin{itemize}
\tightlist
\item
  向井雅明,2016,『ラカン入門』筑摩書房.
\item
  トーマス・クーン,中山茂(訳),1970=1971,『科学革命の構造』みすず書房.
\item
  フレッド・ホイル,中島龍三(訳),1973=1974,『コペルニクス:
  その生涯と業績』法政大学出版局.
\end{itemize}

\subsection{第九章}\label{ux7b2cux4e5dux7ae0}

\begin{itemize}
\tightlist
\item
  市谷聡啓,2018,『カイゼンジャーニー―
  たった1人からはじめて、「越境」するチームをつくるまで』翔泳社.
\item
  今村仁司,2024,『仕事』講談社.
\item
  小田中育生,2023,「Be
  Agile~-アジャイルマインドセットでいきいきと働く-」(2024年10月23日取得,\url{https://speakerdeck.com/ikuodanaka/be-agile-and-work-with-a-iki-iki-spirit})
\item
  斎藤幸平,2023,『マルクス解体―プロメテウスの夢とその先』講談社.
\end{itemize}

\subsection{第十章}\label{ux7b2cux5341ux7ae0}

\begin{itemize}
\tightlist
\item
\end{itemize}

\subsection{第十一章}\label{ux7b2cux5341ux4e00ux7ae0}

\begin{itemize}
\tightlist
\item
\end{itemize}

\subsection{結論1}\label{ux7d50ux8ad6uxff11}

\begin{itemize}
\tightlist
\item
\end{itemize}

\subsection{結論2}\label{ux7d50ux8ad6uxff12}

\begin{itemize}
\tightlist
\item
\end{itemize}

\subsection{結論3}\label{ux7d50ux8ad6uxff13}

\begin{itemize}
\tightlist
\item
\end{itemize}
