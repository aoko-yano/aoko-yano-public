\subsection{第十章:エンジニアリングが持つダイナミズムからの疎外の結果1(抑圧と反抗)}\label{ux7b2cux5341ux7ae0ux30a8ux30f3ux30b8ux30cbux30a2ux30eaux30f3ux30b0ux304cux6301ux3064ux30c0ux30a4ux30caux30dfux30baux30e0ux304bux3089ux306eux758eux5916ux306eux7d50ux679cuxff11ux6291ux5727ux3068ux53cdux6297}

\subsubsection{複雑化・硬直・プレモダン}\label{ux8907ux96d1ux5316ux786cux76f4ux30d7ux30ecux30e2ux30c0ux30f3}

あ

\subsubsection{主人のディスクール}\label{ux4e3bux4ebaux306eux30c7ux30a3ux30b9ux30afux30fcux30eb}

あ

\[
\uparrow\frac{\mathrm{S_1}}{\mathrm{\cancel{S}}}\genfrac{}{}{0pt}{}{\longrightarrow}{//}\frac{\mathrm{S_2}}{a}\downarrow
\]

\subsubsection{大学のディスクール}

あ

\[
\uparrow\frac{\mathrm{S_2}}{\mathrm{S_1}}\genfrac{}{}{0pt}{}{\longrightarrow}{//}\frac{a}{\mathrm{\cancel{S}}}\downarrow
\]

\subsubsection{ヒステリー者のディスクール}\label{ux30d2ux30b9ux30c6ux30eaux30fcux8005ux306eux30c7ux30a3ux30b9ux30afux30fcux30eb}

あ

\[
\uparrow\frac{\mathrm{\cancel{S}}}{a}\genfrac{}{}{0pt}{}{\longrightarrow}{//}\frac{\mathrm{S_1}}{\mathrm{S_2}}\downarrow
\]

\subsubsection{分析家のディスクール}

あ

\[
\uparrow\frac{a}{\mathrm{S_2}}\genfrac{}{}{0pt}{}{\longrightarrow}{//}\frac{\mathrm{\cancel{S}}}{\mathrm{S_1}}\downarrow
\]

\subsubsection{硬直したプレモダン的労働における疎外}\label{ux786cux76f4ux3057ux305fux30d7ux30ecux30e2ux30c0ux30f3ux7684ux52b4ux50cdux306bux304aux3051ux308bux758eux5916}

フォーディズム・設計主義
人間は機械の一部として量的に扱われる(リソース(=資材)としての労働力)
精神分析は帝国主義とフォーディズムの時代の産物かも
