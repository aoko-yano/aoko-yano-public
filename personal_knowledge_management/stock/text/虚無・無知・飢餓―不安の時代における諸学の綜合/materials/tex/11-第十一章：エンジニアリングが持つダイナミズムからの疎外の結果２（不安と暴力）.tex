\subsection{第十一章:エンジニアリングが持つダイナミズムからの疎外の結果2(不安と暴力)}\label{ux7b2cux5341ux4e00ux7ae0ux30a8ux30f3ux30b8ux30cbux30a2ux30eaux30f3ux30b0ux304cux6301ux3064ux30c0ux30a4ux30caux30dfux30baux30e0ux304bux3089ux306eux758eux5916ux306eux7d50ux679cuxff12ux4e0dux5b89ux3068ux66b4ux529b}

\subsubsection{〈父の名〉の衰退}\label{ux7236ux306eux540dux306eux8870ux9000}

あ

\subsubsection{貨幣による秩序}\label{ux8ca8ux5e63ux306bux3088ux308bux79e9ux5e8f}

資本主義に本質的な問題と、プレモダンな専制に本質的な問題とを混同してはならない。資本主義の本質は、その運動を支える「素材」としての諸物をよりよく理解し、より効率よく利用することにある。それは労働力としての人間の扱いについても然りだ。資本主義の激化に伴い発生する労働問題として、労働者が

\subsubsection{不安・排除・レイシズム}\label{ux4e0dux5b89ux6392ux9664ux30ecux30a4ux30b7ux30baux30e0}

あ

\subsubsection{資本主義のディスクール}\label{ux8cc7ux672cux4e3bux7fa9ux306eux30c7ux30a3ux30b9ux30afux30fcux30eb}

双数性

\subsubsection{柔軟性のあるモダンな労働における疎外}\label{ux67d4ux8edfux6027ux306eux3042ux308bux30e2ux30c0ux30f3ux306aux52b4ux50cdux306bux304aux3051ux308bux758eux5916}

モダンな労働における疎外は、欲動の外から与えられた欲望にすり替えられることで起こる
プレモダンな労働と同じく、モダンな労働でも人は生産・消費の両面で量的なリソース(資材)として扱われる

コモンが搾取や疎外を避けるわけではない。株主からかかる利潤第一主義への圧力に従属しないことがコモンズの利点だと斉藤は主張するが、それは株式会社でも可能だ。アマゾンを見ろ。株主を説得できるだけのビジョンがない知的怠惰がダメなのであって、株式会社がダメなわけではない。コモンの中でも疎外と搾取は発生しうる。改善のプロセスから逃げてはならない。

資本主義をマルクス・ガブリエルが言う「倫理資本主義」に近いものに変えると考えられる。

\subsubsection{大義の条件}\label{ux5927ux7fa9ux306eux6761ux4ef6}

何かに対する敵対は、原理としての大義にはならない
大義は問いであり、顕現しない(顕現した原理は専制である)
