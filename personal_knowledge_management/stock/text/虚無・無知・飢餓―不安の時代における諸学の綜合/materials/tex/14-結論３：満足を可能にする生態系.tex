\subsection{結論3:満足を可能にする生態系}\label{ux7d50ux8ad6uxff13ux6e80ux8db3ux3092ux53efux80fdux306bux3059ux308bux751fux614bux7cfb}

\subsubsection{目標3:複数の世界観を共存させる一つの生態系の構想}\label{ux76eeux6a19uxff13ux8907ux6570ux306eux4e16ux754cux89b3ux3092ux5171ux5b58ux3055ux305bux308bux4e00ux3064ux306eux751fux614bux7cfbux306eux69cbux60f3}

善のイデア(プラトン)・コナトゥス(スピノザ)・力への意思(ニーチェ)の復権
唯一の普遍的思想としての「意思」の再発見

物質循環とエネルギー 生命が問いとしての大義となるべきだ
