\subsection{結論2:新しい物語の推奨ボーダーライン}\label{ux7d50ux8ad6uxff12ux65b0ux3057ux3044ux7269ux8a9eux306eux63a8ux5968ux30dcux30fcux30c0ux30fcux30e9ux30a4ux30f3}

\subsubsection{目標2-1:論点の整理}\label{ux76eeux6a19uxff12uxff11ux8ad6ux70b9ux306eux6574ux7406}

こうして、満足の四つのあり方を整理することができた。この整理をもとに、ここからは「新しい物語の『推奨ボーダーライン』」について考察していく。そのためには、新しい物語の要件を再確認する必要がある。その要件は、まず第一に物語が解決すべき問題を確認した上で、続いて第二に現代社会における人間と物語との関係を確認することで確認できる。

まず第一に、物語が解決すべき問題について確認する。その問題とは、そこに意味が存在しないことに耐えられないような謎が存在することだ。人が持つ知はシニフィアンの体系に依存しており、そしてシニフィアンの体系は常に不完全なものである以上、人の知も常に不完全であり、それゆえ人は常に謎(対象a)と共にある。そして、謎のうちには、悲惨や苦痛に満ちた、人を途方に暮れさせるものがある。そうした耐えがたい謎を前にして、人は物語を求めるようになるのだ。

続いて第二に、現代社会における人間と物語の関係を確認する。その関係は、冒頭の「問題意識あるいは問いの設定」において「過去より存在する物語を信仰し続けるのには筋金入りの信心が必要であり、ナイーブに未来を信じるのには事態は差し迫りすぎ」ていると記述した通りだ。本稿のここまでの議論とこの記述を踏まえて新しい物語の要件を洗練化させると、それは「科学的知見と整合的であり、人類の未来はナイーブに楽観できるわけではないことを認めた上で、人の生に意味を与える物語」だということになる。そのような新しい物語があれば、それに救われる人もいるだろう。

\subsubsection{目標2-2:考察と方法(1:物語を求める前に)}\label{ux76eeux6a19uxff12uxff12ux8003ux5bdfux3068ux65b9ux6cd5uxff11ux7269ux8a9eux3092ux6c42ux3081ux308bux524dux306b}

それでは、そうした新しい物語は、具体的にどのような科学的知見や未来観に束縛されているのだろうか。その問いに答える前に、\textbf{根源的な欲動}のレベルでの満足に触れておこう。

まず指摘しなければならないのは、「人が常に謎と共に在る」からといって、「人が常に物語を求める」わけではないということだ。実際、意味を求めることなしに欲動が解消され、満足が得られる場合がある。それは第九章で述べたように、運動や芸術などによって人が充実する場合があるからだ。例えば、人は忙しく身体を動かしているとき、意味を求めようとすることを忘れてしまう。また、友人とおしゃべりをしているときや気晴らしにやるゲームに熱中しているときなどには、少なくない人が人生の謎について考えることを忘れている。他にも、芸術作品の制作に熱中したり、あるいは芸術作品をのめり込むように鑑賞している時に、人は自身が抱える謎から距離を取ることができる。そして、いずれの場合でも、人はそれらの行為によって「スッキリ」することができる。この論点から分かることは、人は運動や芸術を通じて欲動を解消することで、向き合うべき謎を軽減し、抱える問題を軽くできるということだ。

身も蓋もない話ではあるが、抱える問題をなるべく軽くすることを軽視してはいけないと筆者は考えている。なぜなら、抱える問題を軽減することを怠れば、その分だけ、その問題を意味を用いて合理化しなければならなくなるからだ。物語を信仰することは、結局のところ合理化であり、現実に対する防衛にすぎない。そして、防衛というのは、情動を消すことができるわけではなく、情動に上書きされる思考を曲げるものにすぎないのだった。だから、合理化すればするだけ、自分自身から目を背けることが増えていく。そうすると、ますます自身を苦境に追い込んでいき、本来ならば抱えずに済んだ問題まで抱えることにもなりかねないわけである。

しかし一方で、運動や芸術によって欲動を解消することについて、哲学は無条件な称揚を避けてきたように見える。この点については、どう考えるべきか。筆者が考えるところによれば、それは、運動や芸術といった手段によってのみ欲動を解消し続けた場合、運動や芸術以外の対象についての知識や思考を発達させることが難しくなるからだ。もしそこで人生についての知識や思考の発達が阻害されれば、いざ自身の生に謎が現れた際になす術がなくなってしまう。

こうした知識や思考の発達が阻害される状況は、運動や芸術への一面的な耽溺のみによって起こるわけではない。実際、\textbf{権威主義的な世界観}の下では「世界の多くの物事に対して固定的な意味が規定され、様々な行動が禁止されていることで、人は知識や思考を発達させる契機を奪われている」のであり、\textbf{資本主義的な体制}の下では「お仕着せの欲望をあてがうことによって欲動を解消させてしまうことで、人は自分自身の欲望の核を探求する必要がなくなり、\textbf{特異性}について知識や思考を発達させる契機を失うよう仕向けられている」のだった。

\subsubsection{「欲動の解消」・「知識と思考の発達」・「特異性の探求」の三面を並立させる方法}\label{ux6b32ux52d5ux306eux89e3ux6d88ux77e5ux8b58ux3068ux601dux8003ux306eux767aux9054ux7279ux7570ux6027ux306eux63a2ux6c42ux306eux4e09ux9762ux3092ux4e26ux7acbux3055ux305bux308bux65b9ux6cd5}

この困難を突破するためには、どうすれば良いのか。言い換えれば、運動や芸術による欲動の解消を活用しつつも、知識や思考を発達させ、その過程で自分自身の\textbf{特異性}を見つけ出していくには、どうすれば良いのか。

そこで注目できるのが、\textbf{資本主義的な体制}の下での\textbf{基本的な力}の行使による秩序の構築と放棄の繰り返しだ。具体的に言えば、労働の場の中で自身の知識や思考を「ああでもない、こうでもない」と試行錯誤を繰り返すということだ。この繰り返しを通じて、人は自身の知識や思考を自然や社会に適応させるために洗練させ続けることができる。特に、\textbf{資本主義的な体制}が労働者に対して発揮する労働への圧力ゆえに、人は自分の限界がどのあたりにあるのかを比較的容易に知ることができる。それは、自分の身体の限界であり、自分の欲望の限界だろう。自分自身に担えるものの重さの上限であり、自分自身が「まだやりたい!」と思える範囲と深さの輪郭だ。それらの限界を知ることで、自分自身の振れ幅を知ることに繋がり、そこから自分自身の「求めずにはいられないもの」が浮かび上がってくる。労働は、自身の核となる部分にある愛を探し求めるプロセスにもなりうるのだ。

とはいえ、もちろんこれは相当に器用な生き方だと言わざるを得ない。何故ならば、そのような生き方を実践するためには、第一に自分が「やりたい」と感じた労働に身を投じるための能力と、第二に自身の限界を速やかに感じ取る能力と、第三に自身の限界を感じ取った際に禍根を残さず速やかに撤収する能力が必要だからだ。それらの能力を身に着けるためにはそれだけで多大な才能と境遇と努力が必要になるだろう。これを万人に可能な生き方だと主張するのには無理がある。

先に述べた隘路を突破するための方法について次点として挙げられるのは、ワーク・ライフ・バランスを保つという月並みなものになるだろう。これは先の方法と比べれば(効率は劣るものの)より人を選ばない方法だと言うことができる。とはいえ、これも万人が選び取れる方法だとは言えないだろう。ワーク・ライフ・バランスを重視するだけの恵まれた労働にありつけない者もいるからだ。これも、この状況をどう改善すれば良いかというテーマだけで長大な考察が必要になってしまう。

そして、この二つの方法を実現できなかった人々だけれでなく、この二つの方法を実現できた人々であっても、人生において途方に暮れてしまう謎に突き当たってしまう場合がある(ヤスパースが述べるところの「限界状況」)。そのような謎に遭遇せずに、あるいはそうした謎に遭遇しても囚われずに上手く生きて死ぬ者は幸福だ。しかし、実際問題として、少なくない人が謎に突き当たって途方に暮れてしまうことになる。そこで初めて、物語が避け難く求められるようになるのだ。

\subsubsection{目標2-2:考察と方法(2:これからの物語)}\label{ux76eeux6a19uxff12uxff12ux8003ux5bdfux3068ux65b9ux6cd5uxff12ux3053ux308cux304bux3089ux306eux7269ux8a9e}

人はそうした限界状況においてどのような物語を求めれば良いのかについて早速考えていきたいところだが、その前に、本稿のそうした問いの立て方に対して当然あって然るべき次の問いに答えておこう。それは、「自分が信じる物語を自分で考えて作るのはナンセンスではないのか」という問いだ。この問いに対して本稿は、「『自分で考えて作り、それを自分で信じる』という行為に矛盾が起こらないような意味内容の物語は存在しうる。そして、そうした物語についてであれば、『自分で考えて作り、それを自分で信じる』を取ることには問題がない」と考えている。確かに、「神の啓示を受ける」ことから創始される啓示宗教などにおいては、「自分で考えて作り、それを自分で信じる」という行為に無理が生じる。しかし、それはあくまでその物語の持つ意味内容がそうした創始に矛盾するからにすぎない。「自分が考えた」という創始の経緯は、「それゆえにその物語は誤っている」という結論を帰結するわけではないのだ。例えば、クトゥルフ神話は創作神話であるが、このことが「クトゥルフ神話は世界の真理を表していない」という命題を証明するわけではない。

それでは改めて、限界状況においてどのような物語を求めれば良いかについて考えていこう。まず本稿が考える\textbf{これからの物語}に対して事実上課せられている制限は、冒頭に述べた通り、「この宇宙の中で起こる出来事について、物語が科学的知見と争わない」というものだ。もちろん、科学的知見と争うこと自由だが、そうした場合には「部外者も認める広い再現性を確認することができない主張」をすることになる。それでも(例えば、「この宇宙の歴史上、ただ一度きりしかその奇跡は起こらないのだ」などといったように)補助理論を追加することで誤りを認めないことは無際限に可能だが、これも冒頭に述べた通りの理由で、本稿はそうした態度を積極的には推奨しない。このように、本稿は「\textbf{これからの物語}は、超越的な領域についてのみ語るべきだ」という推奨ボーダーラインを提案する。

この推奨ボーダーラインに従った場合、物語の側が科学的知見に対して要求できる制限はほとんどないのだが、逆に科学的知見の側が物語に要求する制限はある。即ち、残酷な過去や現在についてだけでなく、科学的知見にしたがって導出される未来予測の範囲についても、\textbf{これからの物語}は遍く「何かしら救いのある」意味を与えられるようになっていなければならないのだ。例えば、現代の科学的知見と宇宙論は、相当高い確度で「この宇宙が永遠に生命の生存を許す場所であり続けることはない」ということを示している。その場合、あらゆる命の系譜は最終的に死と共に終わることになる。まずもって、その絶滅の過程に、\textbf{これからの物語}は意味を与えることができなければならない。他にも、科学的知見が\textbf{これからの物語}に対して要求する制限には様々なものがある(例えば、「他の諸物と同じく、生命現象も物理学の法則に従う」とか、あるいは「自由意志は錯覚である」とか\ldots\ldots)。その個別的な制限についてはここまでの議論を振り返って参照されたい。

こうした科学的知見が課す制限の多さにかかわわらず、\textbf{これからの物語}の具体的な実装は、無数にありうる。ここまで考えた上で、自分の考えと矛盾を来さない(あるいは、矛盾の少ない)既存の宗教を信じることにするのも可能だろう(そこで信じる宗教として啓示宗教を選ぶことも、論理的に可能だ)。あるいは、自分が信じることのできる物語をゼロから作り出すこともできる。

この選択の場面において、万人がその物語を進んで信じたいと思えるのか否かを考慮する必要性は必ずしもない。他の人がどう思うかは、その物語の真偽にも関わらないし、当人がそこに「救い」を見出だせるかにも関わらないからだ。したがって、世界を呪詛するのがその人にとって「救い」に感じられるのなら、そうした物語を選ぶことも可能だ。

\subsubsection{神が存在しないわけではない世界を生きる}\label{ux795eux304cux5b58ux5728ux3057ux306aux3044ux308fux3051ux3067ux306fux306aux3044ux4e16ux754cux3092ux751fux304dux308b}

理性が導けるのはここまでである以上、ここから先は「決めの問題」でしかない。このような結論に落ち着くと、「結局は理性からの(キルケゴールが述べるところの)『跳躍』に縋るしかないのなら、これまで行ってきた科学的知見や\textbf{特異性}についての探究は単なる遠回りだったのではないか?」と思われるかもしれないが、そうではない。

それは第一に、反社会的な嘘をそれと知っていてでっち上げる者は(歴史上これまでずっと)多く存在してきたが、地道な自己理解と科学的知見についての探究はそうした連中が弄する虚偽に騙されるリスクを少しは下げてくれるからだ(その上で、これもまた科学的知見がもたらす知識だが、自信過剰は虚偽に対する耐性を下げるがゆえに慎まねばならないことと、そしてどれほどの鍛錬を積んでも騙されるときは騙されるのだということを肝に銘じておくようにしよう)。

続いて第二に、自身の\textbf{特異性}をよりよく掴んでいれば、無数の物語の中から自身にとってより「しっくり」くる物語を選び取ることも、より容易になるだろうからだ。この意味でも、\textbf{特異性}の探究には効果があるわけだ。

そして第三に、科学的知見と自己理解に至るまでの仕組みについて知ることで、場合によっては物語を避け難く求めるしかない状況に陥りうるのだと理解できるからでもある。つまり、物語を避け難く求めることになるのは必ずしも不勉強や意志薄弱ゆえではないと理解できるからだ。そのように理解することで、どうしようもない状況を「これは本当に、自分にとってはどうしようもない状況なのだ」と自覚しなおすことができ、そこで「決然と」物語を信じるという決定を下せるようになるだろう。

それにしても、人類が生きている論理形式と世界構造が「自由な仕方で、超越的な存在を信じること」を許容するようにできているのは驚嘆すべきことであるように筆者には思われる。それらは何故か「端的にそうなっているから、そうなっているのだ」という仕方でそうなっているのであり、そうである必然性はないからだ。これは「世界が『その背面に神秘が存在する余地がある』という仕方で存在している」ことに対する驚愕である。人間は「神が存在しないわけではない」ような世界に住んでいるのだ。そのような世界に住んでいて、それを考え、求めてしまう時にそれを求めないでいるのは、強情でもあるように筆者には思われる。
