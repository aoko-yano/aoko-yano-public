% Options for packages loaded elsewhere
\PassOptionsToPackage{unicode}{hyperref}
\PassOptionsToPackage{hyphens}{url}
%
\documentclass[titlepage, 8pt, a5paper]{ltjsarticle}
\title{虚無・無知・飢餓\\―不安の時代における諸学の綜合―}
\author{Aoko Yano}
\date{2024-09-22}
\usepackage{amsmath,amssymb}
\usepackage{iftex}
\usepackage {pdfpages}
\usepackage {fancyhdr}
\ifPDFTeX
  \usepackage[T1]{fontenc}
  \usepackage[utf8]{inputenc}
  \usepackage{textcomp} % provide euro and other symbols
\else % if luatex or xetex
  \usepackage{unicode-math} % this also loads fontspec
  \defaultfontfeatures{Scale=MatchLowercase}
  \defaultfontfeatures[\rmfamily]{Ligatures=TeX,Scale=1}
\fi
\usepackage{lmodern}
\ifPDFTeX\else
  % xetex/luatex font selection
\fi
% Use upquote if available, for straight quotes in verbatim environments
\IfFileExists{upquote.sty}{\usepackage{upquote}}{}
\IfFileExists{microtype.sty}{% use microtype if available
  \usepackage[]{microtype}
  \UseMicrotypeSet[protrusion]{basicmath} % disable protrusion for tt fonts
}{}
\makeatletter
\@ifundefined{KOMAClassName}{% if non-KOMA class
  \IfFileExists{parskip.sty}{%
    \usepackage{parskip}
  }{% else
    \setlength{\parindent}{0pt}
    \setlength{\parskip}{6pt plus 2pt minus 1pt}}
}{% if KOMA class
  \KOMAoptions{parskip=half}}
\makeatother
\usepackage{xcolor}
\setlength{\emergencystretch}{3em} % prevent overfull lines
\providecommand{\tightlist}{%
  \setlength{\itemsep}{0pt}\setlength{\parskip}{0pt}}
\setcounter{secnumdepth}{-\maxdimen} % remove section numbering
\usepackage{bookmark}
\IfFileExists{xurl.sty}{\usepackage{xurl}}{} % add URL line breaks if available
\urlstyle{same}
\hypersetup{
  hidelinks,
  pdfcreator={LaTeX via pandoc}}
\begin{document}
\maketitle


\section{緒言}\label{ux7dd2ux8a00}

本稿において、「序論」が「問題設定とそれに対する答え方の構え」を示すものである一方で、「緒言」は「本稿が書かれた時代の知の配置の中で、本稿がどのような位置づけの考察なのか」を示すものだ。本稿は、誰に何をもたらすのか。そして、それを可能にするために何をするのか。

本稿がまず念頭においている読者は、私たちを駆り立てるべくますます騒がしさを増してきつつある世の喧騒に対してうんざりしつつも、実際のところどのように生きるべきかについて思いめぐらせている人々だ。そうした人々にとって、本稿は静寂を取り戻す助けとなることだろう。本稿は、人々が混乱の中に沈むことを防ぐ小舟のような存在となるに違いない。その小舟の上からは、世の喧騒が如何に簡単な無理解に立脚したものであるかがすぐに分かるようになる。そして読者は、本稿を読むことで、本質的に未解決な問題については地道な探究を重ねるしかなく、そこに魔法のような解決があることは稀なのだと、腹の底から納得できるようになるだろう。

このようにして得られる静寂において、読者の前にはこの宇宙の中で、私たちが如何に小さな存在であるかが分かるようになる。そこでは宇宙が、古の人間が思い描いたであろうような意味に満ちたものではなく、茫漠とした無根拠と無意味として現れてくる。

しかし、それにもかかわらず、読者はそこでただ単にすべてから切り離された根無し草になるわけではない。読者はそこで、自身の身体に出会う。その身体とは、自身がそこでそのように感じ、そう欲する以外の仕方では欲することができないような「固着した飢え」である。

飢えは救いではないが、飢えを無視することは苦しみを生む。固着した飢えという立脚点を得て、その渇きを癒すために、幸運ならば、読者は本稿という知識の小舟から無知が渦巻く宇宙という大海へと最後は飛び込むだろう。本稿が提供する知識の体系は、その際に自身の身を混沌から守る一助にもなるだろう。本稿は、そのようなものだ。

そのような議論を、本稿はどのようにして可能にするのか。それは、本稿は複数の専門分野で発見された知識を相互に束縛させあい、接続させあうことによってだ。そうすることで、本稿は複数の専門分野を一つに繋ぎ合わせ、統合的な知の道筋を浮かび上がらせる。その際に、本稿は各専門分野の基本的な知識しか用いない。それは、本稿が示す道筋が、実は現代人の眼前に当然のごとく在り続けていたにもかかわらず、ほとんど誰にも顧みられなかったものだからだ。この意味で、本稿は特定の専門分野に直接寄与するものではないし、文学的なテクストによって人々の情動を揺さぶるものでもない。それらは本稿の目的ではないからだ。同様に、特定の専門分野への寄与や文学的なテクストを期待する読者に、本稿が与えられるものはない。それにもかかわらず、本稿が示す道筋は、現代人の知識を整理することによって、様々な専門分野や文学に従事する人々を含む幅広い層に寄与するだろう。

\newpage

\input{00-序論:本稿の意義}
\newpage

\section{第一部:生命一般を縛る限界から脳を支配する弁証法へ}

\input{01-第一章:生命と環境}
\newpage
\includepdf[pages=-, scale=0.95, pagecommand={\thispagestyle{plain}}]{../figures/図1.pdf}

\input{02-第二章:同一性と変化}
\newpage
\includepdf[pages=-, scale=0.95, pagecommand={\thispagestyle{plain}}]{../figures/図2.pdf}

\input{03-第三章:脳と自由エネルギー原理}
\newpage
\includepdf[pages=-, scale=0.95, pagecommand={\thispagestyle{plain}}]{../figures/図3.pdf}

\section{第二部:体験とシニフィアンとの弁証法が形成する準安定状態としての四つのディスクール}

\subsection{第四章:体験とシニフィアン}\label{ux7b2cux56dbux7ae0ux4f53ux9a13ux3068ux30b7ux30cbux30d5ux30a3ux30a2ux30f3}

\newpage
\includepdf[pages=-, scale=0.95, pagecommand={\thispagestyle{plain}}]{../figures/図4.pdf}

\subsection{第五章:対象aと欲動の主体}\label{ux7b2cux4e94ux7ae0ux5bfeux8c61aux3068ux6b32ux52d5ux306eux4e3bux4f53}

\newpage
\includepdf[pages=-, scale=0.95, pagecommand={\thispagestyle{plain}}]{../figures/図5.pdf}

\subsection{人間の行為}\label{ux4ebaux9593ux306eux884cux70ba}

知と無知の弁証法以外の知性は存在しない
変分ベイズ推論以外の思考は存在しない 生は無知を孕んでいる

\newpage
\includepdf[pages=-, scale=0.90, pagecommand={\thispagestyle{plain}}]{../figures/結節点1.pdf}

\subsection{第六章:神経症と精神病における葛藤の解消}\label{ux7b2cux516dux7ae0ux795eux7d4cux75c7ux3068ux7cbeux795eux75c5ux306bux304aux3051ux308bux845bux85e4ux306eux89e3ux6d88}

\newpage
\includepdf[pages=-, scale=0.90, pagecommand={\thispagestyle{plain}}]{../figures/図6.pdf}

\section{第七章:エディプス・コンプレックスの諸段階}\label{ux7b2cux4e03ux7ae0ux30a8ux30c7ux30a3ux30d7ux30b9ux30b3ux30f3ux30d7ux30ecux30c3ux30afux30b9ux306eux8af8ux6bb5ux968e}

あ

\subsection{この章のまとめ}\label{ux3053ux306eux7ae0ux306eux307eux3068ux3081}

あ

\newpage
\includepdf[pages=-, scale=0.90, pagecommand={\thispagestyle{plain}}]{../figures/図7.pdf}

\input{08-第八章:神経症的主体における四つのディスクールと自然についての理解}
\newpage
\includepdf[pages=-, scale=0.90, pagecommand={\thispagestyle{plain}}]{../figures/図8.pdf}

\section{第三部:四つのディスクールが形成する人間社会のダイナミズム}

\input{09-第九章:自然への抵抗としてのエンジニアリングと芸術}
\newpage
\includepdf[pages=-, scale=0.90, pagecommand={\thispagestyle{plain}}]{../figures/図9.pdf}

\section{第十章:エンジニアリングが持つダイナミズムからの疎外の結果1(抑圧と反抗)}\label{ux7b2cux5341ux7ae0ux30a8ux30f3ux30b8ux30cbux30a2ux30eaux30f3ux30b0ux304cux6301ux3064ux30c0ux30a4ux30caux30dfux30baux30e0ux304bux3089ux306eux758eux5916ux306eux7d50ux679cuxff11ux6291ux5727ux3068ux53cdux6297}

\subsection{複雑化・硬直・プレモダン}\label{ux8907ux96d1ux5316ux786cux76f4ux30d7ux30ecux30e2ux30c0ux30f3}

あ

\subsection{主人のディスクール}\label{ux4e3bux4ebaux306eux30c7ux30a3ux30b9ux30afux30fcux30eb}

あ

\[
\uparrow\frac{\mathrm{S_1}}{\mathrm{\cancel{S}}}\genfrac{}{}{0pt}{}{\longrightarrow}{//}\frac{\mathrm{S_2}}{a}\downarrow
\]

\subsection{大学のディスクール}\label{ux5927ux5b66ux306eux30c7ux30a3ux30b9ux30afux30fcux30eb}

あ

\[
\uparrow\frac{\mathrm{S_2}}{\mathrm{S_1}}\genfrac{}{}{0pt}{}{\longrightarrow}{//}\frac{a}{\mathrm{\cancel{S}}}\downarrow
\]

\subsection{ヒステリー者のディスクール}\label{ux30d2ux30b9ux30c6ux30eaux30fcux8005ux306eux30c7ux30a3ux30b9ux30afux30fcux30eb}

あ

\[
\uparrow\frac{\mathrm{\cancel{S}}}{a}\genfrac{}{}{0pt}{}{\longrightarrow}{//}\frac{\mathrm{S_1}}{\mathrm{S_2}}\downarrow
\]

\subsection{分析家のディスクール}\label{ux5206ux6790ux5bb6ux306eux30c7ux30a3ux30b9ux30afux30fcux30eb}

あ

\[
\uparrow\frac{a}{\mathrm{S_2}}\genfrac{}{}{0pt}{}{\longrightarrow}{//}\frac{\mathrm{\cancel{S}}}{\mathrm{S_1}}\downarrow
\]

\subsection{硬直したプレモダン的労働における疎外}\label{ux786cux76f4ux3057ux305fux30d7ux30ecux30e2ux30c0ux30f3ux7684ux52b4ux50cdux306bux304aux3051ux308bux758eux5916}

フォーディズム・設計主義
人間は機械の一部として量的に扱われる(リソース(=資材)としての労働力)
精神分析は帝国主義とフォーディズムの時代の産物かも

\newpage
\includepdf[pages=-, scale=0.90, pagecommand={\thispagestyle{plain}}]{../figures/図10.pdf}

\subsection{第十一章:エンジニアリングが持つダイナミズムからの疎外の結果2(不安と暴力)}\label{ux7b2cux5341ux4e00ux7ae0ux30a8ux30f3ux30b8ux30cbux30a2ux30eaux30f3ux30b0ux304cux6301ux3064ux30c0ux30a4ux30caux30dfux30baux30e0ux304bux3089ux306eux758eux5916ux306eux7d50ux679cuxff12ux4e0dux5b89ux3068ux66b4ux529b}

\subsubsection{〈父の名〉の衰退}\label{ux7236ux306eux540dux306eux8870ux9000}

あ

\subsubsection{貨幣による秩序}\label{ux8ca8ux5e63ux306bux3088ux308bux79e9ux5e8f}

資本主義に本質的な問題と、プレモダンな専制に本質的な問題とを混同してはならない。資本主義の本質は、その運動を支える「素材」としての諸物をよりよく理解し、より効率よく利用することにある。それは労働力としての人間の扱いについても然りだ。資本主義の激化に伴い発生する労働問題として、労働者が

\subsubsection{不安・排除・レイシズム}\label{ux4e0dux5b89ux6392ux9664ux30ecux30a4ux30b7ux30baux30e0}

あ

\subsubsection{資本主義のディスクール}\label{ux8cc7ux672cux4e3bux7fa9ux306eux30c7ux30a3ux30b9ux30afux30fcux30eb}

双数性

\subsubsection{柔軟性のあるモダンな労働における疎外}\label{ux67d4ux8edfux6027ux306eux3042ux308bux30e2ux30c0ux30f3ux306aux52b4ux50cdux306bux304aux3051ux308bux758eux5916}

モダンな労働における疎外は、欲動の外から与えられた欲望にすり替えられることで起こる
プレモダンな労働と同じく、モダンな労働でも人は生産・消費の両面で量的なリソース(資材)として扱われる

コモンが搾取や疎外を避けるわけではない。株主からかかる利潤第一主義への圧力に従属しないことがコモンズの利点だと斉藤は主張するが、それは株式会社でも可能だ。アマゾンを見ろ。株主を説得できるだけのビジョンがない知的怠惰がダメなのであって、株式会社がダメなわけではない。コモンの中でも疎外と搾取は発生しうる。改善のプロセスから逃げてはならない。

資本主義をマルクス・ガブリエルが言う「倫理資本主義」に近いものに変えると考えられる。

\subsubsection{大義の条件}\label{ux5927ux7fa9ux306eux6761ux4ef6}

何かに対する敵対は、原理としての大義にはならない
大義は問いであり、顕現しない(顕現した原理は専制である)

\newpage
\includepdf[pages=-, scale=0.90, pagecommand={\thispagestyle{plain}}]{../figures/図11.pdf}

\section{結論:人間の限界とその先}

\input{12-結論1:人間の不満と満足が現れるダイナミズムのモデル化}
\newpage

\input{13-結論2:新しい物語の推奨ボーダーライン}
\newpage

\input{14-結論3:満足を可能にする生態系}
\newpage

\section{参考文献}\label{ux53c2ux8003ux6587ux732e}

\subsection{序論}\label{ux5e8fux8ad6}

\begin{itemize}
\tightlist
\item
  フリードリヒ・ニーチェ,三島憲一(訳),1974=1984,『ニーチェ全集 第九巻(第Ⅱ期) 遺された断想(一八八五年秋--八七年秋)』白水社.
\item
  --,木場深定(訳),1887=1940,『道徳の系譜』岩波書店.
\item
  ジャン=フランソワ・リオタール,小林康夫(訳),1979=1989,『ポストモダンの条件』水声社.
\item
  ハンス・ロスリング/オーラ・ロスリング/アンナ・ロスリング・ロンランド,上杉周作/関美和(訳),2018=2019,『FACTFULNESS(ファクトフルネス)
  10の思い込みを乗り越え、データを基に世界を正しく見る習慣』日経BP.
\item
  浅田彰,1983,『構造と力--記号論を超えて』勁草書房.
\item
  市川遊佐,2021,「ラカニアン・アジャイル―『四つのディスクール』から考える中間集団論/組織論としての『スクラム』」(『アレ』Vol.9収録,アレ★Club:240-313).
\item
  落合陽一,2015,『魔法の世紀』PLANETS.
\item
  小泉悠/桒原響子/小宮山功一郎,2023,『偽情報戦争 あなたの頭の中で起こる戦い』ウェッジ.
\item
  津田正太郎,2024,「メディア社会とは何か 1. 国民国家とマスメディア―トイ人」(2024年8月21日取得,\url{https://www.toibito.com/toibito/articles/\%E5\%9B\%BD\%E6\%B0\%91\%E5\%9B\%BD\%E5\%AE\%B6\%E3\%81\%A8\%E3\%83\%9E\%E3\%82\%B9\%E3\%83\%A1\%E3\%83\%87\%E3\%82\%A3\%E3\%82\%A2}).
\item
  松本卓也,2018,『享楽社会論』人文書院.
\end{itemize}

\subsection{第一章}\label{ux7b2cux4e00ux7ae0}

\begin{itemize}
\tightlist
\item
\end{itemize}

\subsection{第二章}\label{ux7b2cux4e8cux7ae0}

\begin{itemize}
\tightlist
\item
  小林武彦,2021,『生物はなぜ死ぬのか』講談社.
\end{itemize}

\subsection{第三章}\label{ux7b2cux4e09ux7ae0}

\begin{itemize}
\tightlist
\item
  乾敏郎・坂口豊,2020,『脳の大統一理論―自由エネルギー原理とはなにか』岩波書店.
\item
  気象庁,2023,「気象庁|過去の気象データ検索」(2024年9月12日取得,\url{https://www.data.jma.go.jp/obd/stats/etrn/view/daily_h1.php?prec_no=44&block_no=00&year=2023&month=08&day=&view=p3}).
\item
  小坂修平,2004,『図解雑学現代思想』ナツメ社.
\end{itemize}

\subsection{第四章}\label{ux7b2cux56dbux7ae0}

\begin{itemize}
\tightlist
\item
\end{itemize}

\subsection{第五章}\label{ux7b2cux4e94ux7ae0}

\begin{itemize}
\tightlist
\item
\end{itemize}

\subsection{第六章}\label{ux7b2cux516dux7ae0}

\begin{itemize}
\tightlist
\item
\end{itemize}

\subsection{第七章}\label{ux7b2cux4e03ux7ae0}

\begin{itemize}
\tightlist
\item
\end{itemize}

\subsection{第八章}\label{ux7b2cux516bux7ae0}

\begin{itemize}
\tightlist
\item
  トーマス・クーン,中山茂(訳),1970=1971,『科学革命の構造』みすず書房.
\item
  フレッド・ホイル,中島龍三(訳),1973=1974,『コペルニクス:
  その生涯と業績』法政大学出版局.
\item
  向井雅明,2016,『ラカン入門』筑摩書房.
\end{itemize}

\subsection{第九章}\label{ux7b2cux4e5dux7ae0}

\begin{itemize}
\tightlist
\item
  フリードリッヒ・ニーチェ,信太正三(訳),1921=1993,『ニーチェ全集
  (8) 悦ばしき知識』筑摩書房.
\item
  カミール・パーリア,鈴木晶/入江良平/浜名恵美/富山英俊(訳),1990=1998,『性のペルソナ 上:古代エジプトから19世紀末までの芸術とデカダンス』河出書房新社.
\item
  二コラ・フルリー,松本卓也(訳),2010=2020,『現実界に向かって―ジャック=アラン・未レール入門』人文書院.
\item
  赤坂和哉,2011,『ラカン派精神分析の治療論―理論と実践の交点』誠信書房.
\item
  市谷聡啓,2018,『カイゼンジャーニー―
  たった1人からはじめて、「越境」するチームをつくるまで』翔泳社.
\item
  今村仁司,2024,『仕事』講談社.
\item
  小田中育生,2023,「Be
  Agile~-アジャイルマインドセットでいきいきと働く-」(2024年10月23日取得,\url{https://speakerdeck.com/ikuodanaka/be-agile-and-work-with-a-iki-iki-spirit})
\item
  斎藤幸平,2023,『マルクス解体―プロメテウスの夢とその先』講談社.
\item
  作田啓一,2003,『生の欲動―神経症から倒錯へ』みすず書房.
\end{itemize}

\subsection{第十章}\label{ux7b2cux5341ux7ae0}

\begin{itemize}
\tightlist
\item
\end{itemize}

\subsection{第十一章}\label{ux7b2cux5341ux4e00ux7ae0}

\begin{itemize}
\tightlist
\item
\end{itemize}

\subsection{結論1}\label{ux7d50ux8ad6uxff11}

\begin{itemize}
\tightlist
\item
\end{itemize}

\subsection{結論2}\label{ux7d50ux8ad6uxff12}

\begin{itemize}
\tightlist
\item
\end{itemize}

\subsection{結論3}\label{ux7d50ux8ad6uxff13}

\begin{itemize}
\tightlist
\item
\end{itemize}

\newpage

謝辞

\begin{itemize}
\tightlist
\item
  2024-07-27に記載(2024-07-28最終更新):

  \begin{itemize}
  \tightlist
  \item
    この研究は、『アレ
    vol.9』所収の市川遊佐「ラカニアン・アジャイル―『四つのディスクール』から考える中間集団論/組織論としての『スクラム』」を主要先行研究として、サークル〈アレ★Club〉の活動の中で作られた。
  \end{itemize}
\end{itemize}

\newpage

\input{end}
