\section{序論:本稿の意義}\label{ux5e8fux8ad6ux672cux7a3fux306eux610fux7fa9}

\subsection{問題意識あるいは問いの設定}\label{ux554fux984cux610fux8b58ux3042ux308bux3044ux306fux554fux3044ux306eux8a2dux5b9a}

世界のすべてについて知ることができる立場が、広く人々に与えられたことはない。それゆえに、人々はこれまで常に無知を抱え、その無知ゆえに世界に翻弄され、困惑し、そして不安を抱えてきた。不条理な現実、耐えがたい受苦、取り返しのつかない過去の過ちへの後悔\ldots\ldots これらを浴びせられたときに、それが「無駄(umsonst)」であるなどとは、到底受け入れがたいものだ{[}\^{}1{]}。人が不可解な自身の生について、「何も欲さない」でいることは至難の業である{[}\^{}2{]}。人は、飢えているのだ。

そこで、人々は「無を欲する」ことにした{[}\^{}3{]}。自身の知を超えた彼方に崇高なるものの存在を措定し、そこから惨めな己の生に救いの光を当てることにしたのだ。自然界の、血族の、国家の、人類の神性が信じられ、また超越的な神が信じられるようになった。それらが真に実在するかどうかなどは、人間には知るべくもない。しかし、信じるだけでも、そこに「意味」が与えられ、それによって人の心は救われるのだ。

だが、かつて信じられた物語の多くは、科学的営為の積み重ねによる人類の知の増加によって、寓話として捉えるのでもなければ、あまりにも迷信的なものに見えるようになってしまった。それでも、まだかつて人類が希望を失わない時代があった。

それは、「希望に満ちた未来」という物語が代わりに信じられるようになったからだ。新しく人々の前面に現れた資本主義という「欲求の体系{[}\^{}4{]}」においては、貨幣などの通用力を持つ媒介を用いることで、人々の不満と生産力とがボトムアップで調整されながら、ますます人類の力が増していった。そこで必要なのは、暴力的な破壊や盗みを合理的な行動選択として成立させないための安全保障体制だけ、のはずだった。しかし、この動的な物語もまた、瓦解を始めることになった(=「大きな物語の終焉」{[}\^{}5{]})。

まず第一に、20世紀後半になると、無尽蔵にも思われた地球の資源および環境が、人類の活動によって大いに攪乱されうるものであり、ナイーブにそれらを収奪し続けていては文明は維持できなくなるかもしれないのだということが分かってきた。人類の果てしない進歩を漫然と信じられる時代は終わってしまった。「存続するための十分な努力ができなければ、廃れ、やがては滅びるかもしれない」という未来が見えてきたのである。

続いて第二に、21世紀に入ると、科学技術は高度な教育なしではとても信じられないような水準に達するようになった。世界と社会のフロンティアは魔術的{[}\^{}6{]}になり、そうしたフロンティアを理解するだけのリテラシーがない者は、わけもわからず時代の流れに振り回されるようになっていった。それは新たな「疎外」状況であり、その疎外の中で「社会から見捨てられた」と反感を溜め込む者たちも多く現れるようになった。

そして第三に、2010年代以降、玉石混交の情報がインターネットを通じて社会に氾濫するようになった。前世紀に比すれば世界は概ね豊かになった{[}\^{}7{]}ものの、世界各地の酸鼻極まる悲惨が事実として存在しており、人々はそうしさ悲惨についての情報を毎日のように浴びせられている。悲惨がなくならない以上、再びそこに「なぜ?」という問いが持ち上がる。そして、その問いに対しては「その悲惨の背後に邪悪な陰謀を張り巡らせる権力者がおり、説得など通じないならず者がおり、私服を肥やすことにしか関心のない悪徳商人がおり\ldots\ldots」といった物語が与えられることになるわけだ。もちろん、そうした物語りは気に食わない他者の信用を毀損するための偽情報にすぎない場合が少なくない(なお、本当にそうした陰謀や悪が実在する場合もある)のだが、そうした物語を魅力的に感じる人は後を絶たず、今日ではそうした偽情報の氾濫にまつわる混乱が社会問題となっている。しかも、そこでは混乱した状況に対して「もう何も分からん、とりあえず社会が変化するのは何やら恐ろしいからやめてくれ!」と判断させることを目標にしたメタな連中すらもいる始末だ(=「ディスインフレーション」{[}\^{}8{]})。

要するに、過去より存在する物語を信仰し続けるのには筋金入りの信心が必要であり、ナイーブに未来を信じるのには事態は差し迫りすぎており、真面目に社会を前進させるには人類の平均的知力は既に「落伍しないのがやっと」のレベルになっていて、かつ現に取りこぼされた人々が「救い」として縋りつくのは(世界に溢れる悲惨を直に引き受けるのでなければ)往々にして陰謀論的な「真実」であったりする\ldots\ldots という有様なのだ。

結局、人類は自身の不安と悲しみを慰撫するための新たな物語を渇望し続けている。人間の歴史とは、無を仰ぐ「ニヒリズム」の歴史であった。本論にて詳述するが、人類がさらに進化したとしても、おそらくこれからも―少なくとも当分は―そうだろう。

\begin{itemize}
\tightlist
\item
  {[}\^{}1{]}
  um-sonst(無を-狙う)ものとしての無駄については、ニーチェのいわゆる「レンツァーハイデ草稿」(ニーチェ(1974=1984:275-284)所収の「1886年夏―1887年秋 5[71]」に相当)を参照。
\item
  {[}\^{}2{]} ニーチェ(1887=1940:271)を参照。
\item
  {[}\^{}3{]} 同上
\item
  {[}\^{}4{]} ==ヘーゲル『法の哲学』==を参照。
\item
  {[}\^{}5{]}
  言葉としての「大きな物語の終焉」はリオタール(1979=1989)で提唱された。その本稿における意味合いとしては松本(2018:15-16)を参照。
\item
  {[}\^{}6{]} 落合(2015)を参照。
\item
  {[}\^{}7{]} ロスリング/ロスリング/ロンランド(2018=2019)を参照。
\item
  {[}\^{}8{]}
  陰謀論とディスインフレーションについては、小泉/桒原/小宮山(2023)における記述と、津田(2024)における「これは権威主義国家でよく使われる手法なんですけど、ウソか本当か分らない情報を、ソーシャルメディアなどを使ってとにかく大量に流すんですね。すると人びとはどの情報を信じればよいかわからず、変化よりも現状維持を選ぶようになると。つまり、権力者が自分たちの体制を維持しようと思ったら、まじめに説得するより、訳のわからない情報を流して思考停止にさせた方が、人びとを楽にコントロールできるというわけです」という記述を参照。
\end{itemize}

\subsection{本稿の目的あるいは目論見}\label{ux672cux7a3fux306eux76eeux7684ux3042ux308bux3044ux306fux76eeux8ad6ux898b}

本稿には、三つの目標がある。本稿はその三つの目標を達成することによって、先述したような状況の中で人々がより広く「満足」できる社会を構想することを目的としている。

第一の目標は、人類において不満と満足が現れるダイナミズムをモデル化して示すことだ。人はどのようなときに不満を覚え、どのようにして満足を得るのか。その仕組みのどこが必然的なものであり、どこがそうではないのか。それを科学的な知見に基づいて明らかにするのがこの第一の目標だ。そこでは、物語を信じることには不満を解消し満足を獲得する上での様々な有効性があり、それゆえに「信じるに値する物語を獲得すること」が効果的なのだということも説かれる。この目標が達成されることで、以下の第二の目標を目指すことができるようになる。

第二の目標は、同じく科学的な知見に基づいて「新しい物語が満たすべき基準としての『推奨ボーダーライン』」を引くことにある。そのボーダーラインには、二つの要件がある。一つ目の要件は、その物語が科学的な知見との整合性を持つことであり、二つ目の要件は、その物語が人間を「双数的」な関係ではなく「象徴的」あるいは「現実的」関係におくことである(これらの用語の意味は本論で説明する)。ここで、「推奨ボーダーライン」という歯切れの悪い表現をしているのには理由がある。「推奨」という語と「ボーダーライン」という語のそれぞれについて、その語を採用した二つの理由を以下で説明する。

まず第一に、「ボーダーライン」という語を選んだ意味についてだが、これは(少なくとも現時点における人類の)科学的知見は世界の全貌を解き明かすような水準には達していないため、その総体を集成してもそれで世界の全貌について説明してくれる物語が即座に生み出されるわけではないという事情から説明される。すなわち、科学的知見と整合的な物語を志向するにしても、本稿が提示しようとするボーダーラインはあくまでそうした物語の「必要条件」に留まることになるのだ。この必要条件を満たす物語を立ち上げるにしても、本稿が示すボーダーラインの範囲を超えた「肉付け」や「選択」の部分には相当な程度の自由度が残されており、その自由度の扱い方については読者に委ねられることになる。

そして第二に、「推奨」という語を選んだ意味についてだが、これは「そうしたボーダーラインを満たさないことが、直ちにその物語が間違っているということを意味するわけではない」という事情から説明される。科学的知見というのは科学者が発案したストーリーにすぎず、それが永遠普遍的に正しいものであるとは限らない。だから、本稿が提示するボーダーラインを満たさない物語を真理として信仰することは完全に可能なのだ。最初から科学的知見との整合性を度外視した物語であっても、それを真理として信仰するという行為に関しては、何の問題もなく成立する。本稿はこの点について、あくまで「科学的知見との整合性に難のある物語に基づいて生きていくと、そのために実生活上で様々な不都合や不満足が生じやすくなる結果、そこから生を肯定することがより困難な状況に追い込まれる可能性が考えられる」という観点から、それを積極的には推奨しないというだけのことだ。

第三の目標は、以上二つの目標を達成した結果を踏まえて、「社会において複数の世界観が作り出す生態系を如何に構成していくべきか」という問いを立て、これに対して一応の回答を示すことである。そこでは、社会の成員それぞれが選好性を示す生き方のタイプが異なることを踏まえて、「複数の世界観が、それぞれを選好する成員を包括した上で、相互に交流しつつ棲み分ける」ような一つの生態系が構想される。これによって、単一の世界観が多様な成員に押し付けられる事態を避けながら、それぞれの世界観の間にある対立を緩和することが可能になる。そこでこそ、人々は自身が信じたいものを信じながら満足をより安心して追求できるようになるはずだ。

\subsection{本稿の構成および序論}\label{ux672cux7a3fux306eux69cbux6210ux304aux3088ux3073ux5e8fux8ad6}

本稿はこの三つの目標を達成するために、生が持つ可能性を様々な角度から検討することした。そうした検討を行うために、具体的な方法論としては本稿は「専門細分化した諸学のそれぞれから生を検討した上で、それらの結果を統合的に要約する」ことにした。

そこで本稿は、この具体的な方法論に従い、生を検討する際に参照する学問領域ごとに一枚の図と章を切り出すことにした。そして、そのそれぞれの図&章の間に論理的な接続関係を引くことによって、諸学の統合を図ることにした。それぞれの図&章は、第一部から第三部までの三つの部分に分けられた上で、下記のように各学問領域と対応している。

なお、第一部(第一章~第三章)では生命一般について人類という枠を超えて成立する知見を見ていき、生命の限界とその仕組みを理解した上で脳の振る舞いの中に「差異に開かれた弁証法」を見出す。続く第二部(第四章~第八章)では、ラカン派精神分析の知見を見ていき、体験とシニフィアンとの間で形成される安定的な均衡状態としての四つのディスクールに論を進める。そして第三部(第九章~第十一章)では、四つのディスクールを軸に人間社会のダイナミズムを理解していく。

まず、図1&第一章は「物理学・化学・生態学」に対応している。そこでは、この宇宙を支配する原則としての物理的なプロセスが提示され、その部分集合としての化学的なプロセスが提示される。そして、その化学的なプロセスに従って長大な時間的空間的スケールで描かれる生物圏(バイオスフィア)の中で、それぞれのニッチェの中に住まい、そしてそのニッチェと共に変化しておくものとして生物種を生態学的に描く。

図2&第二章は「分子生物学」に対応している。分子生物学の知見を踏まえることで、図1&第一章で示した大局的=積分的な変化がどのような局所的=微分的な力学から生じるかを示す。そこでは、各個体の生存を保ちかつ種の進化を可能にするための仕組みとして、「遺伝情報の自己複製」と「遺伝子の発現制御」と「世代交代」の三つの機能が説明される。この章の記述を前提として、図3&第三章以降の記述が進められることとなる。

図3&第三章では、近年注目を集めている脳科学の学説である「自由エネルギー原理」に基づいて生物の思考と行為が従う原則を説明する。そこで示される「予想」と「予想誤差」をめぐる簡潔な数式は、生物の思考と行為が持つ弁証法的なあり方を表現している。

この弁証法的なあり方を蝶番として、図4&第四章と図5&第五章では「ラカン派精神分析」を用いた自然科学的な世界説明から意味と生の観点からの世界説明へと移る。この架橋は、予想と予想誤差との関係が「シニフィアンの体系」と「(シニフィアンの体系による象徴化を逃れた)残余」との関係に対応していると解釈することによりなされる。続く図6&第六章から図8&第八章までは、ラカン派精神分析の諸概念を本稿の趣旨に関わる最低限のレベルで説明してある。具体的には、ラカン派精神分析における「神経症」と「精神病」の違いと、神経症的な心的構造が確立されるまでの「エディプス・コンプレックスの成立過程」と、神経症者の取る思考と行為のタイプ分けとしての「四つのディスクール」が順に説明される。

図9&第九章では、ラカン派精神分析における四つのディスクールと広い意味でのエンジニアリングが接続される。この接続は「自然を制御する仕組みだけではなく文化や社会制度に至るまでの幅広い範囲の制作物が人間の認識に沿ったものであり、そして人間の認識は四つのディスクールの各局面が現れるのに伴って構築あるいは解体されるのだから、それらの制作物が生み出されたり廃棄されたりする過程もまた四つのディスクールによって記述される」という発想に基づいている{[}\^{}1{]}。

図10&第十章では「エンジニアリングの過程に伴い人間の認識と制作物が複雑に組み合わさっていくと、その前提的な役割を果たしている部分は容易には変えられなくなってしまう」ということを指摘した上で、その硬直が四つのディスクールの各局面にそれぞれ新たな効果を付与することを指摘する。この状態は、社会が「プレモダン」{[}\^{}2{]}の段階に至ったことを意味している。そこでは、国家や宗教の権威が強力な力を持ち、人々は官僚主義的なヒエラルキーの中で生きていくことになる。

図11&第十一章では、このプレモダンの段階から、貨幣などの媒介によって権威が宙吊りにされて、「モダン」{[}\^{}3{]}{]}な資本主義に移行した後の社会について記述する。そこでは、「労働者」あるいは「資本家」が生産活動を通じて貨幣や資本を増大させようとする側面と、「消費者」の持つ不満が商品の購入によって速やかに解消される側面とが描かれる。また、この二つの側面が社会を秩序付ける主な力として台頭する過程でプレモダンな権威が相対的に力を失うことによって、人々が異質な他者に対して耐える力が弱まり、「レイシズム」などの差別が勃興してくることを指摘する。

ここまでの議論を踏まえ、「諸学の綜合」の章となる第十二章から第十四章では本稿の三つの目標を達成していく。すなわち、第一に人間の不満と満足が現れるダイナミズムをモデル化し、第二に「新しい物語が満たすべき基準としての『推奨ボーダーライン』」を示し、第三に(その新しい物語を含めた)多様な物語を社会の中でそれぞれどのように位置付ければ良いかを検討する。こうすることで、本稿はその目的である「人々がより広く『満足』できる社会」を構想する。

\begin{itemize}
\tightlist
\item
  {[}\^{}1{]} 市川(2021)を参照。
\item
  {[}\^{}2{]} 浅田(1983)を参照。
\item
  {[}\^{}3{]} 同上。
\end{itemize}

\subsection{本稿の限界と仰ぎたい協力}\label{ux672cux7a3fux306eux9650ux754cux3068ux4ef0ux304eux305fux3044ux5354ux529b}

本稿は、諸学の綜合によって人間の生が持つ振れ幅を総括し、そこから人類の未来を浮かび上がらせるというアイデアの下で書かれた。この野心的なアイデアは、それを実現するというこの仕事に特有の難しさを持ち込むことになる。

その難しさは、当然のこととして幅広い学識を十分な深さでもって持つことの困難さに起因している。この困難さを可能な限り迂回するための戦術として、本稿は各学問分野のなるべく初歩的な部分に着目しようとした。しかし、本稿が論旨に軸を通す際にその強力な効果を当て込んで参照することとした自由エネルギー原理とラカン派精神分析は、その学問的評価も定まり切っていない研究分野である。

こうした事情(すなわち、第一に幅広く深い学識を持つことは本質的に困難であるという事情、第二に本稿の議論は学問的評価の定まり切っていない研究分野の知見に大きく依存しているという事情)ゆえに、本稿にはその内容面に対して多くの疑問や批判が投げかけられうると思っている。そうした点について、筆者は粘り強く本稿の内容を堅く擁護しきろうとは思っていない。それらは当然の疑問や批判であろうと思うからだ。

ただ、そうした疑問や批判を抱いた人がいるのであれば、是非私のアイデアを引き継いで、自分たちなりに発展していってもらいたい。そこでは、私がここで構築したような諸学間の接続はもはや行われなくなるのかもしれない。しかし、それでも筆者はこのアイデアの下に諸学の綜合から人類の未来が描かれていくのであれば、十分に満足だからだ。
