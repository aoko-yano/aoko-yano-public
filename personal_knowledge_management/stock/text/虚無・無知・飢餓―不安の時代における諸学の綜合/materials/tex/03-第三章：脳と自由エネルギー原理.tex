\subsection{第三章:脳と自由エネルギー原理}\label{ux7b2cux4e09ux7ae0ux8133ux3068ux81eaux7531ux30a8ux30cdux30ebux30aeux30fcux539fux7406}

\subsubsection{脳・学習・最適化}\label{ux8133ux5b66ux7fd2ux6700ux9069ux5316}

多細胞生物の一部には、身体の様々な箇所で起きた出来事について、その情報をまとめた上でそこから全身を適切に動かすための「神経」がある。多くの神経が集まる場所には特に多くの情報が集まり、より統合的な情報処理が行われる。どれくらいの神経がどのように集まるかは生物によりさまざまだ。例えば、昆虫ではその頭部から尾部にかけて「はしご形」に神経が走っており、そのはしご形に走っている神経の頭部以外にも数か所、神経細胞が多く集まっている場所がある。

人類を含む一部の生物に備わるそうした神経細胞の集まりは「脳」と呼ばれる。脳は、外界からの入力に応じて行動を制御する神経網である(*1)。例えば、人類の網膜は二次元上の視覚情報を受け取るが、その情報は脳で様々に分析され、目の前に何が存在しているかが判断される{[}\^{}1{]}。そして、その判断はさらに脳での情報処理を経て、その後に取られる行動に影響を及ぼす。なお、すべての行動が脳による判断に従っているわけではない。実際、内部で水を沸騰させているやかんなどに手で触れてしまった場合、熱いものに触れたという情報が手から脊髄へと伝わるが、脊髄から脳へと情報が伝わるまでもなく、脊髄から腕へと筋肉を収縮されて手を引っ込めるための電気刺激による指令が出る。このような反射に用いられる脳を介さない神経の経路を「反射弓」という。

多くの「脳」は、環境との相互作用を通じて後天的かつ柔軟に変化する(*2)。この後天的な変化は「学習」と呼ばれるが、脳を持つ生物は学習によって遺伝情報というハード面での変化を待つことなく身体の使い方というソフト面を変えることができる。それゆえに、学習は身体が許容する範囲で自らを取り巻く環境との関係性を素早く最適化することができるわけだ。

人間が構築してきた思考や社会もそうした学習による最適化の産物だと言える。そのため、学習と最適化がどのようにして行われるのかを理解することで、人間の思考や社会の成り立ちや構造、そして限界を理解することができる。それゆえに、この章では脳の振る舞いを理解することで、学習や最適化の仕組みを理解することを目指す。

\begin{itemize}
\tightlist
\item
  {[}\^{}1{]}
  乾敏郎・坂口豊『脳の大統一理論―自由エネルギー原理とはなにか』p.1-3,9-10を参照。
\end{itemize}

\subsubsection{感覚信号・予測信号・予測誤差}\label{ux611fux899aux4fe1ux53f7ux4e88ux6e2cux4fe1ux53f7ux4e88ux6e2cux8aa4ux5dee}

本稿では、2006年頃からイギリスの研究者であるカール・フリストン(1959~)によって提唱されるようになった「自由エネルギー原理」(*3)にしたがって、脳の振る舞いを説明していく{[}\^{}1{]}。自由エネルギー原理において、「脳はヘルムホルツの自由エネルギーを最小化するように推論を行う」とされているが{[}\^{}2{]}、これは具体的には脳が「予測誤差」を最小化するように動作することに等しい(*4){[}\^{}3{]}。

まず、脳の仕事とは、脳が受け取った感覚信号 \(s\) から外界の真の状態
\(u\) を推論することである。そのために、脳は「外界が状態 \(u_i\)
を取っていると想定」した上で、「その想定が正しかった場合に外界から受け取るはずの予測信号
\(g(u_i)\) 」と計算する能力を持つ。こうすると、脳は感覚信号 \(s\)
と予測信号 \(g(u_i)\) の差分 \(s-g(u_i)\) から、自身の想定 \(u_i\)
がどれほど間違っていたかを測定することができる。この感覚信号 \(s\)
と予測信号 \(g(u_i)\) の差分 \(s-g(u_i)\)
が予測誤差だ(*5)。脳の仕事は外界の真の状態 \(u\)
を求めることであるので、そのために脳は予測誤差 \(s-g(u_i)\)
をより少なくする新たな外界についての想定 \(u_{i+1}\)
を求めればよいことになる{[}\^{}4{]}。

なお、脳は予測誤差をどれだけ重大に捉えるか(=「精度」)を自主的に制御することができる。例えば、わずかな予測誤差であっても高い精度を求めて受け取るのであれば、それは重大な想定の欠陥を表すものであると認識されて想定を更新しなおすことになるのに対し、逆に大きな予測誤差であっても求める精度が低いのであれば、それは些細な誤差として想定の更新を引き起こさないことになる。この精度の制御が「注意」えある(*7,
8){[}\^{}5{]}。

\begin{itemize}
\tightlist
\item
  {[}\^{}1{]}
  乾敏郎・坂口豊『脳の大統一理論―自由エネルギー原理とはなにか』ivを参照。
\item
  {[}\^{}2{]} 同、p.4を参照。
\item
  {[}\^{}3{]} 同、p.16-24を参照。
\item
  {[}\^{}4{]} 同、p.18を参照。
\item
  {[}\^{}5{]} 同、p.25-33を参照。
\end{itemize}

\subsubsection{自由エネルギー原理・ダイバージェンス}\label{ux81eaux7531ux30a8ux30cdux30ebux30aeux30fcux539fux7406ux30c0ux30a4ux30d0ux30fcux30b8ux30a7ux30f3ux30b9}

予測誤差の最小化は、下記の式の「変分自由エネルギー」の最小化と等価である(*6){[}\^{}1{]}。変分自由エネルギーは、下記の式で表される:
\[
(変分自由エネルギー)=(ダイバージェンス)+(シャノンサプライズ)\tag{1}
\]
この式について説明するため、具体例として東京に2023年から住むようになり、あまりの暑さから東京での8月の最高気温を気にするようになった人のことを考える。その人は、毎日晴れたり曇ったりする天候と共に、最高気温の高い日や低い日を経験する。このとき、この人は\textbf{毎日の最高気温を感覚信号の列
\(s\) として受け取っている}ということになる。

さて、この感覚信号の列 \(s\)
は、地球と宇宙の極めて複雑な依存関係によって生成されるわけだが、そのプロセスの全貌は不可知であり、かつ普通に生きる人にとっては(少なくとも当面は)関心の対象にならない{[}\^{}2{]}。普通の人にとって関心があるのは、受け取った感覚信号の列
\(s\)
を生成した外界の状態を正しく反映した式だ{[}\^{}3{]}。つまり、その人は
\textbf{「東京の8月の最高気温は、 \(x_1\) \%で \(t_1\) 度、 \(x_2\) \%で
\(t_2\) 度\ldots\ldots{} \(x_n\) \%で \(t_n\)
度を取る」といったような、東京における8月の最高気温についての確率分布
\(p(u)\)} に関心があるわけだ。

その確率分布 \(p(u)\)
を求めるために、この人は最初に2023年8月の東京として表1のような感覚信号の列
\(s_0\)
を経験する。すると、2023年9月1日時点でのその人にとっての外界の状態を表す確率分布
\(p(u_0)\)
は(最初の年は極めて素朴に頻度からそのまま確率分布を考えたと前提すれば)表2のようになる。そして次の年、その人は2024年も8月を東京で過ごして表3のような感覚信号の列
\(s_1\) を経験する{[}\^{}4{]}。

\[
\begin{array}{cc}
~~~\textrm{最高気温(℃)}~~~ & ~~~\textrm{日数}~~~ \\
\hline
\textrm{31} & \textrm{1} \\
\textrm{32} & \textrm{6} \\
\textrm{33} & \textrm{2} \\
\textrm{34} & \textrm{13} \\
\textrm{35} & \textrm{8} \\
\textrm{36} & \textrm{1}
\end{array}
\tag{表1}
\]

\[
\begin{array}{cc}
~~~\textrm{最高気温(℃)}~~~ & ~~~\textrm{確率}~~~ \\
\hline
\textrm{31} & \textrm{3.26\%} \\
\textrm{32} & \textrm{19.4\%} \\
\textrm{33} & \textrm{6.45\%} \\
\textrm{34} & \textrm{41.9\%} \\
\textrm{35} & \textrm{25.8\%} \\
\textrm{36} & \textrm{3.26\%}
\end{array}
\tag{表2}
\]

\[
\begin{array}{cc}
~~~\textrm{最高気温(℃)}~~~ & ~~~\textrm{日数}~~~ \\
\hline
\textrm{27} & \textrm{1} \\
\textrm{29} & \textrm{1} \\
\textrm{30} & \textrm{1} \\
\textrm{31} & \textrm{4} \\
\textrm{32} & \textrm{1} \\
\textrm{33} & \textrm{7} \\
\textrm{34} & \textrm{9} \\
\textrm{35} & \textrm{7}
\end{array}
\tag{表3}
\]

いま、確率分布 \(p(u_0)\) を、感覚信号の列 \(s_1\)
を用いて新しい確率分布 \(p(u_1)\)
に更新することを考える。この新しい確率分布 \(p(u_1)\)
は、ベイズ推定という確率論の考え方を用いて事後確率分布 \(p(u|s)\)
として表現できる。したがって、その人の\textbf{脳が解くべきタスクは、外界
\(u\) がとる状態の確率分布 \(p(u)\)
を知るために、ある時点までに考えていた(事前)確率分布 \(p(u_i)\)
と、新たに得られた感覚信号の列 \(s_{i+1}\)
を用いて、更新された(事後)確率分布 \(p(u|s_{i+1})\)
」を知ること}となる{[}\^{}5{]}。

この \(p(u|s_{i+1})\)
を自由エネルギー原理の文脈では「真の事後確率分布」という。「真の」と銘打ってはいるものの、この確率分布は感覚信号の列
\(s\)
から計算可能な外界についての暫定的な理解を表すものであり、それゆえ暗黙的・潜在的に与えられているというのがポイントだ。例えば、いま考えている東京で8月の最高気温を気にする人の場合、(\(s_0\)
について27度から36度の範囲でラプラス・スムージングすることで \(p(u_0)\)
を補正した上で、尤度計算の際に正規分布の確率密度関数を用いたと前提すれば)2024年9月1日時点での真の事後確率分布
\(p(u|s_1)\) は表4のようになる。

\[
\begin{array}{cc}
~~~\textrm{最高気温(℃)}~~~ & ~~~\textrm{確率}~~~ \\
\hline
\textrm{27} & \textrm{0.03\%} \\
\textrm{28} & \textrm{0.11\%} \\
\textrm{29} & \textrm{0.39\%} \\
\textrm{30} & \textrm{1.00\%} \\
\textrm{31} & \textrm{3.96\%} \\
\textrm{32} & \textrm{21.0\%} \\
\textrm{33} & \textrm{10.4\%} \\
\textrm{34} & \textrm{42.6\%} \\
\textrm{35} & \textrm{18.5\%} \\
\textrm{36} & \textrm{2.11\%}
\end{array}
\tag{表4}
\]

このように、真の事後確率分布 \(p(u|s_1)\)
を知るためには、ここで扱っているような簡単な例ですら、かなりしっかりとした計算が必要となる。上段で「(理念的には)計算できる」や「暗黙的・潜在的に与えられている」などといった歯切れの悪い表現をしていたのは、その計算が多くの場合で事実上不可能だということを示すためだ。複雑な事例に囲まれた実生活においては、真の事後確率分布は計算できない場合が多いだろう。

そのため、日々の最高気温を気にするだけの普通の人が持つ外界についての認識は、もっと簡単なものに留まる。例えば、その認識は「東京では、8月の最高気温は概ね30度より高い。35度より暑くなることもままある」という程度のものだろう。重要なのは、それくらいの認識でも生存には十分に役立つということだ。この
\textbf{真の事後確率分布 \(p(u|s_{i+1})\)
の近似として使う「外界がとる状態についての確率論的な認識」を「認識確率分布
\(q(u_{i+1})\) 」と表現する} {[}\^{}6{]}。

ここで、認識確率分布 \(q(u_1\)) と真の事後確率分布 \(p(u|s_1)\)
との違いを確率論の言葉で表現したものが上述の式における「(カルバック・ライブラー)ダイバージェンス」だ(*9){[}\^{}7{]}。このダイバージェンスを計算する式は下記のようになっている:
\[
(ダイバージェンス)=(変分自由エネルギー)-(シャノンサプライズ)\tag{2}
\] それゆえに、この節の冒頭の式 (1)
で表したように変分自由エネルギーを記述できるわけだ{[}\^{}9{]}。なお、シャノンサプライズは、感覚信号の列
\(s_{i+1}\) が観測される確率 \(p(s_{i+1})\)
の対数の符号を反転させたものとして定義されるもので、その感覚信号の列が与えられることが稀であるほど大きな値をとる(*10)。

\begin{itemize}
\tightlist
\item
  {[}\^{}1{]}
  乾敏郎・坂口豊『脳の大統一理論―自由エネルギー原理とはなにか』p.21-22を参照。
\item
  {[}\^{}2{]} 同、付録p.11を参照。
\item
  {[}\^{}3{]} 同、付録p.8を参照。
\item
  {[}\^{}4{]}
  気象庁「気象庁|過去の気象データ検索」(\url{https://www.data.jma.go.jp/obd/stats/etrn/view/daily_h1.php?prec_no=44&block_no=00&year=2023&month=08&day=&view=p3})(2024年9月12日取得)を参照。
\item
  {[}\^{}5{]}
  乾敏郎・坂口豊『脳の大統一理論―自由エネルギー原理とはなにか』p.10-12およびp.19-20を参照。
\item
  {[}\^{}6{]} 同、p.21および付録p.7-8を参照。
\item
  {[}\^{}7{]} 同、付録p.7-9を参照。
\item
  {[}\^{}8{]} 同、p.114を参照。
\item
  {[}\^{}9{]} 同、付録p.9を参照。
\end{itemize}

\subsubsection{信念の更新・運動}\label{ux4fe1ux5ff5ux306eux66f4ux65b0ux904bux52d5}

式 (1)
から分かる通り、変分自由エネルギーを小さくするためには二つの方法がある。それは、ダイバージェンスを小さくする方法と、シャノンサプライズを小さくする方法だ。自由エネルギー原理においては、脳はこの二つを実現する機能を備えていると考えられている。では、この二つの方法について順にみていこう。

まず、ダイバージェンスを小さくする方法についてだ。この方法では、脳内で認識確率分布
\(q(u)\)
について「勾配降下法」と見なすことができる計算が行われ、ダイバージェンスの最小化が図られるとされている(*11){[}\^{}1{]}。勾配降下法とは、求める評価値(ここではダイバージェンス)がそれ以上下がらなくなるまで、少しずつ関数(ここでは認識確率分布
\(q(u)\) )を変化させていくことによって、求める関数(
ここでは真の事後確率分布 \(p(u|s)\)
)へと操作する関数を近づけていく計算手法である。このような方法を用いて、認識確率分布を更新する行為を、自由エネルギー原理では「無意識的推定」という(=「信念の更新」)(*12){[}\^{}2{]}。

もう一つが、シャノンサプライズを小さくする方法だ。その方法では、脳は認識確率分布
\(q(u_i)\) は変動させずに、シャノンサプライズ \(-\log{p(s_{i+1})}\)
が低くなるように、観測される確率が高い感覚信号の列 \(s_{i+1}\)
を狙って獲得しようとする(*16)。もし、その試みが成功すれば、それはその時点で採用している外界についての想定
\(u_i\)
が正しいことの証拠になる(*14){[}\^{}3{]}。つまり、脳は外界のサンプリングを通じて自分の推定が正しいことの証拠を能動的に集めているのであり、このような能動的なサンプリングを通じて脳は「自己証明」しているのだ(=「能動的推論」)(*15){[}\^{}4{]}。

このような「認識確率分布 \(q(u_i)\) を変えずに、得られる感覚信号の列
\(s_{i+1}\)
の側を変える」という行為を実現する仕組みが「(身体の)運動」だ。身体の運動を司る大脳の一部分を運動野と呼ぶが、運動野からは「運動するとこのような筋感覚信号が観測されるはずだ」という「筋感覚の予測信号」が出力される。すると、それが反射弓に伝わって筋収縮を起こし、身体が動く。そして、反射弓では「α運動ニューロン」という神経細胞が筋感覚の予測信号に合致するように筋肉が制御される(*13){[}\^{}5{]}。この一連の過程において認識確率分布
\(q(u_i)\)
が更新されないのは、運動野などの運動皮質には大脳皮質外からの入力信号が入ってくる第Ⅳ層という部分がほとんど見られないことから、感覚フィードバックの影響を受けないからだと考えられている{[}\^{}6{]}。

\begin{itemize}
\tightlist
\item
  {[}\^{}1{]}
  乾敏郎・坂口豊『脳の大統一理論―自由エネルギー原理とはなにか』付録p.14を参照。
\item
  {[}\^{}2{]} 同、p.2-3および付録p.9-10を参照。
\item
  {[}\^{}3{]} 同、付録p.11を参照。
\item
  {[}\^{}4{]} 同、p.24を参照。
\item
  {[}\^{}5{]} 同、p.35-45を参照。
\item
  {[}\^{}6{]} 同、p.46-47を参照。
\end{itemize}

\subsubsection{自由エネルギー原理から差異に開かれた弁証法へ}\label{ux81eaux7531ux30a8ux30cdux30ebux30aeux30fcux539fux7406ux304bux3089ux5deeux7570ux306bux958bux304bux308cux305fux5f01ux8a3cux6cd5ux3078}

さて、ここまで自由エネルギー原理に基づいて統一的に見てきた脳の振る舞いを、「脳と外界との間でなされる弁証法」として解釈しなおすことにしよう。なお、本稿では「ある存在(正)が、その存在を否定する物事(反)に出会うことで、変化する(合)」ようなプロセスとして弁証法を理解することにする。そうすることで、ここから先の議論を「見通しが良く、十分に自由だが適切に制限された視座」から統一的に把握できるようになるからだ。本稿がここまで「(いわゆる)理系」的な科学的知見をおさらいしてきたのは、そうした視座を得るためだった。

弁証法のプロセスとして脳の振る舞いを考えるとき、自由エネルギー原理では予測誤差の最小化が図られているのだから、そのプロセスは予測信号と感覚信号との間から起こると言える。つまり、脳が持つ外界についての信念をモデルとして、そのモデルに基づいた未来についての予期がなされ、それが予測信号として発せられる。そこに外界から入ってきた感覚信号が照らし合わされ、両者の間にある差異が予測誤差として発生する。そして、予測誤差を最小化する二つの方法に対応して信念の更新や身体の運動が起こるわけだが、この予測誤差の最小化が弁証法における正と反から合が作られる段階に相当していると考えられるだろう。その一方である信念の更新においては、予測誤差を最小化するような予測信号を生成できる信念が合として作り出される。また他方である身体の運動においては、正としての自身の信念に対して、それにそぐわない感覚信号としての反が得られないことを確認することにより、信念がより確証の深まったものとしての合になる。

このような構図から弁証法についての理解を更新することで、弁証法にまつわる様々な概念についても理解を改め、今後の章での見通しを良くすることができる。本稿が更新の対象として念頭に置いているのは、20世紀後半のいわゆるフランス現代思想における弁証法に対する理解だ。そこでは、弁証法を「世界のあらゆるところに差異を見出しつつも、それらを真なる世界の本質から生じた見せかけ上の対立だと見なす」発想に基づいたものだと見なした上で、その発想からは「あらゆるものの背後に理想が見出され、現に存在している多様な差異が抑圧されてしまう」とされてきた。そのため、フランス現代思想は「差異を肯定し、解放する」ことを提唱してきた。こうした弁証法に対する批判的な態度の背景には、西洋近代思想が「本当の〇〇(例えば、本当の『ドイツ人』)」といった理想を様々な対象の背後に見出し、その理想から逸脱するもの(例えば、「ユダヤ人」)に対して有形無形の暴力を振るってきたことの反省があった{[}\^{}1{]}。

しかし、このような弁証法理解は、少なくとも本稿が自由エネルギー原理に沿って考える弁証法に対しての理解とは合致しない。まず、本稿が考える弁証法においては、真なる世界の本質はたしかに予測誤差の最小化という形で目指されるものの、それが脳に与えられることはない。なぜならば、真なる世界の本質に向かうための手段として脳が使用できるのは変分ベイズ推論でしかないため、予測誤差を最小化できる信念は「場当たり的・発見的に作られる」しかないからだ。つまり、真なる世界の本質は、永遠不変の原理原則から導出されるようなものではなく、そこに向かって彷徨い歩くようにして近づいていくしかないものなのだ。

たしかに、普遍的な科学法則などについては人間の知は宇宙の様々な現象を説明できる域にまで達するかもしれない。だが、それでも人間の知は、具体的な個別の状況については、依然として無に等しいままに留まる他ない。何故ならば、そもそも量子力学の知見からして、厳密に対象の状態について知ることは不可能であり、よしんば対象の状態について厳密に知ることができたとしても、その対象がどのように変わっていくかの予測は確率的なものとならざるを得ないからだ。その上さらに、決定論的なシステムを対象に未来を予測しようとした場合ですら、三体問題のように任意の時間が経過した後の状態をピタリと計算することが不可能なシステムで世の中は満ちている。

だから、予測誤差は最小化を目指されるに過ぎず、それがゼロになることは決してありえない。ある生物が、具体的な物事についてその未来を正確に知ることはありえないわけだ。そのため、脳が思い描く理想というのはどこまでいっても曖昧で実現可能性に疑問符が付く無責任なものでしかない。ゆえに、生物が何らかの物事に対してそれを「善導」できるような立場に立つこともないのだ。

このように、脳は予測誤差がゼロにならない状況から逃れることはできない。そのような状況下で現前した予測誤差を受け入れないことは、単なる現状否認であり、弁証法の停止でしかない。脳は未来を予期し、能動的推論としての運動に際しては未来に対して目的を立てて能動的に行動することを可能にする。しかし、その結末は常に未知の差異に対して開かれているのだ。この「差異に開かれた、予測誤差の最小化を目指して弁証法的に変わりし続ける脳」という観点が、先に述べた、ここから先の議論を統一的に把握するのに役立つ「見通しが良く、十分に自由だが適切に制限された視座」だ。

{[}\^{}1{]} 小坂修平『図解雑学現代思想』を参照。
