\subsection{第九章:自然への抵抗としてのエンジニアリングと芸術}\label{ux7b2cux4e5dux7ae0ux81eaux7136ux3078ux306eux62b5ux6297ux3068ux3057ux3066ux306eux30a8ux30f3ux30b8ux30cbux30a2ux30eaux30f3ux30b0ux3068ux82b8ux8853}

\subsubsection{横溢する自然への抵抗としてのエンジニアリング}\label{ux6a2aux6ea2ux3059ux308bux81eaux7136ux3078ux306eux62b5ux6297ux3068ux3057ux3066ux306eux30a8ux30f3ux30b8ux30cbux30a2ux30eaux30f3ux30b0}

自由エネルギーの最小化としての葛藤の解消の帰結として、人間は自然を予測可能なものとするために、自然についての理解を体系化した上で、その知識に基づいて自然を制御する制度や仕組みを構築する。このような行為を本稿では「エンジニアリング」と呼ぼう(*1)。このとき、エンジニアリングが持つダイナミズムを四つのディスクールを用いて表現することができる(*2)。そのダイナミズムは、エンジニアリングによる捕獲を逃れる不確実性としての対象\(a\)により駆動される(*3)。

この章では、四つのディスクールに沿ってエンジニアリングが持つダイナミズムを説明していく。だがその前に、本稿が述べるエンジニアリングと労働との関係について軽く触れておくことにしよう。それは、エンジニアリングや労働という言葉について、第十章や第十一章で扱うような、その二次的なあり方のイメージが広く染みつきすぎているからだ。この章で扱うエンジニアリングや労働をそうしたイメージに沿って理解してはならないということを、最初に説明しておきたいのである。

\subsubsection{労働のオリジナル(=本源的)なあり方としてのエンジニアリングとカイゼン、そしてアジャイルな組織}\label{ux52b4ux50cdux306eux30aaux30eaux30b8ux30caux30ebux672cux6e90ux7684ux306aux3042ux308aux65b9ux3068ux3057ux3066ux306eux30a8ux30f3ux30b8ux30cbux30a2ux30eaux30f3ux30b0ux3068ux30abux30a4ux30bcux30f3ux305dux3057ux3066ux30a2ux30b8ux30e3ux30a4ux30ebux306aux7d44ux7e54}

まず、生命が世界を生きる過程において、およそ何の予想外の出来事にも出会わないような時間は存在しないという事実について改めて確認するところから始めよう。この事実は、具体的なシニフィアンがあれば必ずその傍らに取りこぼされた純粋差異としての〈物〉があるがゆえに、その顕現が予想外の出来事となるという事情に起因する(*6)。そこでは、私たちの無知の証である自然の横溢が、克服し征服すべきスリリングな知的興味の対象として表れている。

この事実は、エンジニアリングがどれほど発達しようとも、不確実性としての対象\(a\)が根絶されることはないということを意味している(*4)。自然の横溢を前にして、エンジニアリングを行う人間は自身の認識を弁証法的に更新し、その成果を制度や仕組みに反映させ続けることになる。この意味で、生に満ちている予想外の出来事のすべてが、エンジニアリングの入り口になっている。

この「生命にとって、予想外の出来事に出会わないような時間は存在しない」という命題は、生活の糧を得るための行動としての「労働」の場でも成り立つ。労働とは、まず第一に欲望の対象を生産する行為だといえる(ここでいう「生産」には、対象を市場の中に晒すことなどの社会的な行動も含まれる)。この欲望の対象は無から勝手に生じるのではなく、認識されるまでは謎でしかない自然の中から現れてくる。初めのうちは、欲望の対象は偶発的な事象に依存して発生するに留まることもありうる(例えば、「火の発生を落雷に頼る」や「獲物の兎が勝手に切り株に衝突して死ぬ」や「自分の所持品を高価な物と交換してくれる人に遭遇する」など)。この段階では、予測誤差は自然の恵みとしての偶発性自体と共にある。だが、そこから欲望の対象を人間が計画的に自然から取り出させるようになれば、その段階で自然から欲望の対象を取り出す生産方法についての認識が確立できたと言える(「火起こしの手法を確立する」や「獲物を捕まえる罠の作成法を確立する」や「所持品を交換したい人が集まる場を作る」など)。この段階では、予想外の出来事は認識に伴う取りこぼされた純粋差異としてその場に付随する(「たまたま、火起こしに使う棒が予想よりも少し湿っている」や「たまたま、その日は兎が通りがからなかった」や「たまたま、自分の所持品を欲望する人が現れなかった」など)。いずれにせよ、労働は予想外の出来事と共にあることになる。

そして、労働の場に予想外の出来事があるということは、労働を支える認識と行為にもそれが揺るがされる余地があるということだから、その克服は労働の「カイゼン」に繋がる。ここでいうカイゼンとは、私たちの認識と外界との間で発生する予測誤差の縮小に根本的な起源を持つものだ。その帰結は、単純な生産量の増加とは限らない。労働に伴う苦痛の削減や、生産する対象の変更、労働の目的の更新など、様々な帰結がありうる(例えば、「火おこしに使う棒を徹底的に乾かす方法を確立」できれば「火起こしの際の労力が減る」だろうし、「兎の生態に詳しくなる」ことができれば「罠の効率的な設置ができる」だろうし、「自分の所持品を欲する人を探り当てる方法が確立」できれば「所持品を交換しそびれるリスクが減る」あろう)。だが、そのどの場合でも、労働を支える認識は外界の実情により近づこうとすることになる(もちろん、認識はその際に変分ベイズ推論的に変わるので、一時的に外界の実情から遠のくこともありうる)。

カイゼンにおいて、認識とその産物は過去のそれらに強固に拘束されることはない。過去に作られた認識・制度・仕組みなどは必要とあれば撤廃されることになる。このような労働のあり方は、人間が原初の頃より自然と格闘しながら繰り広げてきた労働のオリジナル(=本源的)なあり方だといえるだろう。

このように、エンジニアリングの観点からみれば、労働は驚きの連続と共にある知的創造だと言える。このような「いきいき」とした労働は、社会の中で実現可能であり、それは多くの場所で実践されている{[}\^{}1{]}。その典型が、「アジャイル」が全体に浸透しているような組織だ。アジャイルとは、不確実性と直面する機会を積極的に織り込むことでエンジニアリングのダイナミズムを活性化させて、より良い満足を生みだしていうとする思想のことだ(*5)。そこでは、組織のあらゆる場所での発見が、組織の他の部分へと対話を通じて伝播していく。アジャイルな組織においては、階層的な秩序は存在しうるが、決して硬直的にはなっていない。予想外の出来事としての現実に向き合い、その不確実性を弁証法的に克服しようとる不断の運動が、アジャイルな組織を「いきいき」とした活動的なものにしている。

そこには、労働についての批判的な議論で出てくるような「有無を言わさず従うしかない強制的な労働力の搾取」としての労働の面影はほとんどない{[}\^{}2{]}。そうした労働は、エンジニアリングの産物が「疎外」を生むことによって生じた、労働の二次的な形態でしかないからだ。エンジニアリングの産物は認識を外化したものとしての仕組みや所持品に他ならないが、そうした認識を外化したものは自己自身ではない。そのため、それらは身体に対して予測誤差を生む結果へと必然的に帰結してしまう{[}\^{}3{]}。このとき、その予測誤差を解決するために外化したものをカイゼンするのでなければ、その外化したものはやがて人間を疎外するようになるのだ。

そうした疎外が四つのディスクールをどのようなものに変えるかについては、次章で見ていくことになる。そこでは、四つのディスクールが、構造はそのままに、より抑圧的、あるいは搾取的な性格を持つようになる。だが、この章ではエンジニアリングと四つのディスクールとの関係に焦点を当てることにしよう。

\begin{itemize}
\tightlist
\item
  {[}\^{}1{]}
  ここでの「いきいき」という副詞は、小田中育生氏によるものを用いた。その意味については小田中(2023)を参照。
\item
  {[}\^{}2{]}
  今村(2024)には、労働がそのようなものとして捉えられてきた様子が記されている。
\item
  {[}\^{}3{]} ==ヘーゲル『精神現象学』==
\end{itemize}

\subsubsection{エンジニアリングにおける主人のディスクール}\label{ux30a8ux30f3ux30b8ux30cbux30a2ux30eaux30f3ux30b0ux306bux304aux3051ux308bux4e3bux4ebaux306eux30c7ux30a3ux30b9ux30afux30fcux30eb}

エンジニアリングにおける主人のディスクールとは、新しく確立された視点や問題の枠組み(=\(\textrm{S}_1\))から、さまざまな物事(=\(\textrm{S}_2\))が規定され位置づけなおされていく(=\(\textrm{S}_1\rightarrow\textrm{S}_2\))過程である。そこで主体(=\(\cancel{\textrm{S}}\))は\(\textrm{S}_1\)を確立すること(=\(\uparrow\frac{\textrm{S}_1}{\cancel{\textrm{S}}}\))で不確実性を解消しようとするが、その他方で新たな不確実性が生まれる(=\(\frac{\textrm{S}_2}{a}\downarrow\))。この新たな不確実性には、その視点に立つ限り解消できない部分が含まれる(\(\cancel{\textrm{S}}//a\))(*9)。

\[
\uparrow\frac{\mathrm{S_1}}{\mathrm{\cancel{S}}}\genfrac{}{}{0pt}{}{\longrightarrow}{//}\frac{\mathrm{S_2}}{a}\downarrow
\]

\subsubsection{エンジニアリングにおける大学のディスクール}\label{ux30a8ux30f3ux30b8ux30cbux30a2ux30eaux30f3ux30b0ux306bux304aux3051ux308bux5927ux5b66ux306eux30c7ux30a3ux30b9ux30afux30fcux30eb}

エンジニアリングにおける大学のディスクールとは、既に確立された視点や問題の枠組み(=\(\textrm{S}_1\))に根拠を持つ様々な命題/仕組み/制度など(=\(\uparrow\frac{\textrm{S}_2}{\textrm{S}_1}\))を、\(\textrm{S}_1\)に変更を加えないまま拡張していくことで不確実性を解消していこうとする(=\(\textrm{S}_2\rightarrow a\))過程である。その過程は不徹底に終わるため、残存する予測誤差が主体(=\(\cancel{\textrm{S}}\))を発生させる(=\(\frac{a}{\cancel{\textrm{S}}}\downarrow\))が、このディスクールに立つ限り不確実性の解消は一応作動し続けているため、主体は\(\textrm{S}_1\)に変更を敢えて加えようとはしなくなる(=\(\textrm{S}_1//\cancel{\textrm{S}}\))(*10)。

\[
\uparrow\frac{\mathrm{S_2}}{\mathrm{S_1}}\genfrac{}{}{0pt}{}{\longrightarrow}{//}\frac{a}{\mathrm{\cancel{S}}}\downarrow
\]

\subsubsection{エンジニアリングにおけるヒステリー者のディスクール}\label{ux30a8ux30f3ux30b8ux30cbux30a2ux30eaux30f3ux30b0ux306bux304aux3051ux308bux30d2ux30b9ux30c6ux30eaux30fcux8005ux306eux30c7ux30a3ux30b9ux30afux30fcux30eb}

エンジニアリングにおけるヒステリー者のディスクールとは、

\[
\uparrow\frac{\mathrm{\cancel{S}}}{a}\genfrac{}{}{0pt}{}{\longrightarrow}{//}\frac{\mathrm{S_1}}{\mathrm{S_2}}\downarrow
\]

\subsubsection{エンジニアリングにおける分析家のディスクール}\label{ux30a8ux30f3ux30b8ux30cbux30a2ux30eaux30f3ux30b0ux306bux304aux3051ux308bux5206ux6790ux5bb6ux306eux30c7ux30a3ux30b9ux30afux30fcux30eb}

エンジニアリングにおける分析家のディスクールとは、自身がそれまで依拠していた認識/仕組み/制度など(=\(\textrm{S}_2\))に起因するうまくいかなさ(=\(\frac{a}{\textrm{S}_2}\))が眼前に現れる(=\(a\rightarrow\cancel{\textrm{S}}\))ことで、主体はそのうまくいかなさの解消を目的とした新たな視点や問題の枠組み(=\(\textrm{S}_1\))を生みだすように思考を強いられる(=\(\frac{\cancel{\textrm{S}}}{\textrm{S}_1}\downarrow\))。なお、ここで新しく生み出された\(\textrm{S}_1\)は、それまで依拠されていた\(\textrm{S}_2\)とは整合性を持たない(=\(\textrm{S}_2//\textrm{S}_1\))ため、速やかに主体は主人のディスクールへと移って世界の再構築が行われる(*12)。

\[
\uparrow\frac{a}{\mathrm{S_2}}\genfrac{}{}{0pt}{}{\longrightarrow}{//}\frac{\mathrm{\cancel{S}}}{\mathrm{S_1}}\downarrow
\]

\subsubsection{布石1:エンジニアリングが安楽をもたらすわけではない}\label{ux5e03ux77f3uxff11ux30a8ux30f3ux30b8ux30cbux30a2ux30eaux30f3ux30b0ux304cux5b89ux697dux3092ux3082ux305fux3089ux3059ux308fux3051ux3067ux306fux306aux3044}

さて、こうして四つのディスクールとエンジニアリングとの関係を説明することができた。しかし、この章でも次章に進む前に打っておかなければおかない布石がある。この章の布石は多く、四つもある。それらの布石を打つことで、エンジニアリングについての理解を深めることができるだろう。

まずはじめに、エンジニアリングは「それに従えば安楽が得られる」ような行為ではないという点について見ていく。それは、エンジニアリングは、自らの置かれた境遇をあくまで「より良く」し続けるための方途でしかないからだ。この点をみていくことで、エンジニアリングについての理解を深めることができるだろう。

そもそも、「それに従えば安楽が得られる」ようなユートピアは不可能だ。このことを理解するためには、疎外を避ける方法について考えてみればよい。まず、外化したものを耐えず無化し続けることで「未開」の状態に留まる場合を考えよう。外化したものを無化するためには、外化したもののエントロピーを増大させればよい。したがって、そのためには外化したものを自然の荒ぶる風化作用によって滅びるに任せたり、あるいは外化したものを蕩尽の場で享楽的に破壊し尽くすだけで十分だ{[}\^{}1{]}。それでは、外化したものを蓄積し、組み立てて文明を成立させ、その上で生きる場合はどうだろうか。この場合では、外化したものを安易に無化することは許されない。それでも疎外を避けたければ、外化したものの手綱を握り続け、繰り返しそれを修正し続けるほかない。この議論から分かる通り、生物は「未開か、文明か」の二者択一を強いられており、そこで文明を選んだ生物には「勤勉か、疎外か」の二者択一が押し付けられているわけだ。

ここで述べる勤勉なあり方が、まさしくエンジニアリングに他ならない。これらを実践するためは、「自身が体験した予測誤差に向き合い、それを解消する」というプロセスが欠かせない。そこでは、予測誤差が生じた原因とその解消方法についての文脈に沿った正確な分析が必要不可欠だ。つまり、先達が発見した既知のノウハウを没文脈的に適用するだけでは事態のカイゼンは望むべくもないわけだ{[}\^{}2{]}。この意味で、エンジニアリングは日々の生活の場に密着した具体的な弁証法的運動に他ならない。これらの実践は、「それに従う」ような構えの生き方とは両立しないのだ{[}\^{}3{]}。

疎外にも第十章と第十一章で検討する二つの現れ方があるという論点を先取りして補足すれば、エンジニアリングを柄谷行人がいう「交換様式」になぞらえて考えることもできる。エンジニアリングは、未開社会に見られるような原初的な生活様式(=交換様式A)とも文明社会に見られるような帝国主義(=交換様式B)や資本主義(=交換様式C)とも異なる、交換様式Aを部分的に取り戻しつつもそれを文明社会に適合させた「交換様式D」だと言えるからだ{[}\^{}4{]}。だが、エンジニアリングは、たしかに疎外を退けうるものの、決して未来のユートピア的な在り方ではないという点には留意しなければならない。なぜならば、先に述べた通り、それらは「それに従う」というような知的怠惰を退ける限りで、ここまで見てきたように、既に、そして幅広い場所で手に届く生き方だからだ。

\begin{itemize}
\tightlist
\item
  {[}\^{}1{]}
  ここでは「未開」という言葉をレヴィ=ストロースのいう定常的な「冷たい社会」に対応している。冷たい社会では社会が築いた財が蓄積されず、祝祭などの機会に浪費される結果、社会が一定の状態に保たれる。詳しくはレヴィ=ストロース(1962=1976)を参照。
\item
  {[}\^{}2{]}
  この章の布石2の部分で出している観点は、生産力の最大化よりも最適化を志向する点だけでなく、そこで「何を、どのように、どれだけ生産するかの決定に、労働者たちが能動的に参与する」(同、p.364)ことを重視する点などでも斎藤(2023)にも通じるところがある。ただし、斎藤は同書で「脱成長コミュニズムは無限の経済成長を目指すのを止め、贅沢な消費を促すような部門の生産を減少させるための社会計画と規制を導入する」(同、p.352)と論じた上で乱暴にも「資本の価値増殖のための大量生産は、広告、マーケティング、金融、コンサルティングなどの非エッセンシャルな仕事を増やす。マルクスは、資本主義の発展とともに必然的に増加する、無駄な仕事について、次のように書いている」(同、p.359)や「『ブルシット・ジョブ』(Graeber
  2018)、つまり労働者自身さえも社会にとって無意味だと自覚しているような仕事が蔓延している」と述べている。ここで斎藤は、何らかの仕事を「贅沢」や「無駄」と判断することをあまりにも簡単に捉えすぎていると言わざるを得ない。どの仕事が不要で無駄なのかは、部外者が簡単に判断できるようなものではない。そのような思い込みの元に労働に介入するのは、それこそ悪い意味での「コンサルティング」だと言わざるを得ない。
\item
  {[}\^{}3{]}
  カイゼンをどのように組織に根付かせていけばよいかについては、市谷(2018)を参照。
\item
  {[}\^{}4{]} ==柄谷行人『探求Ⅲ』==
\end{itemize}

\subsubsection{布石2:エンジニアリングは搾取に結びついているとは限らない}\label{ux5e03ux77f3uxff12ux30a8ux30f3ux30b8ux30cbux30a2ux30eaux30f3ux30b0ux306fux643eux53d6ux306bux7d50ux3073ux3064ux3044ux3066ux3044ux308bux3068ux306fux9650ux3089ux306aux3044}

本稿におけるエンジニアリングは、各々の生が抱える予測誤差としての〈物〉に注目し、各々が持つシニフィアンの秩序を変化させるところにその本質があるがゆえに、幅広い場所で実践できるのだった。これは、エンジニアリング(=交換様式D)が未開社会・帝国主義・資本主義(=交換様式A~C)と両立しないわけではないことを意味している。エンジニアリングは、未開社会の中でも帝国主義の中でも資本主義の中でも行われうる。

ここに先の、「エンジニアリングは、世界についての認識を深め、世界への関わり方をカイゼンする」という点を鑑みれば、エンジニアリングは、帝国主義や資本主義が行ってきた自然環境からの持続不可能な搾取について「それらの体制が持つモメントを利用して、それらを解消する」という脱構築の方途を開くものだということが見えてくる。ここでは、文明社会の持続可能性という論点についてカイゼンがどのような役割を果たしうるかをみることにしよう。

文明社会を持続可能なものとするためには、宇宙規模の視点から地球が受け取るエネルギー量を計算し、その中で均衡を保てるような文明を構築する必要がある。それを実現するためには、そして、そこで可能な限り多くの生命を、苦痛を避けつつも崩壊も避けるような仕方で育んでいくためには、コストパフォーマンスに優れた技術の利用が避けられない。コストパフォーマンスという観点は、帝国主義的・資本主義的な搾取を加速させうるという理由で批判の対象となることも多いが、「浪費を避け、資源を大切に丁寧に扱う」という意味では持続可能な社会を構想する上で無視すべからざる観点だという点は認める必要があるだろう。

コストパフォーマンスの観点に立つことで、エネルギー消費量の増加ばかりに依存しない形での満足の追及が可能になる。私たちは、より多様で多くの生命の繁栄を目指すのならば、常に顕現し続ける予想外の出来事へと知的関心を開き、世界の隅々に至るまで勤勉にその場の文脈を読み解きながら、自らの認識をカイゼンし、社会をカイゼンし続ける必要がある。これは、避けられない私たちに課された所与の制約だと言える。

\subsubsection{布石3:特異性への旅としてのエンジニアリング}\label{ux5e03ux77f3uxff13ux7279ux7570ux6027ux3078ux306eux65c5ux3068ux3057ux3066ux306eux30a8ux30f3ux30b8ux30cbux30a2ux30eaux30f3ux30b0}

ところで、そのような「終わりなきカイゼン」としてのエンジニアリングを生きたところで、それが私たちの満足に(疎外を解消し生活を便利にするという点以外で)どう寄与するというのだろうか。三つ目の布石としてこの点についても考えてみよう。この三つ目の布石は、後に論じる資本主義社会に対して私たちがどのように関わるべきかを示すものとなる。

終わりなきカイゼンは、予測誤差の最小化を動力として駆動するのであった。予測誤差の最小化が帰結としてもたらすものは、外界につてのより正確な認識だけではない。予測誤差の最小化は、様々な試行錯誤を通じて身体を多様な環境下に置くことを通じて、脳自身が抱える限界についての認識も増やすことになる。これは、特別なことや神秘的なことを論じているのではない。例えば、友人に誘われて試しに流行りのファッションをしてみたとして、そのファッションが(たとえ他人からは「似あっている」と言われようとも)「なんか、自分らしく感じられない」「なんか、しっくりこない」と思われるとき、それは脳が脳自身の持つ限界に遭遇しているのだ。つまり、様々な場に身を置いてみると脳が何らかの反応を起こすわけだが、その反応はその経験をする以前に脳が予測したものとは多かれ少なかれ異なるのだ。

そして、そのようにして遭遇する限界には何らかの規則性がある。これも、特別な話ではない。タイミングによって多少の変動幅はあるにしろ、人には好き嫌いがあるということだ。ファッションの例でいえば、ある人は清潔感のあるシンプルな服装を着こなすことに「馴染む感じ(=予測誤差の少なさ)」を抱くだろうし、ある人は装飾の多い服装を着こなすことに馴染む感じを抱くだろう。他にも、アウトロー的なラフさのある服装に馴染む人もいれば、そもそも人間の服を着るということがなんともしっくりこないという人だっているかもしれない。ファッションだけでもこれほど多彩な馴染み方があるわけだが、こうした好き嫌いの傾向性というのは生活上のかなり多くのものに存在する。

このように、様々な経験を通じて脳自身が持つ限界をその傾向性とともに浮き上がらせることで、人は好き嫌いという自分自身の輪郭に出会えるようになる。ただし、ただやみくもに様々な環境に身を置くことが特異性への道を開くと思ってはならない。第三章で述べた通り、予測誤差が最小化されるためには、そこにある誤差を最小化の対象として捉えるための精度が必要になるからだ。様々な環境を意識的に引き受けることでその細部に注意し続け、高い精度で自分がどのような感情をそこに感じるのかを追い続けるのでなければ、人は自分自身の輪郭に出会えるようにはならないのだ。

このような自分自身の輪郭に出会いに行く旅を能動的にするのであれば、それは自分自身の満足についての終わりなきカイゼンだと言うことができる。このようにして、終わりなきカイゼンは人の満足に寄与することができるのだ。

さらに付け加えるべきことは、外界についての認識を洗練させて外界としての自然を御するための仕組みや制度を作ること(外界についてのカイゼン)と、好き嫌いのような内面についての認識を洗練させて内面としての自然を御するための仕組みや制度を作ること(内面についてのカイゼン)は、同時に達成されうるということだ。今日では「満足の追及は労働外の時間である余暇に行われる」と考える習慣が広まっているため、労働という外界についてのカイゼンと満足についてのカイゼンは同時に達成されえないと思われるかもしれないが、そうではない。

それは、労働における動機付けや労働環境の改善といった分野も、内面のカイゼンと並走して行われるからだ。まず第一に、同じ目標を達成することが求められている場面でも、その目標を達成することのどこに意欲を感じるかは人によって異なる。意欲を感じるポイントの違いは、その人が注意をもって高い精度で認識できる場所の違いにもつながるのだから、どこに意欲を感じて取り組むかは労働の成果を左右する一大論点だ。続いて第二に、労働の中で自分が苦痛を感じる部分について敏感になり、その状況をカイゼンすることも重要だ。それは、労働に含まれる苦痛を取り除くことができれば、より集中して労働ができるようになり、労働の成果をもより良くすることができるからだ。勤勉さは必ずしも苦痛ではなく、むしろ苦痛を解消し、世界と自己自身についての認識を深め、富と幸福をもたらすのだ。

\subsubsection{布石4:芸術という美的昇華}\label{ux5e03ux77f3uxff14ux82b8ux8853ux3068ux3044ux3046ux7f8eux7684ux6607ux83ef}

このように特異性との関係から理解されたエンジニアリングは、「芸術」についての理解を可能にする。最後の布石として、この点をみていくことにしよう。

残留する対象\(a\)はトラウマ的な体験として反復強迫するが、それを「美的」に解消する試みが「芸術」である(*7)。美的とは、「将来における対象aの発生を防ぐためのエンジニアリング的機能を持たない」という意味である(*8)。

\subsubsection{この章のまとめ}\label{ux3053ux306eux7ae0ux306eux307eux3068ux3081}

あ
