% Options for packages loaded elsewhere
\PassOptionsToPackage{unicode}{hyperref}
\PassOptionsToPackage{hyphens}{url}
%
\documentclass[8pt, a5paper]{ltjsarticle}
\pagestyle{empty}
\title{結論1:人間の不満と満足が現れる\\ダイナミズムのモデル化}
\author{}
\date{}
\usepackage{amsmath,amssymb}
\usepackage{iftex}
\ifPDFTeX
  \usepackage[T1]{fontenc}
  \usepackage[utf8]{inputenc}
  \usepackage{textcomp} % provide euro and other symbols
\else % if luatex or xetex
  \usepackage{unicode-math} % this also loads fontspec
  \defaultfontfeatures{Scale=MatchLowercase}
  \defaultfontfeatures[\rmfamily]{Ligatures=TeX,Scale=1}
\fi
\usepackage{lmodern}
\ifPDFTeX\else
  % xetex/luatex font selection
\fi
% Use upquote if available, for straight quotes in verbatim environments
\IfFileExists{upquote.sty}{\usepackage{upquote}}{}
\IfFileExists{microtype.sty}{% use microtype if available
  \usepackage[]{microtype}
  \UseMicrotypeSet[protrusion]{basicmath} % disable protrusion for tt fonts
}{}
\makeatletter
\@ifundefined{KOMAClassName}{% if non-KOMA class
  \IfFileExists{parskip.sty}{%
    \usepackage{parskip}
  }{% else
    \setlength{\parindent}{0pt}
    \setlength{\parskip}{6pt plus 2pt minus 1pt}}
}{% if KOMA class
  \KOMAoptions{parskip=half}}
\makeatother
\usepackage{xcolor}
\setlength{\emergencystretch}{3em} % prevent overfull lines
\providecommand{\tightlist}{%
  \setlength{\itemsep}{0pt}\setlength{\parskip}{0pt}}
\setcounter{secnumdepth}{-\maxdimen} % remove section numbering
\usepackage{bookmark}
\IfFileExists{xurl.sty}{\usepackage{xurl}}{} % add URL line breaks if available
\urlstyle{same}
\hypersetup{
  hidelinks,
  pdfcreator={LaTeX via pandoc}}

\author{}
\date{}

\begin{document}
\maketitle

\section{目標1:四つのあり方と特異性}\label{ux76eeux6a19uxff11ux56dbux3064ux306eux3042ux308aux65b9ux3068ux7279ux7570ux6027}

本稿のここまでの議論から、人間の満足には四つのあり方があることが分かった。ここでは、その四つのあり方を総括していく。四つのあり方のそれぞれでどのように欲動が解消されるかが、それぞれにおける満足のあり方を規定する。そこで特異性の観点を忘れないようにすることが、後の考察を実りあるものにする上で決定的に重要となる。

\section{根源的な欲動の解消}\label{ux6839ux6e90ux7684ux306aux6b32ux52d5ux306eux89e3ux6d88}

第一のあり方は、\textbf{根源的な欲動}のレベルでの満足である。欲動の解消はあらゆる満足の条件である。その意味では、欲動の満足は第二から第四までのあり方の背後を通底している。欲動の満足は、第二から第四までのあり方は社会的な意味の場で得られるものであるから、非社会的で非意味的な満足だと言うことができる。欲動の満足の具体例としては、「身体を動かすことが(社会的な意味を伴う場合であっても)単にそれ自体で楽しい」とか「声を聞くことが(同じく)単にそれ自体で楽しい」などといったケースが挙げられる。これらは即ち、運動や芸術などによる欲動の解消である。

ここで、\textbf{各人の欲動に備わった特異性}(以下、単に「\textbf{特異性}」と表記)についてて指摘しておくべきことがある。それは、欲動のあり方には人による違いがあり、そして解消困難な欲動を軸に欲望は構築されるということとだ。後に詳しく考察するが、この点を忘れて「新しい物語の『推奨ボーダーライン』」について考察すると、実践的な生き方は導き出せなくなってしまう。

\section{神経症的な欲望の場の展開}\label{ux795eux7d4cux75c7ux7684ux306aux6b32ux671bux306eux5834ux306eux5c55ux958b}

第二のあり方は、神経症的な欲望の場を展開させることで生じる満足である。これは第三と第四のあり方を可能にする\textbf{基本的な力}を行使する満足である。ここでは「その場を秩序付ける根拠となる項が制定され、その根拠に基づいた秩序が形成された後に、その秩序の不完全さが露呈させられ、また新たな根拠となる項が制定される」といった秩序の構築と放棄の繰り返しが行われる。この繰り返しが持つ意味の変動がもたらす満足が、満足の第二のあり方だ。

この第二のあり方は、秩序が持つ不完全性と密接な関係を保ちながら作用するため、その時々に人が抱える欲動とも近しい関係を維持できる「健康的」なあり方だと言うことができる。その具体例には「それまで解けなかったパズルが解けた(非意味的に現れていた対象を意味の体系へと統合できた)」であるとか「広く正しいと思われていた説明のおかしな点を発見できた(意味の体系が孕む非意味的な対象を露呈させることができた)」などのケースが挙げられる。

ここで、第一のあり方の最後に触れた\textbf{特異性}について、第二のあり方がどのように関わるかの関係を付記しておこう。第二のあり方において、人は秩序の構築と放棄の繰り返しを経験する。すると、この繰り返し通じて、人は自身の欲望が様々に姿を変えることと、そこで欲望が見せる振れ幅の中に「核」になって維持されているような部分がある(「〈一〉部分あり(Y'a
d'l'Un)」)ことに気付くことがある。これが「その人が求めずにはいられないもの」であり、その核の中にその人に特有の欲動があるのだ。こうして「それを知り、それを裏切らないようにする」という境涯が開かれることになる。そして、人はこの境涯において正直になり、自身に嘘を付くことをやめ、罪責感との縁を絶つことができるようになるのだ。

\section{神経症的な欲望の場における承認}\label{ux795eux7d4cux75c7ux7684ux306aux6b32ux671bux306eux5834ux306bux304aux3051ux308bux627fux8a8d}

第三のあり方は、神経症的な欲望の場が形成するヒエラルキーの中で充足することによる満足である。そこでは超越的な存在(=象徴的父)が措定され(=父性隠喩)、その存在を根拠としてこの世界のあらゆるものが固定された意味を持つようになる(=ファリックな意味作用)。これは世界の全体に効果をもたらす\textbf{権威主義的な世界観}に依拠した満足だと言える。こうした権威主義的な世界観の、最も強力な形態が絶対的な価値秩序を持つ宗教によって提供される世界観であり、世俗的な形態の一つが自明な文化的価値秩序があるとする世界観であり、もう一つが客観的な真理の存在を前提とする(科学的な態度の前提となるような)世界観である。これらはプレモダンな世界観だと言うことができる。そこでは固定された意味それ自体にアクセスできるのは超越的な存在に限られており、人々はその断片を知ることしかできないのだが、それでも「固定された意味が存在する」と信じることは人に安心をもたらす。そして、その安心の中で特定の物事を達成すると、その達成にも意味が与えられることになり、満足も得られる。これは、〈父〉なる存在により「承認」される満足だということもできる。

こうした満足の具体例としては、宗教的な場合では「特定の人物に対する信仰を持つことを公言した。これは死後の救済に必要な行為である。なので私は死後の救いに近づいた。喜ばしい」であるとか「子孫を残すことで先祖は救われる。私は子孫を残すことができたので、先祖に対する孝行を果たした。喜ばしい」などといったケースが挙げられ、また世俗的な場合では「(社会的に『一人前の大人ならば結婚して子供を作るべきだ』と言われている状況下で)それを達成できた。喜ばしい」であるとか「事件の真相が分からず社会が不安であった状況で、真犯人を見つけ出した。これで社会に安心を取り戻すことができた。そこに貢献できて喜ばしい」などといったケースが挙げられる。

この権威主義的な世界観において、構築された秩序は硬直しており、その基礎となる部分に据えられた根拠を放棄するのが難しくなっている。そうした状況の中では、第二のあり方と\textbf{特異性}との関係について論じた際に触れたような「自身の欲望の振れ幅」を知る機会も抑圧されてしまう。こうして第三のあり方は、人を安心させることによって、人を無知の中に置き去りにしてしまう危険性を持つことになる。

\section{資本主義的な場における不満の排除}\label{ux8cc7ux672cux4e3bux7fa9ux7684ux306aux5834ux306bux304aux3051ux308bux4e0dux6e80ux306eux6392ux9664}

第四のあり方は、権威主義的な世界観において物事の意味や価値を定義していた存在(その最たるものが超越的な存在)が、価値を媒介する項に代替されて不要になることで現れる\textbf{資本主義的な体制}の下で可能となる満足である。このモダンで資本主義的な体制下では、人は自らが持つ価値の媒介を増やすべく、他者に対してその他者持つ欲動を解消する手段を提示し、他者に「その手段が欲しい!」という欲望を抱かせることで、他者が持つ媒介と手段との交換を成立させる。

より資本主義経済の文脈に引きつけた具体例を挙げると、このプロセスは「資本家や労働者は自らが持つ貨幣を増やすべく、他者に対してその他者が持つ不満を解消する商品を提示し、他者に『その商品が欲しい!』という欲望を抱かせることで、他者が持つ貨幣との交換を成立させる」という話においても成立しているといえる(このとき、他者は自身の欲動を欲望にすり替えられることで、消費者へと変化させられている)。こうして人々は、一方では価値の媒介をより多く持とうとすることで他者の不満を発見し解消する運動へと秩序付けられ、他方では自身の不満を解消すべく自身が持つ価値の媒介を他者に差し出し続けることになる。

この資本主義的なプロセスにおいて消費者の位置に置かれた者が晒されている危険は、自身がより深く満足するか否かの熟慮をせずに他者が提示する手段に飛びついてそれを欲望の対象とすることで、欲動を浅いレベルで解消し続けてしまうことにある。もし、そのように振る舞うのであれば、消費者の位置に置かれた者は、第二のあり方のところで述べたような自己理解に至る道を辿ることが困難になってしまう。この意味で、資本主義的なプロセスは人を\textbf{特異性}に向かわせない傾向を持つ。

ただし、資本主義的な体制下での人間の行為には\textbf{特異性}について付け加えておくべき論点がある。それは、資本主義的な体制は、権威主義的な世界観よりも広い範囲の行為を許容しているという点だ。理念的に言えば、権威主義的な世界観の下ではあらゆる行為の意味や価値が固定されており、多くの行動が禁止されている(まさしく、そうした禁止によってプレモダンの秩序は保たれているのだ)。これに対して、理念的には、資本主義的な体制下ではあらゆる行為の意味や価値は貨幣と商品との交換を妨げない限りで許容されている。したがって、自分自身が持ちうる欲望の振れ幅を探究するという行為もまた、資本主義的な体制下では許容されうるのだ。

\end{document}
